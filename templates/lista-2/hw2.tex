\documentclass[leqno]{article}

\usepackage[brazil]{babel} 
\usepackage[utf8]{inputenc}
\usepackage{a4wide}
\setlength{\oddsidemargin}{-0.2in}
% % \setlength{\oddsidemargin}{0.2in}
\setlength{\evensidemargin}{-0.2in}
% % \setlength{\evensidemargin}{0.5in}
% % \setlength{\textwidth}{5.5in}
\setlength{\textwidth}{6.5in}
\setlength{\topmargin}{-1.2in}
\setlength{\textheight}{10in}
\usepackage[]{amsfonts} \usepackage[]{amsmath}
\usepackage[]{amssymb} \usepackage[]{latexsym}
\usepackage{graphicx,color} \usepackage{amsthm}
\usepackage{mathrsfs} \usepackage{url}
\usepackage{cancel} \usepackage{enumerate}
\usepackage{xifthen} \usepackage{tikz}
\usetikzlibrary{automata,arrows,positioning,calc}

\numberwithin{equation}{section}

\setlength{\parindent}{12 pt}

\begin{document}

\newtheorem{teo}{Teorema}[section] \newtheorem*{teo*}{Teorema}
\newtheorem{prop}[teo]{Proposição} \newtheorem*{prop*}{Proposição}
\newtheorem{lema}[teo]{Lemma} \newtheorem*{lema*}{Lema}
\newtheorem{cor}[teo]{Corolário} \newtheorem*{cor*}{Corolário}

\theoremstyle{definition}
\newtheorem{defi}[teo]{Definição} \newtheorem*{defi*}{Definição}
\newtheorem{exem}[teo]{Exemplo} \newtheorem*{exem*}{Exemplo}
\newtheorem{obs}[teo]{Observação} \newtheorem*{obs*}{Observação}
\newtheorem*{hipo}{Hipóteses}
\newtheorem*{nota}{Notação}

\newcommand{\ds}{\displaystyle} \newcommand{\nl}{\newline}
\newcommand{\eps}{\varepsilon} \newcommand{\ssty}{\scriptstyle}
\newcommand{\bE}{\mathbb{E}}
\newcommand{\cB}{\mathcal{B}}
\newcommand{\cF}{\mathcal{F}}
\newcommand{\cA}{\mathcal{A}}
\newcommand{\cM}{\mathcal{M}}
\newcommand{\cD}{\mathcal{D}}
\newcommand{\cN}{\mathcal{N}}
\newcommand{\cL}{\mathcal{L}}
\newcommand{\cLN}{\mathcal{LN}}
\newcommand{\bP}{\mathbb{P}}
\newcommand{\bQ}{\mathbb{Q}}
\newcommand{\bN}{\mathbb{N}}
\newcommand{\bR}{\mathbb{R}}
\newcommand{\bZ}{\mathbb{Z}}

\newcommand{\bfw}{\mathbf{w}}
\newcommand{\bfv}{\mathbf{v}}
\newcommand{\bfu}{\mathbf{u}}

\newenvironment{sol}
{
    \vspace{4mm}
    \noindent\textbf{Resolução:}
    \strut\newline
    \smallskip
    \hspace{-3.5mm}
}
{}

\newcommand{\bvecc}[2]{%
  \begin{bmatrix} #1 \\ #2  \end{bmatrix}
}
\newcommand{\bveccc}[3]{%
  \begin{bmatrix} #1 \\ #2 \\ #3  \end{bmatrix}
}


\title{Álgebra Linear - Lista de Exercícios 2}

\author{escreva seu nome aqui}

\date{}

\maketitle

\begin{enumerate}

\item Ache a matriz de eliminação $E$ que reduz a matriz de Pascal em uma menor:
$$E \begin{bmatrix} 1 & 0 & 0 & 0 \\
1 & 1 & 0 & 0 \\
1 & 2 & 1 & 0 \\
1 & 3 & 3 & 1
\end{bmatrix} = \begin{bmatrix} 1 & 0 & 0 & 0 \\
0 & 1 & 0 & 0 \\
0 & 1 & 1 & 0 \\
0 & 1 & 2 & 1 
\end{bmatrix}.$$
Qual matriz $M$ reduz a matriz de Pascal à matriz identidade?

\begin{sol} 
	% escreva sua solução aqui.  
\end{sol} 

\item Use o método de Gauss-Jordan para achar a inversa da matriz triangular inferior:
$$U = \begin{bmatrix} 1 & a & b  \\
0 & 1 & c  \\
0 & 0 & 1 
\end{bmatrix}.$$

\begin{sol} 
	% escreva sua solução aqui.  
\end{sol} 

\item Para quais valores de $a$ o método de eliminação não dará 3 pivôs?
$$\begin{bmatrix} 
a & 2 & 3  \\
a & a & 4  \\
a & a & a 
\end{bmatrix}.$$

\begin{sol} 
	% escreva sua solução aqui.  
\end{sol} 

\item Verdadeiro ou falso (prove ou forneça um contra-exemplo):

\begin{enumerate}

\item Se $A^2$ está bem definida, então $A$ é quadrada.

\item Se $AB$ e $BA$ estão bem definidas, então $A$ e $B$ são quadradas.

\item Se $AB$ e $BA$ estão bem definidas, então $AB$ e $BA$ são quadradas.

\item Se $AB = B$, então $A = I$.

\end{enumerate}

\begin{sol} 
	\begin{enumerate} 
		\item % escreva aqui. 

		\item % escreva aqui. 

		\item % escreva aqui. 

		\item % escreva aqui. 
	\end{enumerate} 
\end{sol} 

\item Mostre que se $BA = I$ e $AC = I$, então $B=C$.

\begin{sol} 
	% escreva sua solução aqui.  
\end{sol} 

\item Ache uma matriz não-zero $A$ tal que $A^2 = 0$ e uma matriz $B$ com $B^2 \neq 0$ e $B^3 = 0$.

\begin{sol} 
	% escreva sua solução aqui.  
\end{sol} 

\item Ache as inversas de 
$$\begin{bmatrix} 
3 & 2 & 0 & 0 \\
4 & 3 & 0 & 0 \\
0 & 0 & 6 & 5 \\
0 & 0 & 7 & 6
\end{bmatrix} \mbox{ e } 
\begin{bmatrix} 
0 & 0 & 0 & 2 \\
0 & 0 & 3 & 0 \\
0 & 5 & 0 & 0 \\
1 & 0 & 0 & 0
\end{bmatrix}$$

\begin{sol} 
	% escreva sua solução aqui.  
\end{sol} 

\item Verifique que a inversa de $M = I - \bfu \bfv^T$ é dada por $M^{-1} = I + \frac{\bfu\bfv^T}{1 - \bfv^T\bfu}$. Verifique também que a inversa de $N = A - UW^{-1}V$ é dada por $N^{-1} = A^{-1} + A^{-1}U(W - VA^{-1}U)^{-1}VA^{-1}$.

\begin{sol} 
	% escreva sua solução aqui.  
\end{sol} 

\item Sabemos que a matriz de diferenças tem a seguinte inversa
$$L^{-1} = \begin{bmatrix}
1 & 0 & 0\\
-1 & 1 & 0\\
0 & -1 & 1
\end{bmatrix}^{-1} = \begin{bmatrix}
1 & 0 & 0\\
1 & 1 & 0\\
0 & 1 & 1
\end{bmatrix}.$$
Use essa propriedade (e sua versão triangular superior) para achar a inversa de
$$T = \begin{bmatrix}
1 & -1 & 0\\
-1 & 2 & -1\\
0 & -1 & 2
\end{bmatrix}.$$

\textit{Dica: escreva $T$ como produto de duas matrizes.}

\begin{sol} 
	% escreva sua solução aqui.  
\end{sol} 

\item Mostre que $I + BA$ e $I + AB$ são ambas invertíveis ou singulares. Relacione a inversa de $I + BA$ com a inversa de $I + AB$, caso elas existam.

\begin{sol} 
	% escreva sua solução aqui.  
\end{sol}

\item (Bônus) Mostre que se $\alpha_kA^k + \alpha_{k-1}A^{k-1} + \cdots + \alpha_1 A + \alpha_0 I = 0$, com $\alpha_0 \neq 0$, então $A$ é invertível

\begin{sol} 
	% escreva sua solução aqui.  
\end{sol} 
\end{enumerate}













\end{document} 
