\documentclass[leqno]{article}

\usepackage[brazil]{babel} 
\usepackage[utf8]{inputenc}
\usepackage{a4wide}
\setlength{\oddsidemargin}{-0.2in}
% % \setlength{\oddsidemargin}{0.2in}
\setlength{\evensidemargin}{-0.2in}
% % \setlength{\evensidemargin}{0.5in}
% % \setlength{\textwidth}{5.5in}
\setlength{\textwidth}{6.5in}
\setlength{\topmargin}{-1.2in}
\setlength{\textheight}{10in}
\usepackage[]{amsfonts} \usepackage[]{amsmath}
\usepackage[]{amssymb} \usepackage[]{latexsym}
\usepackage{graphicx,color} \usepackage{amsthm}
\usepackage{mathrsfs} \usepackage{url}
\usepackage{cancel} \usepackage{enumerate}
\usepackage{xifthen} \usepackage{tikz} 
\usepackage{float, framed}  
\usetikzlibrary{automata,arrows,positioning,calc}

\numberwithin{equation}{section}

\setlength{\parindent}{12 pt} 

\newfloat{Box}{h}{lob}[section]

% Changes tombstone
% \renewcommand{\qedsymbol}{\textcolor{white}{\rule{1.3ex}{1.3ex}}}

\newenvironment{sol}
{
    \vspace{4mm}
    \noindent\textbf{Resolução:}
    \strut\newline
    \smallskip
    \hspace{-3.5mm}
}
{}
\begin{document}

\newtheorem{teo}{Teorema}[section] \newtheorem*{teo*}{Teorema}
\newtheorem{prop}[teo]{Proposição} \newtheorem*{prop*}{Proposição}
\newtheorem{lema}[teo]{Lemma} \newtheorem*{lema*}{Lema}
\newtheorem{cor}[teo]{Corolário} \newtheorem*{cor*}{Corolário}

\theoremstyle{definition}
\newtheorem{defi}[teo]{Definição} \newtheorem*{defi*}{Definição}
\newtheorem{exem}[teo]{Exemplo} \newtheorem*{exem*}{Exemplo}
\newtheorem{obs}[teo]{Observação} \newtheorem*{obs*}{Observação}
\newtheorem*{hipo}{Hipóteses}
\newtheorem*{nota}{Notação}
% \newtheorem*{sol}{Solução} 

\newcommand{\ds}{\displaystyle} \newcommand{\nl}{\newline}
\newcommand{\eps}{\varepsilon} \newcommand{\ssty}{\scriptstyle}
\newcommand{\bE}{\mathbb{E}}
\newcommand{\cB}{\mathcal{B}}
\newcommand{\cF}{\mathcal{F}}
\newcommand{\cA}{\mathcal{A}}
\newcommand{\cM}{\mathcal{M}}
\newcommand{\cD}{\mathcal{D}}
\newcommand{\cN}{\mathcal{N}}
\newcommand{\cL}{\mathcal{L}}
\newcommand{\cLN}{\mathcal{LN}}
\newcommand{\bP}{\mathbb{P}}
\newcommand{\bQ}{\mathbb{Q}}
\newcommand{\bN}{\mathbb{N}}
\newcommand{\bR}{\mathbb{R}}
\newcommand{\bZ}{\mathbb{Z}}

\newcommand{\bfw}{\mathbf{w}}
\newcommand{\bfv}{\mathbf{v}}
\newcommand{\bfu}{\mathbf{u}}
\newcommand{\bfb}{\mathbf{b}}
\newcommand{\bfx}{\mathbf{x}}
\newcommand{\bfa}{\mathbf{a}}

\newcommand{\bvecc}[2]{%
  \begin{bmatrix} #1 \\ #2  \end{bmatrix}
}
\newcommand{\bveccc}[3]{%
  \begin{bmatrix} #1 \\ #2 \\ #3  \end{bmatrix}
}


\title{Álgebra Linear - Lista de Exercícios 1}

\author{escreva seu nome aqui}

\date{}

\maketitle

\begin{enumerate}

%\item Ache uma combinação linear de $\bfw_1$, $\bfw_2$ e $\bfw_3$ que dê o vetor zero:
%$$\bfw_1 = \bveccc{1}{2}{3}, \bfw_2 = \bveccc{4}{5}{6} \mbox{ e } \bfw_3 = \bveccc{7}{8}{9}.$$
%
%\item Multiplique $\begin{bmatrix}
%1 & 2 & 0 \\
%2 & 0 & 3 \\
%4 & 1 & 1
%\end{bmatrix} \bveccc{3}{-2}{1}$.

\item Quais condições para $y_1, y_2$ e $y_3$ fazem com que os pontos $(0, y_1)$, $(1, y_2)$ e $(2, y_3)$ caiam numa reta?

\begin{sol} 
	 % escreva sua solução aqui. 
\end{sol} 

\item Se $(a,b)$ é um múltiplo de $(c,d)$ e são todos não-zeros, mostre que $(a,c)$ é um múltiplo de $(b,d)$. O que isso nos diz sobre a matriz
$$A = \begin{bmatrix}
a & b \\
c & d
\end{bmatrix}?$$ 

\begin{sol} 
	 % escreva sua solução aqui. 
\end{sol} 

\item Se $\bfw$ e $\bfv$ são vetores unitários, calcule os produtos internos de (a) $\bfv$ e $-\bfv$; (b) $\bfv + \bfw$ e $\bfv - \bfw$; (c) $\bfv - 2\bfw$ e $\bfv + 2\bfw$.

\begin{sol} 
	 % escreva sua solução aqui. 
\end{sol} 

\item Se $\| \bfv \| = 5$ e $\| \bfw \| = 3$, quais são o menor e maior valores possíveis para $\|\bfv - \bfw\|$? E para $\bfv \cdot \bfw$?

\begin{sol} 
	 % escreva sua solução aqui. 
\end{sol} 

\item Considere o desenho dos vetores $\bfw$ e $\bfv$ abaixo. Hachure as regiões definidas pelas combinações lineares $c \bfv + d \bfw$ considerando as seguintes restrições: $c + d = 1$ (não necessariamente positivos), $c,d \in [0,1]$ e $c,d \geq 0$ (note que são três regiões distintas).

\begin{center}
 \begin{tikzpicture}[scale=0.7]
    \draw[->] (-1,0)--(5,0) node[right]{};
    \draw[->] (0,-1)--(0,5) node[above]{};
    
    \draw[black,-stealth,line width = 0.5mm] (0,0) -- (5,2) node[pos=0.5, anchor=north]{$\bfv$};
    
    \draw[black,-stealth,line width = 0.5mm] (0,0) -- (0.7,4)node[pos=0.5, anchor=east]{$\bfw$};
    
    \end{tikzpicture}
\end{center} 

\begin{sol} 
	 % escreva sua solução aqui. 
\end{sol} 

\item É possível que três vetores em $\bR^2$ tenham $\bfu \cdot \bfv < 0$, $\bfv \cdot \bfw < 0$ e $\bfu \cdot \bfw < 0$? Argumente.

\begin{sol} 
	 % escreva sua solução aqui. 
\end{sol} 

\item Sejam $x,y,z$ satisfazendo $x + y + z = 0$. Calcule o ângulo entre os vetores $(x,y,z)$ e $(z,x,y)$.

\begin{sol} 
	 % escreva sua solução aqui. 
\end{sol} 

\item Resolva o sistema linear abaixo:
$$\begin{bmatrix}
1 & 0 & 0\\
1 & 1 & 0\\
1 & 1 & 1
\end{bmatrix} \begin{bmatrix}
x_1\\
x_2\\
x_3
\end{bmatrix} = \begin{bmatrix}
b_1\\
b_2\\
b_3
\end{bmatrix}.$$
Escreva a solução $\bfx$ como uma matriz $A$ vezes o vetor $\bfb$.

\begin{sol} 
	 % escreva sua solução aqui. 
\end{sol} 

\item Repita o problema acima para a matriz:
$$\begin{bmatrix}
-1 & 1 & 0\\
0 & -1 & 1\\
0 & 0 & -1
\end{bmatrix}.$$

\item Considere a equação de recorrência $-x_{i+1} + 2x_i - x_{i-1} = i$ para $i=1,2,3,4$ com $x_0 = x_5 = 0$. Escreva essas equações em notação matricial $A\bfx = \bfb$ e ache $\bfx$.

\begin{sol} 
	% escreva sua solução aqui. 
\end{sol} 

\item (Bônus) Use o seguinte código em \texttt{numpy} para gerar um vetor aleatório $\bfv=$\texttt{numpy.random.normal(size=[3,1])} em $\bR^3$. Fazendo $\bfu = \bfv/\|\bfv\|$ criamos então um vetor unitário aleatório. Crie 30 outros vetores unitários aleatórios $\bfu_j$ (use \texttt{numpy.random.normal(size=[3,30])}). Calcule a média dos produtos internos $|\bfu \cdot \bfu_j$ e compare com o valor exato $\frac{1}{\pi}\int_0^\pi |\cos \theta| d\theta = \frac{2}{\pi}$.

\begin{sol} 
	% escreva sua solução aqui.  
\end{sol} 
\end{enumerate}















\end{document} 
