\documentclass[leqno]{article}

\usepackage[brazil]{babel} \usepackage[utf8]{inputenc}
\usepackage{a4wide}
\setlength{\oddsidemargin}{-0.2in}
% % \setlength{\oddsidemargin}{0.2in}
\setlength{\evensidemargin}{-0.2in}
% % \setlength{\evensidemargin}{0.5in}
% % \setlength{\textwidth}{5.5in}
\setlength{\textwidth}{6.5in}
\setlength{\topmargin}{-1.2in}
\setlength{\textheight}{10in}
\usepackage[]{amsfonts} \usepackage[]{amsmath}
\usepackage[]{amssymb} \usepackage[]{latexsym}
\usepackage{graphicx,color} \usepackage{amsthm}
\usepackage{mathrsfs} \usepackage{url}
\usepackage{cancel} \usepackage[inline]{enumitem}
\usepackage{xifthen} \usepackage{tikz}
\usetikzlibrary{automata,arrows,positioning,calc}

\numberwithin{equation}{section}

\setlength{\parindent}{12 pt}

\newenvironment{sol}
{
    \vspace{4mm}
    \noindent\textbf{Resolução:}
    \strut\newline
    \smallskip
    \hspace{-3.5mm}
}
% Objetos que aparecem *após* o ambiente; 
% nestas configurações, estamos desenhando uma 
% linha horizontal. 
% (você pode, por exemplo, modificar 
% ou remover este elemento gráfico) 
{\noindent\rule{4cm}{.1mm}}

\begin{document}

\newtheorem{teo}{Teorema}[section] \newtheorem*{teo*}{Teorema}
\newtheorem{prop}[teo]{Proposição} \newtheorem*{prop*}{Proposição}
\newtheorem{lema}[teo]{Lemma} \newtheorem*{lema*}{Lema}
\newtheorem{cor}[teo]{Corolário} \newtheorem*{cor*}{Corolário}

\theoremstyle{definition}
\newtheorem{defi}[teo]{Definição} \newtheorem*{defi*}{Definição}
\newtheorem{exem}[teo]{Exemplo} \newtheorem*{exem*}{Exemplo}
\newtheorem{obs}[teo]{Observação} \newtheorem*{obs*}{Observação}
\newtheorem*{hipo}{Hipóteses}
\newtheorem*{nota}{Notação}

\newcommand{\ds}{\displaystyle} \newcommand{\nl}{\newline}
\newcommand{\eps}{\varepsilon} \newcommand{\ssty}{\scriptstyle}
\newcommand{\bE}{\mathbb{E}}
\newcommand{\cB}{\mathcal{B}}
\newcommand{\cF}{\mathcal{F}}
\newcommand{\cA}{\mathcal{A}}
\newcommand{\cM}{\mathcal{M}}
\newcommand{\cD}{\mathcal{D}}
\newcommand{\cN}{\mathcal{N}}
\newcommand{\cL}{\mathcal{L}}
\newcommand{\cLN}{\mathcal{LN}}
\newcommand{\bP}{\mathbb{P}}
\newcommand{\bQ}{\mathbb{Q}}
\newcommand{\bN}{\mathbb{N}}
\newcommand{\bR}{\mathbb{R}}
\newcommand{\bZ}{\mathbb{Z}}

\newcommand{\bfw}{\mathbf{w}}
\newcommand{\bfv}{\mathbf{v}}
\newcommand{\bfu}{\mathbf{u}}
\newcommand{\bfx}{\mathbf{x}}
\newcommand{\bfb}{\mathbf{b}}

\newcommand{\bvecc}[2]{%
  \begin{bmatrix} #1 \\ #2  \end{bmatrix}
}
\newcommand{\bveccc}[3]{%
  \begin{bmatrix} #1 \\ #2 \\ #3  \end{bmatrix}
}


\title{Álgebra Linear - Lista de Exercícios 11 - Simulado}

\author{escreva seu nome aqui} 

\date{}

\maketitle

\begin{enumerate}

%%%%%%%%%%%%%%%%%%%%%%%%%%%%%%%%%%%%%%%%%%%%%%%%%%%%%%%%%
%%%%%%%%%%%%%%%%%%%%%% Exercício 1 %%%%%%%%%%%%%%%%%%%%%%
%%%%%%%%%%%%%%%%%%%%%%%%%%%%%%%%%%%%%%%%%%%%%%%%%%%%%%%%%

\item Verdadeiro ou falso (prove ou dê um contra-exemplo):

\begin{enumerate}

\item Se $A$ é singular, então $AB$ também é singular.

\item O determinante de $A$ é sempre o produto de seus pivôs.

\item O determinante de $A - B$ é $\det A - \det B$.

\item $AB$ e $BA$ tem o mesmo determinante.

\end{enumerate}

\begin{sol} 
	\vspace{-\baselineskip} 
	\begin{enumerate} 
		\item % item (a) 
		\item % item (b) 
		\item % item (c) 
		\item % item (d)
	\end{enumerate}    
\end{sol} 

%%%%%%%%%%%%%%%%%%%%%%%%%%%%%%%%%%%%%%%%%%%%%%%%%%%%%%%%%
%%%%%%%%%%%%%%%%%%%%%% Exercício 2 %%%%%%%%%%%%%%%%%%%%%%
%%%%%%%%%%%%%%%%%%%%%%%%%%%%%%%%%%%%%%%%%%%%%%%%%%%%%%%%%

\item Sejam $u$ e $v$ vetores ortonormais em $\bR^2$ e defina $A = uv^T$. Calcule $A^2$ para descobrir os autovalores de $A$. Verifique que o traço de $A$ é $\lambda_1 + \lambda_2$.

\begin{sol} 
	% escreva sua solução aqui.    
\end{sol} 

%%%%%%%%%%%%%%%%%%%%%%%%%%%%%%%%%%%%%%%%%%%%%%%%%%%%%%%%%
%%%%%%%%%%%%%%%%%%%%%% Exercício 3 %%%%%%%%%%%%%%%%%%%%%%
%%%%%%%%%%%%%%%%%%%%%%%%%%%%%%%%%%%%%%%%%%%%%%%%%%%%%%%%%

\item A matriz $B$ tem autovalores 1 e 2, $C$ tem autovalores 3 e 4 e $D$ tem autovalores 5 e 7 (todas são matrizes $2 \times 2$). Ache os autovalores de $A$:
$$A = \begin{bmatrix}
B & C \\
0 & D
\end{bmatrix}.$$

\begin{sol} 
	% escreva sua solução aqui.    
\end{sol} 

%%%%%%%%%%%%%%%%%%%%%%%%%%%%%%%%%%%%%%%%%%%%%%%%%%%%%%%%%
%%%%%%%%%%%%%%%%%%%%%% Exercício 4 %%%%%%%%%%%%%%%%%%%%%%
%%%%%%%%%%%%%%%%%%%%%%%%%%%%%%%%%%%%%%%%%%%%%%%%%%%%%%%%%

\item Seja $D$ uma matriz $n \times n$ só com 1's em suas entradas. Procure a inversa da matriz $A = I + D$ dentre as matrizes $I + cD$ e ache o número $c$ correto.

\begin{sol} 
	% escreva sua solução aqui.    
\end{sol} 

%%%%%%%%%%%%%%%%%%%%%%%%%%%%%%%%%%%%%%%%%%%%%%%%%%%%%%%%%
%%%%%%%%%%%%%%%%%%%%%% Exercício 5 %%%%%%%%%%%%%%%%%%%%%%
%%%%%%%%%%%%%%%%%%%%%%%%%%%%%%%%%%%%%%%%%%%%%%%%%%%%%%%%%

\item Vamos resolver uma EDO de segunda ordem usando o que aprendemos. Considere $y'' = 5y' + 4y$ com $y(0) = C_1$ e $y'(0) = C_2$. Defina $u_1 = y$ e $u_2 = y'$. Escreva $\bfu'(t) = A\bfu(t)$ e ache a solução da equação.

\begin{sol} 
	% escreva sua solução aqui.    
\end{sol} 

%%%%%%%%%%%%%%%%%%%%%%%%%%%%%%%%%%%%%%%%%%%%%%%%%%%%%%%%%
%%%%%%%%%%%%%%%%%%%%%% Exercício 6 %%%%%%%%%%%%%%%%%%%%%%
%%%%%%%%%%%%%%%%%%%%%%%%%%%%%%%%%%%%%%%%%%%%%%%%%%%%%%%%%

\item Se $A$ é simétrica e todos seus autovalores são iguais a $\lambda$. O que podemos dizer sobre $A$?

\begin{sol} 
	% escreva sua solução aqui.    
\end{sol} 

%%%%%%%%%%%%%%%%%%%%%%%%%%%%%%%%%%%%%%%%%%%%%%%%%%%%%%%%%
%%%%%%%%%%%%%%%%%%%%%% Exercício 7 %%%%%%%%%%%%%%%%%%%%%%
%%%%%%%%%%%%%%%%%%%%%%%%%%%%%%%%%%%%%%%%%%%%%%%%%%%%%%%%%

\item Suponha que $C$ é positiva definida e que $A$ tenha as colunas LI. Mostre que $A^TCA$ é positiva definida.

\begin{sol} 
	% escreva sua solução aqui.    
\end{sol} 

%%%%%%%%%%%%%%%%%%%%%%%%%%%%%%%%%%%%%%%%%%%%%%%%%%%%%%%%%
%%%%%%%%%%%%%%%%%%%%%% Exercício 8 %%%%%%%%%%%%%%%%%%%%%%
%%%%%%%%%%%%%%%%%%%%%%%%%%%%%%%%%%%%%%%%%%%%%%%%%%%%%%%%%

\item Quais são os autovalores de $A$ se ela for similar a $A^{-1}$?

\begin{sol} 
	% escreva sua solução aqui.    
\end{sol} 

%%%%%%%%%%%%%%%%%%%%%%%%%%%%%%%%%%%%%%%%%%%%%%%%%%%%%%%%%
%%%%%%%%%%%%%%%%%%%%%% Exercício 9 %%%%%%%%%%%%%%%%%%%%%%
%%%%%%%%%%%%%%%%%%%%%%%%%%%%%%%%%%%%%%%%%%%%%%%%%%%%%%%%%

\item Suponha que $A$ é quadrada, mostre que $\sigma_1 \geq |\lambda|$, para qualquer autovalor $\lambda$ de $A$, onde $\sigma_1$ é o primeiro valor singular de $A$.

\begin{sol} 
	% escreva sua solução aqui.    
\end{sol} 

%%%%%%%%%%%%%%%%%%%%%%%%%%%%%%%%%%%%%%%%%%%%%%%%%%%%%%%%%
%%%%%%%%%%%%%%%%%%%%%% Exercício 10 %%%%%%%%%%%%%%%%%%%%%%
%%%%%%%%%%%%%%%%%%%%%%%%%%%%%%%%%%%%%%%%%%%%%%%%%%%%%%%%%

\item Ache a decomposição SVD da matriz
$$A = \begin{bmatrix}
1 & 0 & 1 & 0\\
0 & 1 & 0 & 1 \\
\end{bmatrix}.$$

\begin{sol} 
	% escreva sua solução aqui.    
\end{sol} 
\end{enumerate}















\end{document} 
