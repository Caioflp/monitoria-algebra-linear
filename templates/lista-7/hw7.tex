\documentclass[leqno]{article}

\usepackage[brazil]{babel} \usepackage[utf8]{inputenc}
\usepackage{a4wide}
\setlength{\oddsidemargin}{-0.2in}
% % \setlength{\oddsidemargin}{0.2in}
\setlength{\evensidemargin}{-0.2in}
% % \setlength{\evensidemargin}{0.5in}
% % \setlength{\textwidth}{5.5in}
\setlength{\textwidth}{6.5in}
\setlength{\topmargin}{-1.2in}
\setlength{\textheight}{10in}
\usepackage[]{amsfonts} \usepackage[]{amsmath}
\usepackage[]{amssymb} \usepackage[]{latexsym}
\usepackage{graphicx,color} \usepackage{amsthm}
\usepackage{mathrsfs} \usepackage{url}
\usepackage{cancel} \usepackage{enumerate} 
\usepackage{enumitem} 
\usepackage{xifthen} \usepackage{tikz}
\usetikzlibrary{automata,arrows,positioning,calc}

% \numberwithin{equation}{section}

\setlength{\parindent}{12 pt}

\begin{document}

\newtheorem{teo}{Teorema} \newtheorem*{teo*}{Teorema}
\newtheorem{prop}[teo]{Proposição} \newtheorem*{prop*}{Proposição}
\newtheorem{lema}[teo]{Lemma} \newtheorem*{lema*}{Lema}
\newtheorem{cor}[teo]{Corolário} \newtheorem*{cor*}{Corolário}

\theoremstyle{definition}
\newtheorem{defi}[teo]{Definição} \newtheorem*{defi*}{Definição}
\newtheorem{exem}[teo]{Exemplo} \newtheorem*{exem*}{Exemplo}
\newtheorem{obs}[teo]{Observação} \newtheorem*{obs*}{Observação}
\newtheorem*{hipo}{Hipóteses}
\newtheorem*{nota}{Notação}

\newcommand{\ds}{\displaystyle} \newcommand{\nl}{\newline}
\newcommand{\eps}{\varepsilon} \newcommand{\ssty}{\scriptstyle}
\newcommand{\bE}{\mathbb{E}}
\newcommand{\cB}{\mathcal{B}}
\newcommand{\cF}{\mathcal{F}}
\newcommand{\cA}{\mathcal{A}}
\newcommand{\cM}{\mathcal{M}}
\newcommand{\cD}{\mathcal{D}}
\newcommand{\cN}{\mathcal{N}}
\newcommand{\cL}{\mathcal{L}}
\newcommand{\cLN}{\mathcal{LN}}
\newcommand{\bP}{\mathbb{P}}
\newcommand{\bQ}{\mathbb{Q}}
\newcommand{\bN}{\mathbb{N}}
\newcommand{\bR}{\mathbb{R}}
\newcommand{\bZ}{\mathbb{Z}}

\newcommand{\bfw}{\mathbf{w}}
\newcommand{\bfv}{\mathbf{v}}
\newcommand{\bfu}{\mathbf{u}}
\newcommand{\bfx}{\mathbf{x}}
\newcommand{\bfb}{\mathbf{b}}

\newcommand{\bvecc}[2]{%
  \begin{bmatrix} #1 \\ #2  \end{bmatrix}
}
\newcommand{\bveccc}[3]{%
  \begin{bmatrix} #1 \\ #2 \\ #3  \end{bmatrix}
}

\newenvironment{sol} 
{
    \vspace{4mm}
    \noindent\textbf{Resolução:}
    \strut\newline
    \smallskip
    \hspace{-3.5mm} 
} 
% Objetos que aparecem *após* o ambiente. 
% (você pode, por exemplo, modificar, 
% ou remover, a barra horizontal} 
{\noindent\rule{4cm}{.1mm}}

\title{Álgebra Linear - Lista de Exercícios 7}

\author{escreva seu nome aqui}

\date{\today}

\maketitle

\begin{enumerate}

%%%%%%%%%%%%%%%%%%%%%%%%%%%%%%%%%%%%%%%%%%%%%%%%%%%%%%%%%
%%%%%%%%%%%%%%%%%%%%%% Exercício 1 %%%%%%%%%%%%%%%%%%%%%%
%%%%%%%%%%%%%%%%%%%%%%%%%%%%%%%%%%%%%%%%%%%%%%%%%%%%%%%%%

\item Se $AB = 0$, as colunas de $B$ estão em qual espaço fundamental de $A$? E as linhas de $A$ estão em qual espaço fundamental de $B$? É possível que $A$ e $B$ sejam $3 \times 3$ e com posto 2?

\begin{sol} 
	% escreva sua solução aqui.    
\end{sol} 


%%%%%%%%%%%%%%%%%%%%%%%%%%%%%%%%%%%%%%%%%%%%%%%%%%%%%%%%%
%%%%%%%%%%%%%%%%%%%%%% Exercício 2 %%%%%%%%%%%%%%%%%%%%%%
%%%%%%%%%%%%%%%%%%%%%%%%%%%%%%%%%%%%%%%%%%%%%%%%%%%%%%%%% 

\item Se $Ax = b$ e $A^Ty = 0$, temos $y^Tx = 0$ ou $y^Tb=0$?

\begin{sol} 
	% escreva sua solução aqui.    
\end{sol} 


%%%%%%%%%%%%%%%%%%%%%%%%%%%%%%%%%%%%%%%%%%%%%%%%%%%%%%%%%
%%%%%%%%%%%%%%%%%%%%%% Exercício 3 %%%%%%%%%%%%%%%%%%%%%%
%%%%%%%%%%%%%%%%%%%%%%%%%%%%%%%%%%%%%%%%%%%%%%%%%%%%%%%%% 

\item O sistema abaixo não tem solução:
$$\begin{cases}
x + 2y + 2z = 5\\
2x + 2y + 3z = 5\\
3x + 4y + 5z = 9
\end{cases}$$
Ache números $y_1,y_2,y_3$ para multiplicar as equações acima para que elas somem $0=1$. Em qual espaço fundamental o vetor $y$ pertence? Verifique que $y^Tb = 1$. O caso acima é típico e conhecido como a \textit{Alternativa de Fredholm}: ou $Ax = b$ ou $A^Ty = 0$ com $y^Tb = 1$.

\begin{sol} 
	% escreva sua solução aqui.    
\end{sol} 


%%%%%%%%%%%%%%%%%%%%%%%%%%%%%%%%%%%%%%%%%%%%%%%%%%%%%%%%%
%%%%%%%%%%%%%%%%%%%%%% Exercício 4 %%%%%%%%%%%%%%%%%%%%%% 
%%%%%%%%%%%%%%%%%%%%%%%%%%%%%%%%%%%%%%%%%%%%%%%%%%%%%%%%% 

\item Mostre que se $A^TAx = 0$, então $Ax = 0$. O oposto é obviamente verdade e então temos $N(A^TA) = N(A)$.

\begin{sol} 
	% escreva sua solução aqui.    
\end{sol} 


%%%%%%%%%%%%%%%%%%%%%%%%%%%%%%%%%%%%%%%%%%%%%%%%%%%%%%%%%
%%%%%%%%%%%%%%%%%%%%%% Exercício 5 %%%%%%%%%%%%%%%%%%%%%%
%%%%%%%%%%%%%%%%%%%%%%%%%%%%%%%%%%%%%%%%%%%%%%%%%%%%%%%%% 

\item Seja $A$ uma matriz $3 \times 4$ e $B$ uma $4 \times 5$ tais que $AB = 0$. Mostre que $C(B) \subset N(A)$. Além disso, mostre que posto$(A)$ $ + $ posto$(B) \leq 4$.

\begin{sol} 
	% escreva sua solução aqui.    
\end{sol} 


%%%%%%%%%%%%%%%%%%%%%%%%%%%%%%%%%%%%%%%%%%%%%%%%%%%%%%%%% 
%%%%%%%%%%%%%%%%%%%%%% Exercício 6 %%%%%%%%%%%%%%%%%%%%%% 
%%%%%%%%%%%%%%%%%%%%%%%%%%%%%%%%%%%%%%%%%%%%%%%%%%%%%%%%%

\item Sejam $\mathbf{a,b,c,d}$ vetores não-zeros de $\bR^2$.

\begin{enumerate}

\item Quais são as condições sobre esses vetores para que cada um possa ser, respectivamente, base dos espaços $C(A^T)$, $N(A)$, $C(A)$ e $N(A^T)$ para uma dada matriz $A$ que seja $2 \times 2$. \textit{Dica: cada espaço fundamental vai ter somente um desses vetores como base.}

\begin{sol} 
	% escreva sua solução aqui.    
\end{sol} 

\item Qual seria uma matriz $A$ possível?

\begin{sol} 
	% escreva sua solução aqui.    
\end{sol} 

\end{enumerate}


%%%%%%%%%%%%%%%%%%%%%%%%%%%%%%%%%%%%%%%%%%%%%%%%%%%%%%%%% 
%%%%%%%%%%%%%%%%%%%%%% Exercício 7 %%%%%%%%%%%%%%%%%%%%%% 
%%%%%%%%%%%%%%%%%%%%%%%%%%%%%%%%%%%%%%%%%%%%%%%%%%%%%%%%%

\item Ache $S^{\perp}$ para os seguintes conjuntos:

\begin{enumerate}

\item $S = \{0\}$

\item $S = span\{[1,1,1]\}$

\item $S = span\{[1,1,1], [1,1,-1]\}$

\item $S = \{[1,5,1], [2,2,2]\}$. Note que $S$ não é um subespaço, mas $S^\perp$ é.

\end{enumerate}

\begin{sol}
	\vspace{-\baselineskip} 
	\begin{enumerate}    
		\item % item (a)  
		\item % item (b) 
		\item % item (c) 
		\item % item (d)  
	\end{enumerate}    
\end{sol} 


%%%%%%%%%%%%%%%%%%%%%%%%%%%%%%%%%%%%%%%%%%%%%%%%%%%%%%%%%
%%%%%%%%%%%%%%%%%%%%%% Exercício 8 %%%%%%%%%%%%%%%%%%%%%%
%%%%%%%%%%%%%%%%%%%%%%%%%%%%%%%%%%%%%%%%%%%%%%%%%%%%%%%%%

\item Seja $A$ uma matriz $4 \times 3$ formada pela primeiras 3 colunas da matriz identidade $4 \times 4$. Projeta o vetor $b = [1,2,3,4]$ no espaço coluna de $A$. Ache a matriz de projeção $P$.

\begin{sol} 
	% escreva sua solução aqui.    
\end{sol} 


%%%%%%%%%%%%%%%%%%%%%%%%%%%%%%%%%%%%%%%%%%%%%%%%%%%%%%%%%
%%%%%%%%%%%%%%%%%%%%%% Exercício 9 %%%%%%%%%%%%%%%%%%%%%%
%%%%%%%%%%%%%%%%%%%%%%%%%%%%%%%%%%%%%%%%%%%%%%%%%%%%%%%%%

\item Se $P^2 = P$, mostre que $(I - P)^2 = I - P$. Para a matriz $P$ do exercício anterior, em qual subespaço a matriz $I - P$ projeta?

\begin{sol} 
	% escreva sua solução aqui.    
\end{sol} 
\end{enumerate}















\end{document} 
