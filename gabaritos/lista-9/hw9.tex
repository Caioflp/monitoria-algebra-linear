\documentclass[leqno]{article}

\usepackage[brazil]{babel} % \usepackage[latin1]{inputenc}
\usepackage{a4wide}
\setlength{\oddsidemargin}{-0.2in}
% % \setlength{\oddsidemargin}{0.2in}
\setlength{\evensidemargin}{-0.2in}
% % \setlength{\evensidemargin}{0.5in}
% % \setlength{\textwidth}{5.5in}
\setlength{\textwidth}{6.5in}
\setlength{\topmargin}{-1.2in}
\setlength{\textheight}{10in}
\usepackage[]{amsfonts} \usepackage[]{amsmath}
\usepackage[]{amssymb} \usepackage[]{latexsym}
\usepackage{graphicx,color} \usepackage{amsthm}
\usepackage{mathrsfs} \usepackage{url}
\usepackage{cancel} \usepackage[inline]{enumitem}
\usepackage{xifthen} \usepackage{tikz}
\usepackage{mathtools}
\usetikzlibrary{automata,arrows,positioning,calc}

%% Packages

\usepackage{mathtools, amsthm}
% \usepackage{tikz}
% \usetikzlibrary{positioning}
% \usepackage{csquotes}
% \usepackage{hyperref}

%% Environments

\theoremstyle{plain} % default
\newtheorem{teo}{Teorema}[section]
\newtheorem*{teo*}{Teorema}
\newtheorem{lem}{Lema}[section]
\newtheorem{prop}{Proposição}[section]
\newtheorem{cor}[teo]{Corolário}
\newtheorem*{axiom}{Axioma}

\newtheorem*{TAU}{Teorema da Aproximação Universal}
\newtheorem*{Riesz}{Teorema da Representação de Riesz}

\theoremstyle{definition}
\newtheorem{defn}{Definição}[section]
\newtheorem{conj}{Conjectura}[section]
\newtheorem{exmp}{Exemplo}[section]
\newtheorem{rem}{Observação}[section]
\newtheorem*{rem*}{Observação}

\theoremstyle{remark}
\newtheorem*{note}{Nota}
\newtheorem{case}{Caso}


% Macros

\renewcommand{\vec}[1]{\mathbf{#1}}
\renewcommand{\Re}{\text{Re}}

\newcommand{\K}{\mathbb{K}}
\newcommand{\I}{\mathbb{I}}

\DeclarePairedDelimiter{\dotprod}{\langle}{\rangle}

\DeclareMathOperator{\rk}{rk}
\DeclareMathOperator{\intt}{int}
\DeclareMathOperator{\diam}{diam}
\DeclareMathOperator{\rref}{rref}
\DeclareMathOperator{\vspan}{span}
\DeclareMathOperator{\proj}{proj}
\DeclareMathOperator{\lin}{Lin}
\DeclareMathOperator{\supp}{supp}

\newcommand{\func}[3]{#1 : #2 \rightarrow #3}
\newcommand{\R}{\mathbb{R}}
\newcommand{\Z}{\mathbb{Z}}
\newcommand{\N}{\mathbb{N}}
\newcommand{\Q}{\mathbb{Q}}
\newcommand{\rr}{R_{r}}
\newcommand{\tq}{ : }
\newcommand{\mdc}{\text{mdc}}
\newcommand{\mmc}{\text{mmc}}
\newcommand{\defeq}{\vcentcolon=}
\newcommand{\comp}{\mathscr{C}}


%% Upper and Lower Integrals
\newcommand{\loint}[4]{
    \lefteqn{\int_{ #1 }^{ #2 } #3}\lefteqn{\hspace{0.0ex}\rule[-2.25ex]{1.1ex}{.05ex}} \phantom{\int_{ #1 }^{ #2 } #3}\mathrm{d}#4
}
\newcommand{\upint}[4]{
    \lefteqn{\int_{ #1 }^{ #2 } #3 \ }\lefteqn{\hspace{1.2ex}\rule[ 3.35ex]{1.1ex}{.05ex}} \phantom{\int_{ #1 }^{ #2 } #3 \ }\mathrm{d}#4
}

\DeclarePairedDelimiter\ceil{\lceil}{\rceil}
\DeclarePairedDelimiter\floor{\lfloor}{\rfloor}
\DeclarePairedDelimiter\abs{\lvert}{\rvert}%
\DeclarePairedDelimiter\norm{\lVert}{\rVert}%
\DeclareMathOperator{\sen}{sen}

% Swap the definition of \abs* and \norm*, so that \abs
% and \norm resizes the size of the brackets, and the 
% starred version does not.

\makeatletter
\let\oldabs\abs
\def\abs{\@ifstar{\oldabs}{\oldabs*}}
%
\let\oldnorm\norm
\def\norm{\@ifstar{\oldnorm}{\oldnorm*}}
\makeatother


\numberwithin{equation}{section}

\setlength{\parindent}{12 pt}

\newenvironment{sol} 
{
    \vspace{4mm}
    \noindent\textbf{Resolução:}
    \strut\newline
    \smallskip
    \hspace{-3.5mm} 
} 
% Objetos que aparecem *após* o ambiente. 
% (você pode, por exemplo, modificar, 
% ou remover, a barra horizontal} 
{\noindent\rule{4cm}{.1mm}}

\begin{document}

\newcommand{\ds}{\displaystyle} \newcommand{\nl}{\newline}
\newcommand{\eps}{\varepsilon} \newcommand{\ssty}{\scriptstyle}
\newcommand{\bE}{\mathbb{E}}
\newcommand{\cB}{\mathcal{B}}
\newcommand{\cF}{\mathcal{F}}
\newcommand{\cA}{\mathcal{A}}
\newcommand{\cM}{\mathcal{M}}
\newcommand{\cD}{\mathcal{D}}
\newcommand{\cN}{\mathcal{N}}
\newcommand{\cL}{\mathcal{L}}
\newcommand{\cLN}{\mathcal{LN}}
\newcommand{\bP}{\mathbb{P}}
\newcommand{\bQ}{\mathbb{Q}}
\newcommand{\bN}{\mathbb{N}}
\newcommand{\bR}{\mathbb{R}}
\newcommand{\bZ}{\mathbb{Z}}

\newcommand{\bfw}{\boldsymbol{w}}
\newcommand{\bfv}{\boldsymbol{v}}
\newcommand{\bfu}{\boldsymbol{u}}
\newcommand{\bfx}{\boldsymbol{x}}
\newcommand{\bfb}{\boldsymbol{b}}
\newcommand{\bfe}{\boldsymbol{e}}

\newcommand{\bvecc}[2]{%
  \begin{bmatrix} #1 \\ #2  \end{bmatrix}
}
\newcommand{\bveccc}[3]{%
  \begin{bmatrix} #1 \\ #2 \\ #3  \end{bmatrix}
}


\title{Álgebra Linear - Soluções da Lista de Exercícios 9}

\author{Caio Lins e Tiago da Silva}

\date{\today}

\maketitle

\begin{enumerate}

%%%%%%%%%%%%%%%%%%%%%%%%%%%%%%%%%%%%%%%%%%%%%%%%%%%%%%%%%
%%%%%%%%%%%%%%%%%%%%%% Exercício 1 %%%%%%%%%%%%%%%%%%%%%%
%%%%%%%%%%%%%%%%%%%%%%%%%%%%%%%%%%%%%%%%%%%%%%%%%%%%%%%%%

\item Seja $B$ uma matriz $3 \times 3$ com autovalores 0, 1 e 2. Com  essa informação, ache:

\begin{enumerate}

\item o posto de $B$;

\item o determinante de $B^TB$;

\item os autovalores de $B^TB$;

\item os autovalores de $(B^2 + I)^{-1}$.

\end{enumerate}

\begin{sol}

    \begin{enumerate}[label=(\alph*)]
        \item Sabemos que a multiplicidade algébrica (MA) de cada autovalor de \( B \) é \( 1 \).
            Com isso, como a multiplicidade geométrica (MG) de cada autovalor deve ser maior ou igual a \( 1 \) e menor ou igual à MA, concluímos que a MG de cada autovalor de \( B \) é exatamente \( 1 \).
            Portanto, a dimensão do autoespaço correspondente ao autovalor \( 0 \) (que é o núcleo de \( B \)) tem dimensão \( 1 \) e, assim, pelo Teorema do Posto, \( B \) tem posto \( 2 \).

        \item Naturalmente, se \( B \bfx = 0 \), então \( B^{ \transpose } B \bfx = 0 \) e, com isso, o núcleo de \( B \) está contido no núcleo de \( B^{ \transpose } B \).
            Dessa forma, \( \dim N ( B^{ \transpose } B ) \geq 1 \) e, assim, \( \det B^{ \transpose } B = 0 \).

        \item Não é possível saber exatamente quais são os autovalores de \( B^{ \transpose } B \), mas podemos afirmar, com certeza, que \( 0 \) é um deles.

        \item Vamos calcular, primeiro, os autovalores de \( B^2 + I \).
            Sabemos que \( \lambda \) é um autovalor de \( B^2 + I \) se, e somente se, \( \det ( B^2 + I - \lambda I ) = \det ( B^2 - ( \lambda - 1 ) I ) = 0 \).
            Portanto, \( \lambda \) ser autovalor de \( B^2 + I \) é equivalente a \( \lambda - 1 \) ser autovalor de \( B^2 \).
            Porém, é fácil perceber que se \( \alpha \) é autovalor de \( B \), então \( \alpha^2 \) é autovalor de \( B^2 \), com o mesmo autovetor.
            Sendo assim, os três autovalores de \( B^2 \) são \( 0, 1 \) e \( 4 \) e, com isso, os autovalores de \( B^2 + I \) são \( 1, 2 \) e \( 5 \).
            Agora, perceba que \( B^2 + I \) é invertível, pois todos seus autovalores são diferentes de \( 0 \).
            Além disso, observe que se \( \lambda \) é autovalor de \( B^2 + I \), então \( \lambda^{ -1 } \) é autovalor de \( ( B^2 + I )^{ -1 } \) com o mesmo autovetor.
            Portanto, os autovalores de \( ( B^2 + I )^{ -1 } \) são \( 1, 1/2 \) e \( 1/5 \).
    \end{enumerate}
    
\end{sol}

%%%%%%%%%%%%%%%%%%%%%%%%%%%%%%%%%%%%%%%%%%%%%%%%%%%%%%%%%
%%%%%%%%%%%%%%%%%%%%%% Exercício 2 %%%%%%%%%%%%%%%%%%%%%%
%%%%%%%%%%%%%%%%%%%%%%%%%%%%%%%%%%%%%%%%%%%%%%%%%%%%%%%%%

\item Ache os autovalores das seguintes matrizes

\begin{enumerate*}

\item $A = \begin{bmatrix}
1 & 2 & 3\\
0 & 4 & 5 \\
0 & 0 & 6 
\end{bmatrix}$;

\item $B = \begin{bmatrix}
0 & 0 & 1\\
0 & 2 & 0 \\
3 & 0 & 0 
\end{bmatrix}$;

\item $C = \begin{bmatrix}
2 & 2 & 2\\
2 & 2 & 2 \\
2 & 2 & 2 
\end{bmatrix}$.

\end{enumerate*}

\begin{sol}

    \begin{enumerate}[label=(\alph*)]
        \item Calculando \( \det ( A - \lambda I ) \) obtemos
            \begin{equation*}
                ( 1 - \lambda ) ( 4 - \lambda ) ( 6 - \lambda )
            .\end{equation*}
            Portanto, os autovalores de \( A \) são os seus elementos da diagonal.
            Observe que isso é verdade para qualquer matriz triangular.

        \item Calculando \( \det ( B - \lambda I ) \) ficamos com
            \begin{equation*}
                \lambda^2 ( 2 - \lambda ) - 3 ( 2 - \lambda )
                = ( 2 - \lambda ) ( \lambda^2 - 3 )
            .\end{equation*}
            Portanto, os autovalores de \( B \) são \( 2, \sqrt{ 3 } \) e \( - \sqrt{ 3 } \).

        \item Aqui podemos prosseguir mais rapidamente reparando que, como \( C \) tem posto \( 1 \), já sabemos que \( 0 \) é um de seus autovalores, com multiplicidade algébrica pelo menos \( 2 \).
            Como \( C \) não é a matriz nula, deve ser exatamente \( 2 \).
            Além disso, não é difícil reparar que \( ( 1, 1, 1 ) \) é autovetor de \( C \), com autovalor correspondente igual a \( 6 \).
            Dessa forma, os autovalores de \( C \) são \( 0 \) e \( 6 \).
    \end{enumerate}
\end{sol}

%%%%%%%%%%%%%%%%%%%%%%%%%%%%%%%%%%%%%%%%%%%%%%%%%%%%%%%%%
%%%%%%%%%%%%%%%%%%%%%% Exercício 3 %%%%%%%%%%%%%%%%%%%%%%
%%%%%%%%%%%%%%%%%%%%%%%%%%%%%%%%%%%%%%%%%%%%%%%%%%%%%%%%%

\item Descreva todas as matrizes $S$ que diagonalizam as matrizes $A$ e $A^{-1}$:
$$A = \begin{bmatrix}
0 & 4 \\
1 & 2 
\end{bmatrix}.$$

\begin{sol}

    Calculando o polinômio característico de \( A \), obtemos
    \begin{equation*}
        p_{ A } ( \lambda ) = - \lambda ( 2 - \lambda ) - 4 = \lambda^2 - 2 \lambda - 4
    ,\end{equation*}
    cujas raízes são \( \lambda_{ 1 } = 1 + \sqrt{ 5 } \) e \( \lambda_{ 2 } = 1 - \sqrt{ 5 } \).
    Vamos obter os autovetores correspondentes.
    Temos
    \begin{equation*}
        A - \lambda_{ 1 } I =
        \begin{bmatrix}
            - 1 - \sqrt{ 5 } & 4 \\
            1 & 1 - \sqrt{ 5 }
        \end{bmatrix}
    .\end{equation*}
    Perceba que multiplicando a primeira coluna por \( \sqrt{ 5 } - 1 \) e somando com a segunda, eliminamos a segunda entrada e, de fato, a primeira também.
    Logo, \( \bfv_{ 1 } = ( \sqrt{ 5 } - 1, 1 ) \).
    Da mesma forma, temos
    \begin{equation*}
        A - \lambda_{ 2 } I =
        \begin{bmatrix}
            \sqrt{ 5 } - 1 & 4 \\
            1 & 1 + \sqrt{ 5 }
        \end{bmatrix}
    .\end{equation*}
    Analogamente, multiplicando a primeira coluna por \( - 1 - \sqrt{ 5 } \) e somando com a segunda, eliminamos a segunda entrada e, com efeito, a primeira também.
    Portanto, \( \bfv_{ 2 } = ( - 1 - \sqrt{ 5 }, 1 ) \).
    Logo, dados \( \alpha, \beta \in \R \backslash \left\{ 0 \right\} \) pondo
    \begin{equation*}
        S =
        \begin{bmatrix}
            \alpha \bfv_{ 1 } & \beta \bfv_{ 2 }
        \end{bmatrix}
        \text{ e }
        \Lambda =
        \begin{bmatrix}
            \lambda_{ 1 } & 0 \\
            0 & \lambda_{ 2 }
        \end{bmatrix}
    ,\end{equation*}
    temos \( A = S \Lambda S^{ -1 } \) e, ainda, \( A^{ -1 } = S \Lambda^{ -1 } S^{ -1 } \).
    Sendo assim, as matrizes \( S \) que diagonalizam \( A \) e \( A^{ -1 } \) são as matrizes da forma
    \begin{equation*}
        S = 
        \begin{bmatrix}
            \alpha \bfv_{ 1 } & \beta \bfv_{ 2 }
        \end{bmatrix}
    ,\end{equation*}
    com \( \alpha, \beta \in \R \backslash \left\{ 0 \right\} \).

\end{sol}

%%%%%%%%%%%%%%%%%%%%%%%%%%%%%%%%%%%%%%%%%%%%%%%%%%%%%%%%%
%%%%%%%%%%%%%%%%%%%%%% Exercício 4 %%%%%%%%%%%%%%%%%%%%%%
%%%%%%%%%%%%%%%%%%%%%%%%%%%%%%%%%%%%%%%%%%%%%%%%%%%%%%%%%

\item Ache $\Lambda$ e $S$ que diagonalizem $A$
$$A = \begin{bmatrix}
0.6 & 0.9 \\
0.4 & 0.1
\end{bmatrix}.$$
Qual limite de $\Lambda^k$ quando $k \to +\infty$? E o limite de $A^k$?

\begin{sol}

    Utilizando a propriedade de que \( \lambda_{ 1 } + \lambda_{ 2 } = \operatorname{tr } A = 0.7 \) e \( \lambda_{ 1 } \lambda_{ 2 } = \det A = - 0.3 \), concluímos que \( \lambda_{ 1 } = 1 \) e \( \lambda_{ 2 } = - 0.3 \).
    Calculando os núcleos de \( A - \lambda_{ 1 } I \) e \( A - \lambda_{ 2 } I \), obtemos \( \bfv_{ 1 } = ( 9, 4 ) \) e \( \bfv_{ 2 } = ( 1, -1 ) \).
    Portanto, pondo
    \begin{equation*}
        S =
        \begin{bmatrix}
            \bfv_{ 1 } & \bfv_{ 2 }
        \end{bmatrix}
        =
        \begin{bmatrix}
            9 & 1 \\
            4 & -1
        \end{bmatrix}
        \text{ e }
        \Lambda =
        \begin{bmatrix}
            \lambda_{ 1 } & 0 \\
            0 & \lambda_{ 2 }
        \end{bmatrix}
        =
        \begin{bmatrix}
            1 & 0 \\
            0 & - 0.3
        \end{bmatrix}
    ,\end{equation*}
    temos \( A = S \Lambda S^{ -1 } \).
    Observe que, como \( \Lambda \) é diagonal (não vamos nos preocupar muito com os detalhes aqui), temos
    \begin{equation*}
        \Lambda^{ k } = 
        \begin{bmatrix}
            1 & 0 \\
            0 & (- 0.3)^{ k }
        \end{bmatrix}
    .\end{equation*}
    Como \( \abs{ - 0.3 } < 1 \), temos que \( ( - 0.3 )^{ k } \to 0 \) quando \( k \to +\infty \).
    Dessa forma,
    \begin{equation*}
        \lim \Lambda^{ k } =
        \begin{bmatrix}
            1 & 0 \\
            0 & 0
        \end{bmatrix}
    .\end{equation*}
    Com isso, vale que
    \begin{align*}
        \lim A^{ k } &= \lim S \Lambda^{ k } S^{ -1 } \\
                     &= S ( \lim \Lambda^{ k } ) S^{ -1 } \\
                     &=
                     S \begin{bmatrix}
                         1 & 0 \\
                         0 & 0
                     \end{bmatrix}
                     S^{ -1 }
    .\end{align*}
    Calculando \( S^{ -1 } \) pela fórmula para a inversa de matriz \( 2 \times 2 \), obtemos
    \begin{equation*}
        S^{ -1 } =
        \frac{ 1 }{ -13 }
        \begin{bmatrix}
            -1 & -1 \\
            -4 & 9
        \end{bmatrix}
    ,\end{equation*}
    e, assim,
    \begin{equation*}
        \lim A^{ k } = 
         S \begin{bmatrix}
             1 & 0 \\
             0 & 0
         \end{bmatrix}
         S^{ -1 }
         =
         \frac{ 1 }{ 13 }
         \begin{bmatrix}
             9 & 9 \\
             4 & 4
         \end{bmatrix}
    .\end{equation*}

\end{sol}

%%%%%%%%%%%%%%%%%%%%%%%%%%%%%%%%%%%%%%%%%%%%%%%%%%%%%%%%%
%%%%%%%%%%%%%%%%%%%%%% Exercício 5 %%%%%%%%%%%%%%%%%%%%%%
%%%%%%%%%%%%%%%%%%%%%%%%%%%%%%%%%%%%%%%%%%%%%%%%%%%%%%%%%

\item Seja $Q(\theta)$ a matriz de rotação do ângulo $\theta$ em $\bR^2$:
$$Q(\theta) = \begin{bmatrix}
\cos \theta & -\mbox{sen} \theta \\
\mbox{sen} \theta & \cos \theta
\end{bmatrix}.$$
Ache os autovalores e autovetores de $Q(\theta)$ (eles podem ser complexos).

\begin{sol}

    Seja \( p_{ \theta } ( \lambda ) \) o polinômio característico de \( Q ( \theta ) \).
    Ou seja, temos
    \begin{equation*}
        p_{ \theta } ( \lambda ) = ( \cos \theta - \lambda )^2 + ( \sen \theta )^2
        = \lambda^2 - 2 ( \cos \theta ) \lambda + 1
    .\end{equation*}
    Pela fórmula quadrática, suas raízes são
    \begin{equation*}
        \lambda_{ 1 } ( \theta ) = \cos \theta + i \abs{ \sen \theta }
        \text{ e }
        \lambda_{ 2 } ( \theta ) = \cos \theta - i \abs{ \sen \theta }
    .\end{equation*}
    Agora observe que, se \( \theta \in [0, \pi) \) então \( \lambda_{ 1 } ( \theta ) = \cos \theta + i \sen \theta \) e \( \lambda_{ 2 } ( \theta ) = \cos \theta - i \sen \theta \) e, se \( \theta \in [\pi, 2 \pi) \), então \( \lambda_{ 1 } ( \theta ) = \cos \theta - i \sen \theta \) e \( \lambda_{ 2 } ( \theta ) = \cos \theta + i \sen \theta \).
    De qualquer forma, os autovalores são \( \cos \theta + i \sen \theta \) e \( \cos \theta - i \sen \theta \).
    Convencionemos, portanto, \( \lambda_{ 1 } ( \theta ) = \cos \theta + i \sen \theta \) e \( \lambda_{ 2 } ( \theta ) = \cos \theta - i \sen \theta \).

    Encontremos agora os autovetores.
    Temos
    \begin{equation*}
        Q ( \theta ) - \lambda_{ 1 } ( \theta ) I =
        \begin{bmatrix}
            - i \sen \theta & - \sen \theta \\
            \sen \theta & - i \sen \theta
        \end{bmatrix}
    .\end{equation*}
    Percebe-se facilmente que \( \bfv_{ 1 } = ( i, 1 ) \) pertence ao núcleo dessa matriz.
    Da mesma forma, temos
    \begin{equation*}
        Q ( \theta ) - \lambda_{ 2 } ( \theta ) I =
        \begin{bmatrix}
            i \sen \theta & - \sen \theta \\
            \sen \theta & i \sen \theta
        \end{bmatrix}
    .\end{equation*}
    Novamente, percebemos sem dificuldade que \( \bfv_{ 2 } = ( i, -1 ) \) pertence ao núcleo dessa matriz.

\end{sol}

%%%%%%%%%%%%%%%%%%%%%%%%%%%%%%%%%%%%%%%%%%%%%%%%%%%%%%%%%
%%%%%%%%%%%%%%%%%%%%%% Exercício 6 %%%%%%%%%%%%%%%%%%%%%%
%%%%%%%%%%%%%%%%%%%%%%%%%%%%%%%%%%%%%%%%%%%%%%%%%%%%%%%%%

\item Suponha que $A$ e $B$ são duas matrizes $n \times n$ com os mesmo autovalores $\lambda_1, \ldots, \lambda_n$ e os mesmos autovetores $\bfx_1, \ldots, \bfx_n$. Suponha ainda que $\bfx_1, \ldots, \bfx_n$ são LI. Prove que $A = B$.

\begin{sol}

    Vamos provar o resultado mais geral:
    \begin{prop*}
        Se \( A \) e \( B \) são matrizes \( n \times n \), \( \mathcal{A} = \left\{ \bfv_{ 1 }, \dots, \bfv_{ n } \right\} \subset \R^{ n } \) um conjunto de vetores L.I. e, ainda, \( A \bfv_{ i } = B \bfv_{ i } \) para todo \( i \), então \( A = B \).
    \end{prop*}
    \begin{proof}
        Como \( \mathcal{A} \) é formado por \( n \) vetores L.I., compõe uma base para \( \R^{ n } \).
        Sendo \( \mathcal{B} = \left\{ \bfe_{ 1 }, \dots, \bfe_{ n } \right\} \) a base canônica do \( \R^{ n } \), sabemos que, dado \( \bfe_{ j } \in \mathcal{B} \), existem escalares \( \alpha_{ 1, j }, \dots, \alpha_{ n, j } \) tais que
        \begin{equation*}
            \bfe_{ j } = \sum_{ i=1 }^{ n } \alpha_{ i, j } \bfv_{ i }
        .\end{equation*}
        Agora, observe que para todo \( j = 1, \dots, n \) temos
        \begin{align*}
            A \bfe_{ j } &=
            A \left(
                \sum_{ i=1 }^{ n } \alpha_{ i, j } \bfv_{ i }
            \right) \\
                         &= \sum_{ i=1 }^{ n } \alpha_{ i, j } A \bfv_{ i } \\
                         &= \sum_{ i=1 }^{ n } \alpha_{ i, j } B \bfv_{ i } \\
                         &= B \left(
                             \sum_{ i=1 }^{ n } \alpha_{ i, j } \bfv_{ i }
                         \right) \\
                         &= B \bfe_{ j }
        .\end{align*}
        Sendo assim, a \( j \)-ésima coluna de \( A \) é igual à \( j \)-ésima coluna de \( B \) para todo \( j = 1, \dots, n \), ou seja, \( A = B \).
    \end{proof}
    Para a questão, basta tomar \( \mathcal{A} = \left\{ \bfx_{ 1 }, \dots, \bfx_{ n } \right\} \).
    
\end{sol}

%\item Sejam $B$, $C$ e $D$ matrizes $2 \times 2$ com auto-valores $\{1,2\}$, $\{3,4\}$ e $\{5, 7\}$, respectivamente. Quais são os autovalores da matriz $4 \times 4$:
%$$A = \begin{bmatrix}
%B & C \\
%0 & D
%\end{bmatrix}.$$

%%%%%%%%%%%%%%%%%%%%%%%%%%%%%%%%%%%%%%%%%%%%%%%%%%%%%%%%%
%%%%%%%%%%%%%%%%%%%%%% Exercício 7 %%%%%%%%%%%%%%%%%%%%%%
%%%%%%%%%%%%%%%%%%%%%%%%%%%%%%%%%%%%%%%%%%%%%%%%%%%%%%%%%

\item Seja $Q(\theta)$ como na Questão 5. Diagonalize $Q(\theta)$ e mostre que
$$Q(\theta)^n = Q(n\theta).$$

\begin{sol}

    Pelo desenvolvimento da Questão \( 5 \), sabemos que pondo
    \begin{equation*}
        S =
        \begin{bmatrix}
            i & i \\
            1 & -1
        \end{bmatrix}
        \text{ e }
        \Lambda =
        \begin{bmatrix}
            \cos \theta + i \sen \theta & 0 \\
            0 & \cos \theta - i \sen \theta
        \end{bmatrix}
    ,\end{equation*}
    temos \( Q ( \theta ) = S \Lambda S^{ -1 } \).
    Dessa forma,
    \begin{align*}
        Q ( \theta )^{ n } &= ( S \Lambda S^{ -1 } )^{ n } \\
                           &= S ( \Lambda^{ n } ) S^{ -1 } \\
                           &=
                           S
                           \begin{bmatrix}
                                ( \cos \theta + i \sen \theta )^{ n } & 0 \\
                                0 & ( \cos \theta - i \sen \theta )^{ n }
                           \end{bmatrix}
                           S^{ -1 }
    .\end{align*}
    Pela Fórmula de De Moivre, sabemos que \( ( \cos \theta \pm i \sen \theta )^{ n } = \cos n \theta \pm i \sen n \theta \).
    Portanto,
    \begin{equation*}
        S
        \begin{bmatrix}
             ( \cos \theta + i \sen \theta )^{ n } & 0 \\
             0 & ( \cos \theta - i \sen \theta )^{ n }
        \end{bmatrix}
        S^{ -1 }
        =
        S
        \begin{bmatrix}
             \cos n \theta + i \sen n \theta & 0 \\
             0 & \cos n \theta - i \sen n \theta
        \end{bmatrix}
        S^{ -1 }
        = Q ( n \theta )
    .\end{equation*}
\end{sol}

%%%%%%%%%%%%%%%%%%%%%%%%%%%%%%%%%%%%%%%%%%%%%%%%%%%%%%%%%
%%%%%%%%%%%%%%%%%%%%%% Exercício 8 %%%%%%%%%%%%%%%%%%%%%%
%%%%%%%%%%%%%%%%%%%%%%%%%%%%%%%%%%%%%%%%%%%%%%%%%%%%%%%%%

\item Suponha que $G_{k+2}$ é a média dos dois números anteriores $G_{k+1}$ e $G_k$. Ache a matriz $A$ que faz com que
$$\begin{bmatrix}
G_{k+2}\\
G_{k+1}\end{bmatrix} = A \begin{bmatrix}
G_{k+1}\\
G_k\end{bmatrix}.$$

\begin{enumerate}

\item Ache os autovalores e autovetores de $A$;

\item Ache o limite de $A^n$ quando $n \to +\infty$;

\item Mostre que $G_n$ converge para $2/3$ quando $G_0 = 0$ e $G_1 = 1$.

\end{enumerate}

\begin{sol}

    \begin{enumerate}[label=(\alph*)]
        \item Não é difícil perceber que devemos ter
            \begin{equation*}
                A =
                \begin{bmatrix}
                    \frac{ 1 }{ 2 } & \frac{ 1 }{ 2 } \\
                    1 & 0
                \end{bmatrix}
            .\end{equation*}
            Pelo traço e determinante de \( A \), temos \( \lambda_{ 1 } = 1 \) e \( \lambda_{ 2 } = - \frac{ 1 }{ 2 } \).
            Calculando os núcleos de \( A - \lambda_{ 1 } I \) e \( A - \lambda_{ 2 } I \), obtemos \( \bfv_{ 1 } = ( 1, 1 ) \) e \( \bfv_{ 2 } = ( 1 , - 2 ) \).

        \item Pondo
            \begin{equation*}
                S =
                \begin{bmatrix}
                    1 & 1 \\
                    1 & -2
                \end{bmatrix}
                \text{ e }
                \Lambda =
                \begin{bmatrix}
                    1 & 0 \\
                    0 & - \frac{ 1 }{ 2 }
                \end{bmatrix}
            \end{equation*}
            temos \( A = S \Lambda S^{ -1 } \), de modo que
            \begin{align*}
                \lim A^{ n } &= \lim ( S \Lambda S^{ -1 } )^{ n } \\
                             &= \lim S \Lambda^{ n } S^{ -1 } \\
                             &= S ( \lim \Lambda^{ n } ) S^{ -1 }
            .\end{align*}
            Analogamente à Questão 4, temos
            \begin{equation*}
                \lim \Lambda^{ n } =
                \begin{bmatrix}
                    1 & 0 \\
                    0 & 0
                \end{bmatrix}
            .\end{equation*}
            Calculando \( S^{ -1 } \) ficamos com
            \begin{equation*}
                S^{ -1 } =
                \frac{ 1 }{ 3 }
                \begin{bmatrix}
                    2 & 1 \\
                    1 & -1
                \end{bmatrix}
            .\end{equation*}
            Assim,
            \begin{equation*}
                S ( \lim \Lambda^{ n } ) S^{ -1 }
                =
                \frac{ 1 }{ 3 }
                \begin{bmatrix}
                    1 & 1 \\
                    1 & -2
                \end{bmatrix}
                \begin{bmatrix}
                    1 & 0 \\
                    0 & 0
                \end{bmatrix}
                \begin{bmatrix}
                    2 & 1 \\
                    1 & -1
                \end{bmatrix}
                =
                \frac{ 1 }{ 3 }
                \begin{bmatrix}
                    2 & 1 \\
                    2 & 1
                \end{bmatrix}
            .\end{equation*}

        \item Seja \( G = \lim G_{ n } \) quando \( n \to \infty \).
            Observe que \( G = \lim G_{ n+1 } = \lim G_{ n+2 } \).
            Com isso,
            \begin{align*}
                \begin{bmatrix}
                    G \\
                    G
                \end{bmatrix}
                &=
                \begin{bmatrix}
                    \lim G_{ n+2 } \\
                    \lim G_{ n+1 }
                \end{bmatrix} \\
                &= \lim
                \begin{bmatrix}
                    G_{ n+2 } \\
                    G_{ n+1 }
                \end{bmatrix} \\
                &= \lim \left(
                    A^{ n+1 }
                    \begin{bmatrix}
                        G_{ 1 } \\
                        G_{ 0 }
                    \end{bmatrix}
                \right) \\
                &= ( \lim A^{ n+1 } )
                \begin{bmatrix}
                    G_{ 1 } \\
                    G_{ 0 }
                \end{bmatrix} \\
                &= \frac{ 1 }{ 3 }
                \begin{bmatrix}
                    2 & 1 \\
                    2 & 1
                \end{bmatrix}
                \begin{bmatrix}
                    G_{ 1 } \\
                    G_{ 0 }
                \end{bmatrix} \\
                &=
                \begin{bmatrix}
                    \frac{ 2 G_{ 1 } + G_{ 0 } }{ 3 } \\
                    \frac{ 2 G_{ 1 } + G_{ 0 } }{ 3 }
                \end{bmatrix}
            .\end{align*}
            Sendo assim, \( G = ( 2 G_{ 1 } + G_{ 0 } ) / 3 \).
            Se \( G_{ 0 } = 0 \) e \( G_{ 1 } = 1 \), então \( G = 2/3 \).
    \end{enumerate}
    
\end{sol}

%%%%%%%%%%%%%%%%%%%%%%%%%%%%%%%%%%%%%%%%%%%%%%%%%%%%%%%%%
%%%%%%%%%%%%%%%%%%%%%% Exercício 9 %%%%%%%%%%%%%%%%%%%%%%
%%%%%%%%%%%%%%%%%%%%%%%%%%%%%%%%%%%%%%%%%%%%%%%%%%%%%%%%%

\item Ache a solução do sistema de EDOs usando o método de diagonalização:
$$\begin{cases}
u_1'(t) = 8u_1(t) + 3u_2(t),\\
u_2'(t) = 2u_1(t) + 7u_2(t),
\end{cases}$$
onde $u(0) = (5, 10)$.

\begin{sol}

    Pondo
    \begin{equation*}
        \bfu ( t ) =
        \begin{bmatrix}
            u_{ 1 } ( t ) \\
            u_{ 2 } ( t )
        \end{bmatrix}
    ,\end{equation*}
    o sistema de EDOs pode ser reescrito como
    \begin{equation*}
        \bfu'( t ) = A \bfu ( t )
    ,\end{equation*}
    onde
    \begin{equation*}
        A =
        \begin{bmatrix}
            8 & 3 \\
            2 & 7
        \end{bmatrix}
    ,\end{equation*}
    com
    \begin{equation*}
        \bfu ( 0 ) = ( 5, 10 )
    .\end{equation*}
    Como visto em aula, a solução desses sistema é dada por
    \begin{equation*}
        \bfu ( t ) = e^{ At } \bfu ( 0 )
    .\end{equation*}
    Para calcular \( e^{ At } \) vamos, primeiro, diagonalizar \( A \).
    Não estou com paciência para fazer isso pela vigésima vez nessa lista, então aceite que temos \( A = S \Lambda S^{ -1 } \), onde
    \begin{equation*}
        S =
        \begin{bmatrix}
            3 & 1 \\
            2 & -1
        \end{bmatrix}
        \text{ e }
        \Lambda =
        \begin{bmatrix}
            10 & 0 \\
            0 & 5
        \end{bmatrix}
    .\end{equation*}
    Com isso,
    \begin{equation*}
        e^{ At } = S e^{ \Lambda t } S^{ -1 } =
        \frac{ 1 }{ 5 }
        \begin{bmatrix}
            3 e^{ 10 t } + 2 e^{ 5 t } & 3 e^{ 10 t } - 3 e^{ 5 t } \\
            2 ( e^{ 10 t } - e ^{ 5t } ) & 2 e^{ 10t } + 3  e^{ 5t }
        \end{bmatrix}
    .\end{equation*}
    Dessa forma,
    \begin{equation*}
        \bfu ( t ) = e^{ At } \bfu ( 0 ) =
        \begin{bmatrix}
            9 e^{ 10 t } - 4 e^{ 5 t } \\
            6 e^{ 10 t } - 4 e^{ 5 t }
        \end{bmatrix}
    .\end{equation*}
\end{sol}

%%%%%%%%%%%%%%%%%%%%%%%%%%%%%%%%%%%%%%%%%%%%%%%%%%%%%%%%%
%%%%%%%%%%%%%%%%%%%%%% Exercício 10 %%%%%%%%%%%%%%%%%%%%%
%%%%%%%%%%%%%%%%%%%%%%%%%%%%%%%%%%%%%%%%%%%%%%%%%%%%%%%%%

\item Seja \( \mathcal{F} ( \R; \R ) \) o espaço vetorial das funções reais de uma variável real.
    Considere em \( \mathcal{F} ( \R; \R ) \) o subespaço
    \begin{equation*}
        S \defeq \vspan \left\{ e^{ 2x } \sen x, e^{ 2x } \cos x, e^{ 2x } \right\}
    .\end{equation*}
    e o operador linear \( D : S \to S \) definido por \( D ( f ) = f' \).
    Considere, ainda, as funções \( f_{ 1 } ( x ) = e^{ 2x } \sen x, f_{ 2 } ( x ) = e^{ 2x } \cos x \) e \( f_{ 3 } ( x ) = e^{ 2x } \) em \( \mathcal{F} ( \R; \R ) \).
    Determine:
    \begin{enumerate}[label=(\alph*)]
        \item a matriz de \( D \) em relação à base \( \mathcal{B} = \left\{ f_{ 1 }, f_{ 2 }, f_{ 3 } \right\} \).
            Lembre-se de que, dada a base \( \mathcal{B} \), podemos enxergar os elementos de \(  \) como vetores em \( \R^{ 3 } \).
            Por exemplo:
            \begin{equation*}
                ( 1, 2, 3 )_{ \mathcal{B} } = f_{ 1 } + 2f_{ 2 } + 3f_{ 3 }
            .\end{equation*}
        \item os autovalores de \( D \) e as funções de \( S \) que são autovetores de \( D \).
    \end{enumerate}

\begin{sol}

    \begin{enumerate}[label=(\alph*)]
        \item Para obter a matriz \( [D]_{ \mathcal{B} } \) de \( D \) com relação à base \( \mathcal{B} \), vamos calcular \( D ( f_{ i } ) \) para cada \( f_{ i } \in \mathcal{B} \):
            \begin{equation*}
                D ( f_{ 1 } ) ( x ) = f_{ 1 }' ( x ) = 2 e^{ 2x } \sen x + e^{ 2x } \cos x = 2 f_{ 1 } ( x ) + f_{ 2 } ( x )
            .\end{equation*}
            Portanto, a primeira coluna de \( [D]_{ \mathcal{B} } \) é dada por \( D ( f_{ 1 } ) = 2 f_{ 1 } + f_{ 2 } = ( 2, 1, 0 )_{ \mathcal{B} } \).
            Analogamente, temos
            \begin{align*}
                D ( f_{ 2 } ) ( x ) &= f_{ 2 }' ( x ) = - e^{ 2x } \sen x + 2 e^{ 2x } \cos x = - f_{ 1 } ( x ) + f_{ 2 } ( x ) \\
                D ( f_{ 3 } ) ( x ) &= f_{ 3 }' ( x ) = 2 e^{ 2x } = 2 f_{ 3 } ( x )
            .\end{align*}
            Portanto, \( D ( f_{ 2 } ) = - f_{ 1 } + 2 f_{ 2 } = ( - 1, 2, 0 )_{ \mathcal{B} } \) e \( D ( f_{ 3 } ) = 2 f_{ 3 } = ( 0, 0, 2 )_{ \mathcal{B} } \).

            Sendo assim, temos
            \begin{equation*}
                [D]_{ \mathcal{B} } =
                \begin{bmatrix}
                    2 & -1 & 0 \\
                    1 & 2 & 0 \\
                    0 & 0 & 2
                \end{bmatrix}
            .\end{equation*}

        \item Fazendo as contas, obtemos que os autovalores de \( [D]_{ \mathcal{B} } \) são \( 2, 2 + i \) e \( 2 - i \).
            Fazendo mais contas ainda, concluímos que os autovetores correspondentes são \( ( 0, 0, 1 )_{ \mathcal{B} }, ( i, 1, 0 )_{ \mathcal{B} } \) e \( ( 1, i, 0 )_{ \mathcal{B} } \).
            Ou seja, as funções que são autovetores de \( D \) são \( g_{ 1 } ( x ) = e^{ 2x } \), \( g_{ 2 } ( x ) = i e^{ 2x } \sen x + e^{ 2x } \cos x \) e \( g_{ 3 } ( x ) = e^{ 2x } \sen x + i e^{ 2x } \cos x \).
    \end{enumerate}
\end{sol}
\end{enumerate}
\end{document} 
