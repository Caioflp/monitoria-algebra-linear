\documentclass[leqno]{article}

\usepackage[brazil]{babel} % \usepackage[utf8]{inputenc}
\usepackage{a4wide}
\setlength{\oddsidemargin}{-0.2in}
% % \setlength{\oddsidemargin}{0.2in}
\setlength{\evensidemargin}{-0.2in}
% % \setlength{\evensidemargin}{0.5in}
% % \setlength{\textwidth}{5.5in}
\setlength{\textwidth}{6.5in}
\setlength{\topmargin}{-1.2in}
\setlength{\textheight}{10in}
\usepackage[]{amsfonts} \usepackage[]{amsmath}
\usepackage[]{amssymb} \usepackage[]{latexsym}
\usepackage{graphicx,color} \usepackage{amsthm}
\usepackage{mathrsfs} \usepackage{url}
\usepackage{cancel} \usepackage{enumerate} 
\usepackage{enumitem} 
\usepackage{xifthen} \usepackage{tikz}
\usetikzlibrary{automata,arrows,positioning,calc}

% \numberwithin{equation}{section}

\setlength{\parindent}{12 pt}

%% Packages

\usepackage{mathtools, amsthm}
% \usepackage{tikz}
% \usetikzlibrary{positioning}
% \usepackage{csquotes}
% \usepackage{hyperref}

%% Environments

\theoremstyle{plain} % default
\newtheorem{teo}{Teorema}[section]
\newtheorem*{teo*}{Teorema}
\newtheorem{lem}{Lema}[section]
\newtheorem{prop}{Proposição}[section]
\newtheorem{cor}[teo]{Corolário}
\newtheorem*{axiom}{Axioma}

\newtheorem*{TAU}{Teorema da Aproximação Universal}
\newtheorem*{Riesz}{Teorema da Representação de Riesz}

\theoremstyle{definition}
\newtheorem{defn}{Definição}[section]
\newtheorem{conj}{Conjectura}[section]
\newtheorem{exmp}{Exemplo}[section]
\newtheorem{rem}{Observação}[section]
\newtheorem*{rem*}{Observação}

\theoremstyle{remark}
\newtheorem*{note}{Nota}
\newtheorem{case}{Caso}


% Macros

\renewcommand{\vec}[1]{\mathbf{#1}}
\renewcommand{\Re}{\text{Re}}

\newcommand{\K}{\mathbb{K}}
\newcommand{\I}{\mathbb{I}}

\DeclarePairedDelimiter{\dotprod}{\langle}{\rangle}

\DeclareMathOperator{\rk}{rk}
\DeclareMathOperator{\intt}{int}
\DeclareMathOperator{\diam}{diam}
\DeclareMathOperator{\rref}{rref}
\DeclareMathOperator{\vspan}{span}
\DeclareMathOperator{\proj}{proj}
\DeclareMathOperator{\lin}{Lin}
\DeclareMathOperator{\supp}{supp}

\newcommand{\func}[3]{#1 : #2 \rightarrow #3}
\newcommand{\R}{\mathbb{R}}
\newcommand{\Z}{\mathbb{Z}}
\newcommand{\N}{\mathbb{N}}
\newcommand{\Q}{\mathbb{Q}}
\newcommand{\rr}{R_{r}}
\newcommand{\tq}{ : }
\newcommand{\mdc}{\text{mdc}}
\newcommand{\mmc}{\text{mmc}}
\newcommand{\defeq}{\vcentcolon=}
\newcommand{\comp}{\mathscr{C}}


%% Upper and Lower Integrals
\newcommand{\loint}[4]{
    \lefteqn{\int_{ #1 }^{ #2 } #3}\lefteqn{\hspace{0.0ex}\rule[-2.25ex]{1.1ex}{.05ex}} \phantom{\int_{ #1 }^{ #2 } #3}\mathrm{d}#4
}
\newcommand{\upint}[4]{
    \lefteqn{\int_{ #1 }^{ #2 } #3 \ }\lefteqn{\hspace{1.2ex}\rule[ 3.35ex]{1.1ex}{.05ex}} \phantom{\int_{ #1 }^{ #2 } #3 \ }\mathrm{d}#4
}

\DeclarePairedDelimiter\ceil{\lceil}{\rceil}
\DeclarePairedDelimiter\floor{\lfloor}{\rfloor}
\DeclarePairedDelimiter\abs{\lvert}{\rvert}%
\DeclarePairedDelimiter\norm{\lVert}{\rVert}%
\DeclareMathOperator{\sen}{sen}

% Swap the definition of \abs* and \norm*, so that \abs
% and \norm resizes the size of the brackets, and the 
% starred version does not.

\makeatletter
\let\oldabs\abs
\def\abs{\@ifstar{\oldabs}{\oldabs*}}
%
\let\oldnorm\norm
\def\norm{\@ifstar{\oldnorm}{\oldnorm*}}
\makeatother


\begin{document}

%% \newtheorem{teo}{Teorema} \newtheorem*{teo*}{Teorema}
%% \newtheorem{prop}[teo]{Proposição} \newtheorem*{prop*}{Proposição}
%% \newtheorem{lema}[teo]{Lemma} \newtheorem*{lema*}{Lema}
%% \newtheorem{cor}[teo]{Corolário} \newtheorem*{cor*}{Corolário}

\theoremstyle{definition}
\newtheorem{defi}[teo]{Definição} \newtheorem*{defi*}{Definição}
\newtheorem{exem}[teo]{Exemplo} \newtheorem*{exem*}{Exemplo}
\newtheorem{obs}[teo]{Observação} \newtheorem*{obs*}{Observação}
\newtheorem*{hipo}{Hipóteses}
\newtheorem*{nota}{Notação}


\newcommand{\ds}{\displaystyle} \newcommand{\nl}{\newline}
\newcommand{\eps}{\varepsilon} \newcommand{\ssty}{\scriptstyle}
\newcommand{\bE}{\mathbb{E}}
\newcommand{\cB}{\mathcal{B}}
\newcommand{\cF}{\mathcal{F}}
\newcommand{\cA}{\mathcal{A}}
\newcommand{\cM}{\mathcal{M}}
\newcommand{\cD}{\mathcal{D}}
\newcommand{\cN}{\mathcal{N}}
\newcommand{\cL}{\mathcal{L}}
\newcommand{\cLN}{\mathcal{LN}}
\newcommand{\bP}{\mathbb{P}}
\newcommand{\bQ}{\mathbb{Q}}
\newcommand{\bN}{\mathbb{N}}
\newcommand{\bR}{\mathbb{R}}
\newcommand{\bZ}{\mathbb{Z}}

\newcommand{\bfw}{\mathbf{w}}
\newcommand{\bfv}{\mathbf{v}}
\newcommand{\bfu}{\mathbf{u}}
\newcommand{\bfx}{\mathbf{x}}
\newcommand{\bfy}{\mathbf{y}}
\newcommand{\bfa}{\mathbf{a}}
\newcommand{\bfb}{\mathbf{b}}
\newcommand{\bfc}{\mathbf{c}}
\newcommand{\bfd}{\mathbf{d}}

\newcommand{\bvecc}[2]{%
  \begin{bmatrix} #1 \\ #2  \end{bmatrix}
}
\newcommand{\bveccc}[3]{%
  \begin{bmatrix} #1 \\ #2 \\ #3  \end{bmatrix}
}

\newenvironment{sol} 
{
    \vspace{4mm}
    \noindent\textbf{Resolução:}
    \strut\newline
    \smallskip
    \hspace{-3.5mm} 
} 
% Objetos que aparecem *após* o ambiente. 
% (você pode, por exemplo, modificar, 
% ou remover, a barra horizontal} 
{\noindent\rule{4cm}{.1mm}}

\title{Álgebra Linear - Soluções Lista de Exercícios 7}

\author{Caio Lins e Tiago Silva}

\date{\today}

\maketitle

\begin{enumerate}

%%%%%%%%%%%%%%%%%%%%%%%%%%%%%%%%%%%%%%%%%%%%%%%%%%%%%%%%%
%%%%%%%%%%%%%%%%%%%%%% Exercício 1 %%%%%%%%%%%%%%%%%%%%%%
%%%%%%%%%%%%%%%%%%%%%%%%%%%%%%%%%%%%%%%%%%%%%%%%%%%%%%%%%

\item Se $AB = 0$, as colunas de $B$ estão em qual espaço fundamental de $A$? E as linhas de $A$ estão em qual espaço fundamental de $B$? É possível que $A$ e $B$ sejam $3 \times 3$ e com posto 2?

\begin{sol} 

    Como \( AB = 0 \), devemos ter \( A\bfb_{ i } = 0 \) para toda coluna \( \bfb_{ i } \) de \( B \).
    Logo, \( \bfb_{ i } \in N ( A ) \).
    Da mesma forma, devemos ter \( \bfa^{ \transpose }_{ i } B = 0 \) para toda linha \( \bfa_{ i }^{ \transpose } \) de \( A \).
    Tomando o transposto de cada lado da equação, temos \( B^{ \transpose } \bfa_{ i } = 0 \), ou seja, \( \bfa_{ i } \in N ( B^{ \transpose } ) \).

    Não é possível que ambas sejam \( 3 \times 3 \) com posto \( 2 \).
    Se \( B \) tem posto \( 2 \), então temos \( \dim N ( A ) \geq 2 \).
    Pelo Teorema do Posto, isso implica \( \dim C ( A ) \leq 1 \) e, assim, \( A \) não pode ter posto \( 2 \).

\end{sol} 


%%%%%%%%%%%%%%%%%%%%%%%%%%%%%%%%%%%%%%%%%%%%%%%%%%%%%%%%%
%%%%%%%%%%%%%%%%%%%%%% Exercício 2 %%%%%%%%%%%%%%%%%%%%%%
%%%%%%%%%%%%%%%%%%%%%%%%%%%%%%%%%%%%%%%%%%%%%%%%%%%%%%%%% 

\item Se $Ax = b$ e $A^Ty = 0$, temos $y^Tx = 0$ ou $y^Tb=0$?

\begin{sol} 
    
    De \( \bfy^{ \transpose } A = 0 \) obtemos \( A^{ \transpose } \bfy = 0 \).
    Portanto, multiplicando ambos lados de \( A \bfx = \bfb \) por \( \bfy^{ \transpose } \), obtemos \( \bfy^{ \transpose } \bfb = 0 \).
    Por outro lado, se \( A \) não é quadrada, \( \bfy \) e \( \bfx \) não pertencem a espaços euclidianos de mesma dimensão.
    Logo, o produto interno \( \bfy^{ \transpose } \bfx \) nem sempre está bem definido.
    Entretanto, mesmo se \( A \) for quadrada essa afirmação ainda não será válida.
    Tome, por exemplo,
    \begin{equation*}
        A =
        \begin{bmatrix}
            1 & 1 \\
            1 & 1
        \end{bmatrix}, \
        \bfb =
        \begin{bmatrix}
            1 \\
            1
        \end{bmatrix}, \
        \bfx =
        \begin{bmatrix}
            1 \\
            0
        \end{bmatrix} \text{ e } \
        \bfy =
        \begin{bmatrix}
            1 \\
            -1
        \end{bmatrix}
    .\end{equation*}

\end{sol} 


%%%%%%%%%%%%%%%%%%%%%%%%%%%%%%%%%%%%%%%%%%%%%%%%%%%%%%%%%
%%%%%%%%%%%%%%%%%%%%%% Exercício 3 %%%%%%%%%%%%%%%%%%%%%%
%%%%%%%%%%%%%%%%%%%%%%%%%%%%%%%%%%%%%%%%%%%%%%%%%%%%%%%%% 

\item O sistema abaixo não tem solução:
$$\begin{cases}
x + 2y + 2z = 5\\
2x + 2y + 3z = 5\\
3x + 4y + 5z = 9
\end{cases}$$
Ache números $y_1,y_2,y_3$ para multiplicar as equações acima para que elas somem $0=1$. Em qual espaço fundamental o vetor $y$ pertence? Verifique que $y^Tb = 1$. O caso acima é típico e conhecido como a \textit{Alternativa de Fredholm}: ou $Ax = b$ ou $A^Ty = 0$ com $y^Tb = 1$.

\begin{sol} 

    O sistema pode ser escrito como \( A \bfx = \bfb \), onde
    \begin{equation*}
        A =
        \begin{bmatrix}
            1 & 2 & 2 \\
            2 & 2 & 3 \\
            3 & 4 & 5
        \end{bmatrix}, \
        \bfx =
        \begin{bmatrix}
            x \\
            y \\
            z
        \end{bmatrix} \text{ e } \
        \bfb =
        \begin{bmatrix}
            5 \\
            5 \\
            9
        \end{bmatrix}
    .\end{equation*}
    Procuramos \( \bfy = ( y_{ 1 }, y_{ 2 }, y_{ 3 } ) \) tal que \( \bfy^{ \transpose } A = 0 \), ou seja, \( \bfy \in N ( A^{ \transpose } ) \), e \( \bfy^{ \transpose } \bfb = 1 \).
    Observe que basta encontrarmos um \( \bfy' \in N ( A^{ \transpose } ) \) e fazer \( \bfy \defeq \norm{ \bfy^{ \transpose } \bfb }^{ -1 } \bfy' \).
    Prosseguindo pelos métodos já estudados, chegamos no vetor
    \begin{equation*}
        \bfy' = ( 1, 1, -1 )
    .\end{equation*}
    Por sorte, já temos \( \bfy^{ \transpose } \bfb = 1 \) e não precisamos realizar a normalização, podendo, então, tomar \( \bfy \defeq \bfy' \).
\end{sol} 


%%%%%%%%%%%%%%%%%%%%%%%%%%%%%%%%%%%%%%%%%%%%%%%%%%%%%%%%%
%%%%%%%%%%%%%%%%%%%%%% Exercício 4 %%%%%%%%%%%%%%%%%%%%%% 
%%%%%%%%%%%%%%%%%%%%%%%%%%%%%%%%%%%%%%%%%%%%%%%%%%%%%%%%% 

\item Mostre que se $A^TAx = 0$, então $Ax = 0$. O oposto é obviamente verdade e então temos $N(A^TA) = N(A)$.

\begin{sol} 

    Multiplicando ambos lados de \( A^{ \transpose } A \bfx = 0 \) por \( \bfx^{ \transpose } \), ficamos com
    \begin{equation*}
        \bfx^{ \transpose } A^{ \transpose } A \bfx = 0
    ,\end{equation*}
    o que implica \( ( A \bfx )^{ \transpose } ( A \bfx ) = 0 \) e, assim, \( \norm{ A \bfx }^{ 2 } = 0 \).
    Com isso, \( \norm{ A \bfx } = 0 \) e \( A \bfx = 0 \).

\end{sol} 


%%%%%%%%%%%%%%%%%%%%%%%%%%%%%%%%%%%%%%%%%%%%%%%%%%%%%%%%%
%%%%%%%%%%%%%%%%%%%%%% Exercício 5 %%%%%%%%%%%%%%%%%%%%%%
%%%%%%%%%%%%%%%%%%%%%%%%%%%%%%%%%%%%%%%%%%%%%%%%%%%%%%%%% 

\item Seja $A$ uma matriz $3 \times 4$ e $B$ uma $4 \times 5$ tais que $AB = 0$. Mostre que $C(B) \subset N(A)$. Além disso, mostre que posto$(A)$ $ + $ posto$(B) \leq 4$.

\begin{sol} 

    Pela questão \( 1 \) sabemos que as colunas de \( B \) pertencem a \( N ( A ) \).
    Como \( C ( B ) \) é, por definição, o \( \vspan \) das colunas de \( B \) e \( N ( A ) \) é um subespaço vetorial, temos \( C ( B ) \subset N ( A ) \).
    Com isso, \( \dim C ( B ) \leq \dim N ( A ) \).
    Pelo Teorema do Posto, temos \( \dim N ( A ) = 4 - \dim C ( A ) \) e, assim, \( \dim C ( B ) \leq 4 - \dim C ( A ) \), ou seja,
    \begin{equation*}
        \dim C ( A ) + \dim C ( B ) \leq 4
    .\end{equation*}

\end{sol} 


%%%%%%%%%%%%%%%%%%%%%%%%%%%%%%%%%%%%%%%%%%%%%%%%%%%%%%%%% 
%%%%%%%%%%%%%%%%%%%%%% Exercício 6 %%%%%%%%%%%%%%%%%%%%%% 
%%%%%%%%%%%%%%%%%%%%%%%%%%%%%%%%%%%%%%%%%%%%%%%%%%%%%%%%%

\item Sejam $\mathbf{a,b,c,d}$ vetores não-zeros de $\bR^2$.

\begin{enumerate}

\item Quais são as condições sobre esses vetores para que cada um possa ser, respectivamente, base dos espaços $C(A^T)$, $N(A)$, $C(A)$ e $N(A^T)$ para uma dada matriz $A$ que seja $2 \times 2$. \textit{Dica: cada espaço fundamental vai ter somente um desses vetores como base.}

\begin{sol} 

    Perceba que, como temos \( C ( A^{ \transpose } ) \perp N ( A ) \) e \( C ( A ) \perp N ( A^{ \transpose } ) \), uma condição necessária para termos
    \begin{align*}
        C ( A^{ \transpose } ) &= \vspan \left\{ \bfa \right\} \\
        N ( A ) &= \vspan \left\{ \bfb \right\} \\
        C ( A ) &= \vspan  \left\{ \bfc \right\} \\
        N ( A^{ \transpose } ) &= \vspan  \left\{ \bfd \right\}
    \end{align*}
    é \( \bfa^{ \transpose } \bfb = \bfc^{ \transpose } \bfd = 0 \).
    Vamos mostrar que essa condição é, na verdade, suficiente, ou seja, vamos encontrar uma matriz \( A \) que satisfaça as igualdades acima, dados os vetores \( \bfa, \bfb, \bfc \) e \( \bfd \).
    Para termos \( C ( A^{ \transpose } ) = \vspan  \left\{ \bfa \right\} \) é necessário que a matriz \( A \) seja da forma
    \begin{equation*}
        \begin{bmatrix}
            \alpha \bfa^{ \transpose } \\
            \beta \bfa^{ \transpose }
        \end{bmatrix}
    ,\end{equation*}
    onde \( \alpha, \beta \in \R \).
    Perceba que, com isso, temos \( \dim C ( A ) = \dim C ( A^{ \transpose } ) = 1 \), o que implica, pelo Teorema do Posto, \( \dim N ( A ) = 1 \).
    Com isso, \( \bfb \) já é uma base para \( N ( A ) \).
    Agora, sendo \( \bfa = ( a_{ 1 }, a_{ 2 } ) \) e \( \bfc = ( c_{ 1 }, c_{ 2 } ) \), observe que pondo \( \alpha \defeq c_{ 1 }/a_{ 1 } \) e \( \beta \defeq c_{ 2 } / a_{ 1 } \) (ou sobre \( a_{ 2 } \), caso \( a_{ 1 } = 0 \), mas vamos supor \( a_{ 1 } \neq 0 \), pois o outro caso é análogo), ficamos com
    \begin{equation*}
        A =
        \begin{bmatrix}
            c_{ 1 } & \frac{ a_{ 2 } }{ a_{ 1 } } c_{ 1 } \\
            c_{ 2 } & \frac{ a_{ 2 } }{ a_{ 1 } } c_{ 2 }
        \end{bmatrix}
    .\end{equation*}
    Definindo \( \lambda \defeq a_{ 2 } / a_{ 1 } \), temos que
    \begin{equation*}
        A = \begin{bmatrix}
            \bfc & \lambda \bfc
        \end{bmatrix}
    .\end{equation*}
    Portanto, seguindo um raciocínio análogo ao anterior, automaticamente temos \( C ( A ) = \vspan \left\{ \bfc \right\} \) e \( N ( A^{ \transpose } ) = \vspan \left\{ \bfd \right\} \).

\end{sol} 

\item Qual seria uma matriz $A$ possível?

\begin{sol} 

    Como mostrado no item anterior, podemos tomar
    \begin{equation*}
        A =
        \begin{bmatrix}
            c_{ 1 } & \frac{ a_{ 2 } }{ a_{ 1 } } c_{ 1 } \\
            c_{ 2 } & \frac{ a_{ 2 } }{ a_{ 1 } } c_{ 2 }
        \end{bmatrix}
    .\end{equation*}

\end{sol} 

\end{enumerate}


%%%%%%%%%%%%%%%%%%%%%%%%%%%%%%%%%%%%%%%%%%%%%%%%%%%%%%%%% 
%%%%%%%%%%%%%%%%%%%%%% Exercício 7 %%%%%%%%%%%%%%%%%%%%%% 
%%%%%%%%%%%%%%%%%%%%%%%%%%%%%%%%%%%%%%%%%%%%%%%%%%%%%%%%%

\item Ache $S^{\perp}$ para os seguintes conjuntos:

\begin{enumerate}

\item $S = \{0\}$


\item $S = span\{[1,1,1]\}$

\item $S = span\{[1,1,1], [1,1,-1]\}$

\item $S = \{[1,5,1], [2,2,2]\}$. Note que $S$ não é um subespaço, mas $S^\perp$ é.

\end{enumerate}

\begin{sol}
    Observe (\emph{tente mostrar}) que se \( V = \vspan \left\{ \bfv_{ 1 }, \dots, \bfv_{ n } \right\} \), então \( \bfw \in V^{ \perp } \) se, e somente se, \( \bfw^{ \transpose } \bfv_{ i } = 0 \) para todo \( i = 1, \dots, n \).
    Vamos usar esse fato nas resoluções.

	\begin{enumerate}    
		\item Para todo \( \bfx \in \R^{ 3 } \), vale \( \bfx^{ \transpose } 0 = 0 \).
            Logo, \( S^{ \perp } = \R^{ 3 } \).

		\item O conjunto \( S^{ \perp } \) é exatamente \( N ( A ) \), onde \( A = 
            \begin{bmatrix}
                1 & 1 & 1 
		    \end{bmatrix} \).
            Como o posto de \( A \) é claramente \( 1 \), o núcleo de \( A \) tem dimensão \( 2 \).
            Como \( ( -1, 0, 1 ) \) e \( ( 0, -1, 1 ) \) são dois elementos linearmente independentes de \( N ( A ) \), temos
            \begin{equation*}
                S^{ \perp } = N ( A ) =
                \vspan
                \left\{ 
                    \begin{bmatrix}
                        -1 \\
                        0 \\
                        1
                    \end{bmatrix},
                    \begin{bmatrix}
                        0 \\
                        -1 \\
                        1
                    \end{bmatrix}
                \right\}
            .\end{equation*}

		\item Analogamente ao item anterior, vamos calcular \( N ( A ) \), onde
            \begin{equation*}
                A =
                \begin{bmatrix}
                    1 & 1 & 1 \\
                    1 & 1 & -1
                \end{bmatrix}
            .\end{equation*}
            Prosseguindo pelos métodos usuais, encontramos a base
            \begin{equation*}
                \left\{ 
                    \begin{bmatrix}
                        1 \\
                        -1 \\
                        0
                    \end{bmatrix}
                \right\}
            .\end{equation*}
            Portanto,
            \begin{equation*}
                S^{ \perp } = \vspan 
                \left\{ 
                    \begin{bmatrix}
                        1 \\
                        -1 \\
                        0
                    \end{bmatrix}
                \right\}
            .\end{equation*}
		\item De maneira análoga,
            \begin{equation*}
                S^{ \perp } =
                N \left(
                    \begin{bmatrix}
                        1 & 5 & 1 \\
                        2 & 2 & 2
                    \end{bmatrix}
                \right) =
                \vspan \left\{ 
                    \begin{bmatrix}
                        1 \\
                        0 \\
                        -1
                    \end{bmatrix}
                \right\}
            .\end{equation*}
	\end{enumerate}    
\end{sol} 


%%%%%%%%%%%%%%%%%%%%%%%%%%%%%%%%%%%%%%%%%%%%%%%%%%%%%%%%%
%%%%%%%%%%%%%%%%%%%%%% Exercício 8 %%%%%%%%%%%%%%%%%%%%%%
%%%%%%%%%%%%%%%%%%%%%%%%%%%%%%%%%%%%%%%%%%%%%%%%%%%%%%%%%

\item Seja $A$ uma matriz $4 \times 3$ formada pela primeiras 3 colunas da matriz identidade $4 \times 4$. Projeta o vetor $b = [1,2,3,4]$ no espaço coluna de $A$. Ache a matriz de projeção $P$.

\begin{sol} 

    Temos
    \begin{equation*}
        A =
        \begin{bmatrix}
            1 & 0 & 0 \\
            0 & 1 & 0 \\
            0 & 0 & 1 \\
            0 & 0 & 0
        \end{bmatrix}
    .\end{equation*}
    Queremos encontrar \( \hat{ \bfx } \) tal que
    \begin{equation*}
        A^{ \transpose } ( A \hat{ \bfx } - \bfb ) = 0
    .\end{equation*}
    Desenvolvendo a equação, obtemos
    \begin{equation*}
        \hat{ \bfx } = ( A^{ \transpose } A )^{ -1 } A^{ \transpose } \bfb
    .\end{equation*}
    Realizando as contas, ficamos com
    \begin{equation*}
        A^{ \transpose }A = I_{ 3 }
    .\end{equation*}
    Assim,
    \begin{equation*}
        \hat{ x } = A^{ \transpose } \bfb =
        \begin{bmatrix}
            1 \\
            2 \\
            3
        \end{bmatrix}
    .\end{equation*}
    A projeção de \( \bfb \) é dada por \( A \hat{ \bfx } = ( 1, 2, 3, 0 ) \).
    A matriz de projeção é dada por
    \begin{equation*}
        P = A ( A^{ \transpose } A )^{ -1 } A^{ \transpose } = A A^{ \transpose } =
        \begin{bmatrix}
            1 & 0 & 0 & 0 \\
            0 & 1 & 0 & 0 \\
            0 & 0 & 1 & 0 \\
            0 & 0 & 0 & 0
        \end{bmatrix}
    .\end{equation*}
\end{sol} 


%%%%%%%%%%%%%%%%%%%%%%%%%%%%%%%%%%%%%%%%%%%%%%%%%%%%%%%%%
%%%%%%%%%%%%%%%%%%%%%% Exercício 9 %%%%%%%%%%%%%%%%%%%%%%
%%%%%%%%%%%%%%%%%%%%%%%%%%%%%%%%%%%%%%%%%%%%%%%%%%%%%%%%%

\item Se $P^2 = P$, mostre que $(I - P)^2 = I - P$. Para a matriz $P$ do exercício anterior, em qual subespaço a matriz $I - P$ projeta?

\begin{sol} 

    Desenvolendo \( ( I - P )^{ 2 } \), obtemos
    \begin{align*}
        ( I - P )^{ 2 } &= I^2 - I P - P I + P^2 \\
                        &= I - 2 P + P \\
                        &= I - P
    .\end{align*}
    Com a matriz \( P \) do item anterior, temos
    \begin{equation*}
        I - P =
        \begin{bmatrix}
            0 & 0 & 0 & 0 \\
            0 & 0 & 0 & 0 \\
            0 & 0 & 0 & 0 \\
            0 & 0 & 0 & 1
        \end{bmatrix}
    .\end{equation*}
    Essa matriz projeta no subespaço
    \begin{equation*}
        \left\{ \bfx = ( x_{ 1 }, \dots, x_{ 4 } ) \in \R^{ 4 } : x_{ 1 } = x_{ 2 } = x_{ 3 } = 0 \right\}
    .\end{equation*}
\end{sol} 
\end{enumerate}
\end{document} 
