\documentclass[leqno]{article}

\usepackage[brazil]{babel} %\usepackage[latin1]{inputenc}
\usepackage{a4wide}
\setlength{\oddsidemargin}{-0.2in}
% % \setlength{\oddsidemargin}{0.2in}
\setlength{\evensidemargin}{-0.2in}
% % \setlength{\evensidemargin}{0.5in}
% % \setlength{\textwidth}{5.5in}
\setlength{\textwidth}{6.5in}
\setlength{\topmargin}{-1.2in}
\setlength{\textheight}{10in}
\usepackage[]{amsfonts} \usepackage[]{amsmath}
\usepackage[]{amssymb} \usepackage[]{latexsym}
\usepackage{graphicx,color} \usepackage{amsthm}
\usepackage{mathrsfs} \usepackage{url}
\usepackage{cancel} \usepackage{enumerate}
\usepackage{xifthen} \usepackage{tikz}
\usetikzlibrary{automata,arrows,positioning,calc}

\numberwithin{equation}{section}

\setlength{\parindent}{12 pt}

\begin{document}

\newtheorem{teo}{Teorema}[section] \newtheorem*{teo*}{Teorema}
\newtheorem{prop}[teo]{Proposição} \newtheorem*{prop*}{Proposição}
\newtheorem{lema}[teo]{Lemma} \newtheorem*{lema*}{Lema}
\newtheorem{cor}[teo]{Corolário} \newtheorem*{cor*}{Corolário}

\theoremstyle{definition}
\newtheorem{defi}[teo]{Definição} \newtheorem*{defi*}{Definição}
\newtheorem{exem}[teo]{Exemplo} \newtheorem*{exem*}{Exemplo}
\newtheorem{obs}[teo]{Observação} \newtheorem*{obs*}{Observação}
\newtheorem*{hipo}{Hipóteses}
\newtheorem*{nota}{Notação}

\newcommand{\ds}{\displaystyle} \newcommand{\nl}{\newline}
\newcommand{\eps}{\varepsilon} \newcommand{\ssty}{\scriptstyle}
\newcommand{\bE}{\mathbb{E}}
\newcommand{\cB}{\mathcal{B}}
\newcommand{\cF}{\mathcal{F}}
\newcommand{\cA}{\mathcal{A}}
\newcommand{\cM}{\mathcal{M}}
\newcommand{\cD}{\mathcal{D}}
\newcommand{\cN}{\mathcal{N}}
\newcommand{\cL}{\mathcal{L}}
\newcommand{\cLN}{\mathcal{LN}}
\newcommand{\bP}{\mathbb{P}}
\newcommand{\bQ}{\mathbb{Q}}
\newcommand{\bN}{\mathbb{N}}
\newcommand{\bR}{\mathbb{R}}
\newcommand{\bZ}{\mathbb{Z}}

\newcommand{\bfw}{\mathbf{w}}
\newcommand{\bfv}{\mathbf{v}}
\newcommand{\bfu}{\mathbf{u}}
\newcommand{\bfx}{\mathbf{x}}
\newcommand{\bfb}{\mathbf{b}}

\newcommand{\bvecc}[2]{%
  \begin{bmatrix} #1 \\ #2  \end{bmatrix}
}
\newcommand{\bveccc}[3]{%
  \begin{bmatrix} #1 \\ #2 \\ #3  \end{bmatrix}
}

\newenvironment{sol} 
{
    \vspace{4mm}
    \noindent\textbf{Resolução:}
    \strut\newline
    \smallskip
    \hspace{-3.5mm} 
} 
% Objetos que aparecem *após* o ambiente. 
% (você pode, por exemplo, modificar, 
% ou remover, a barra horizontal} 
{\noindent\rule{4cm}{.1mm}}


\title{Álgebra Linear - Lista de Exercícios 8}

\author{Meu Nome Maravilhoso}

\date{\today}

\maketitle

\begin{enumerate}

%%%%%%%%%%%%%%%%%%%%%%%%%%%%%%%%%%%%%%%%%%%%%%%%%%%%%%%%%
%%%%%%%%%%%%%%%%%%%%%% Exercício 1 %%%%%%%%%%%%%%%%%%%%%%
%%%%%%%%%%%%%%%%%%%%%%%%%%%%%%%%%%%%%%%%%%%%%%%%%%%%%%%%%

\item Escreva as 3 equações para a reta $b = C + Dt$ passar pelos pontos $(-1,7)$, $(1,7)$, $(2,21)$. Ache a solução de mínimos quadrados $\hat{x}$ e a projeção $p = A\hat{x}$.

\begin{sol}

    % Minha solução genial
    
\end{sol}

%%%%%%%%%%%%%%%%%%%%%%%%%%%%%%%%%%%%%%%%%%%%%%%%%%%%%%%%%
%%%%%%%%%%%%%%%%%%%%%% Exercício 2 %%%%%%%%%%%%%%%%%%%%%%
%%%%%%%%%%%%%%%%%%%%%%%%%%%%%%%%%%%%%%%%%%%%%%%%%%%%%%%%%

\item Dado o problema acima, quais dos quatro subespaços fundamentais contêm o vetor erro $e = b - p$? E o vetor $p$? E o vetor $\hat{x}$? Qual é o núcleo de $A$?

\begin{sol}

    % Minha solução genial
    
\end{sol}

%%%%%%%%%%%%%%%%%%%%%%%%%%%%%%%%%%%%%%%%%%%%%%%%%%%%%%%%%
%%%%%%%%%%%%%%%%%%%%%% Exercício 3 %%%%%%%%%%%%%%%%%%%%%%
%%%%%%%%%%%%%%%%%%%%%%%%%%%%%%%%%%%%%%%%%%%%%%%%%%%%%%%%%

\item Ache a melhor reta que se ajusta aos pontos $t = -2, -1, 0, 1, 2$ e $b = 4, 2, -1, 0, 0$.

\begin{sol}

    % Minha solução genial
    
\end{sol}

%%%%%%%%%%%%%%%%%%%%%%%%%%%%%%%%%%%%%%%%%%%%%%%%%%%%%%%%%
%%%%%%%%%%%%%%%%%%%%%% Exercício 4 %%%%%%%%%%%%%%%%%%%%%%
%%%%%%%%%%%%%%%%%%%%%%%%%%%%%%%%%%%%%%%%%%%%%%%%%%%%%%%%%

\item Dados os vetores
$$v_1 = [1 \ -1 \ 0 \ 0], \ v_2 = [0 \ 1 \ -1 \ 0] \mbox{ e } v_3 = [0 \ 0 \ 1 \ -1],$$
use o método de Gram-Schmidt para achar uma base ortornormal que gera o mesmo espaço de $v_1, v_2, v_3$.

\begin{sol}

    % Minha solução genial
    
\end{sol}

%%%%%%%%%%%%%%%%%%%%%%%%%%%%%%%%%%%%%%%%%%%%%%%%%%%%%%%%%
%%%%%%%%%%%%%%%%%%%%%% Exercício 5 %%%%%%%%%%%%%%%%%%%%%%
%%%%%%%%%%%%%%%%%%%%%%%%%%%%%%%%%%%%%%%%%%%%%%%%%%%%%%%%%

\item Se os elementos de cada linha de uma matriz $A$ somam zero, ache uma solução para $Ax = 0$ e conclua que $\det A = 0$. Se esses elementos somam 1, conclua que $\det(A - I) = 0$.

\begin{sol}

    % Minha solução genial
    
\end{sol}

%%%%%%%%%%%%%%%%%%%%%%%%%%%%%%%%%%%%%%%%%%%%%%%%%%%%%%%%%
%%%%%%%%%%%%%%%%%%%%%% Exercício 6 %%%%%%%%%%%%%%%%%%%%%%
%%%%%%%%%%%%%%%%%%%%%%%%%%%%%%%%%%%%%%%%%%%%%%%%%%%%%%%%%

\item Use as propriedades do determinante (e não suas fórmulas) para mostrar que
$$\det \begin{bmatrix} 1 & a & a^2 \\ 1 & b & b^2 \\ 1 & c & c^2 \end{bmatrix} = (b-a)(c-a)(c-b).$$

\begin{sol}

    % Minha solução genial
    
\end{sol}

%%%%%%%%%%%%%%%%%%%%%%%%%%%%%%%%%%%%%%%%%%%%%%%%%%%%%%%%%
%%%%%%%%%%%%%%%%%%%%%% Exercício 7 %%%%%%%%%%%%%%%%%%%%%%
%%%%%%%%%%%%%%%%%%%%%%%%%%%%%%%%%%%%%%%%%%%%%%%%%%%%%%%%%

\item Calcule
$$\det \begin{bmatrix} 0 & 0 & 0 & 1 \\ 1 & 0 & 0 & 0 \\ 0 & 1 & 0 & 0 \\ 0 & 0 & 1 & 0  \end{bmatrix}.$$

\begin{sol}

    % Minha solução genial
    
\end{sol}

%%%%%%%%%%%%%%%%%%%%%%%%%%%%%%%%%%%%%%%%%%%%%%%%%%%%%%%%%
%%%%%%%%%%%%%%%%%%%%%% Exercício 8 %%%%%%%%%%%%%%%%%%%%%%
%%%%%%%%%%%%%%%%%%%%%%%%%%%%%%%%%%%%%%%%%%%%%%%%%%%%%%%%%

\item Use o fato de que
$$\det \begin{bmatrix} 1 & 1 & 1 & 1 \\ 1 & 2 & 3 & 4 \\ 1 & 3 & 6 & 10 \\ 1 & 4 & 10 & 20  \end{bmatrix} = 1$$
para mostrar que

$$\det \begin{bmatrix} 1 & 1 & 1 & 1 \\ 1 & 2 & 3 & 4 \\ 1 & 3 & 6 & 10 \\ 1 & 4 & 10 & \mathbf{19}  \end{bmatrix} = 0.$$

\begin{sol}

    % Minha solução genial
    
\end{sol}

%%%%%%%%%%%%%%%%%%%%%%%%%%%%%%%%%%%%%%%%%%%%%%%%%%%%%%%%%
%%%%%%%%%%%%%%%%%%%%%% Exercício 9 %%%%%%%%%%%%%%%%%%%%%%
%%%%%%%%%%%%%%%%%%%%%%%%%%%%%%%%%%%%%%%%%%%%%%%%%%%%%%%%%

\item Ache o determinante da seguinte matriz:
$$A = \begin{bmatrix} 1 & 1 & 4 \\ 1 & 2 & 2 \\ 1 & 2 & 5   \end{bmatrix}$$
usando cofatores. O que acontece quando mudamos o valor 4 para 100?

\begin{sol}

    % Minha solução genial
    
\end{sol}

\end{enumerate}
\end{document}
