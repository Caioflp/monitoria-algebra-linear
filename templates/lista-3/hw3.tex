\documentclass[leqno]{article}

\usepackage[brazil]{babel} %\usepackage[latin1]{inputenc}
\usepackage{a4wide}
\setlength{\oddsidemargin}{-0.2in}
% % \setlength{\oddsidemargin}{0.2in}
\setlength{\evensidemargin}{-0.2in}
% % \setlength{\evensidemargin}{0.5in}
% % \setlength{\textwidth}{5.5in}
\setlength{\textwidth}{6.5in}
\setlength{\topmargin}{-1.2in}
\setlength{\textheight}{10in}
\usepackage[]{amsfonts} \usepackage[]{amsmath}
\usepackage[]{amssymb} \usepackage[]{latexsym}
\usepackage{graphicx,color} \usepackage{amsthm}
\usepackage{mathrsfs} \usepackage{url}
\usepackage{cancel} \usepackage{enumerate}
\usepackage{xifthen} \usepackage{tikz}
\usetikzlibrary{automata,arrows,positioning,calc}

\numberwithin{equation}{section}

\setlength{\parindent}{12 pt}

\begin{document}

\newtheorem{teo}{Teorema}[section] \newtheorem*{teo*}{Teorema}
\newtheorem{prop}[teo]{Proposição} \newtheorem*{prop*}{Proposição}
\newtheorem{lema}[teo]{Lemma} \newtheorem*{lema*}{Lema}
\newtheorem{cor}[teo]{Corolário} \newtheorem*{cor*}{Corolário}

\theoremstyle{definition}
\newtheorem{defi}[teo]{Definição} \newtheorem*{defi*}{Definição}
\newtheorem{exem}[teo]{Exemplo} \newtheorem*{exem*}{Exemplo}
\newtheorem{obs}[teo]{Observação} \newtheorem*{obs*}{Observação}
\newtheorem*{hipo}{Hipóteses}
\newtheorem*{nota}{Notação}

\newcommand{\ds}{\displaystyle} \newcommand{\nl}{\newline}
\newcommand{\eps}{\varepsilon} \newcommand{\ssty}{\scriptstyle}
\newcommand{\bE}{\mathbb{E}}
\newcommand{\cB}{\mathcal{B}}
\newcommand{\cF}{\mathcal{F}}
\newcommand{\cA}{\mathcal{A}}
\newcommand{\cM}{\mathcal{M}}
\newcommand{\cD}{\mathcal{D}}
\newcommand{\cN}{\mathcal{N}}
\newcommand{\cL}{\mathcal{L}}
\newcommand{\cLN}{\mathcal{LN}}
\newcommand{\bP}{\mathbb{P}}
\newcommand{\bQ}{\mathbb{Q}}
\newcommand{\bN}{\mathbb{N}}
\newcommand{\bR}{\mathbb{R}}
\newcommand{\bZ}{\mathbb{Z}}

\newcommand{\bfw}{\mathbf{w}}
\newcommand{\bfv}{\mathbf{v}}
\newcommand{\bfu}{\mathbf{u}}

\newenvironment{sol}
{
    \vspace{4mm}
    \noindent\textbf{Resolução:}
    \strut\newline
    \smallskip
    \hspace{-3.5mm}
}
{}

\newcommand{\bvecc}[2]{%
  \begin{bmatrix} #1 \\ #2  \end{bmatrix}
}
\newcommand{\bveccc}[3]{%
  \begin{bmatrix} #1 \\ #2 \\ #3  \end{bmatrix}
}


\title{Álgebra Linear - Lista de Exercícios 3}

\author{escreva seu nome aqui}

\date{\today}

\maketitle

\begin{enumerate}

\item Ache a decomposição LU da matriz:
$$A = \begin{bmatrix} 1 & 3 & 0  \\
2 & 4 & 0  \\
2 & 0 & 1 
\end{bmatrix}.$$

\begin{sol} 
	% escreva sua solução aqui.  
\end{sol} 

\item Ache a decomposição LU da matriz simétrica:
$$A = \begin{bmatrix} a & a & a & a  \\
a & b & b & b  \\
a & b & c & c  \\
a & b & c & d 
\end{bmatrix}.$$
Qual condição para $a,b,c,d$ para que $A$ ter quatro pivots?

\begin{sol} 
	% escreva sua solução aqui.  
\end{sol} 

\item Ache a uma matriz de permutação $P$ tal que:

\begin{enumerate}

\item $P$ é 3x3, $P \neq I$ e $P^3 = I$.

\begin{sol} 
	% escreva sua solução aqui.  
\end{sol} 

\item $S$ é 4x4 e $S^4 \neq I$

\begin{sol} 
	% escreva sua solução aqui.  
\end{sol} 

\end{enumerate}

\item Seja $A$ uma matriz $4x4$. Quantas entradas de $A$ podem ser escolhidas independentemente caso $A$ seja

\begin{enumerate}

\item simétrica ($A^T = A$)?

\begin{sol} 
	% escreva sua solução aqui.  
\end{sol} 

\item anti-simétrica ($A^T = -A$)?

\begin{sol} 
	% escreva sua solução aqui.  
\end{sol} 

\end{enumerate}

\item Suponha que $A$ já é triangular inferior com 1's na diagonal. Mostre que $U = I$.

\begin{sol} 
	% escreva sua solução aqui.  
\end{sol} 

\item Seja 
$$A = \begin{bmatrix} 1 & c & 0  \\
2 & 4 & 1  \\
3 & 5 & 1 
\end{bmatrix}.$$

\begin{enumerate}

\item Qual é o número $c$ que leva o segundo pivô a ser 0? O que podemos fazer para resolver tal problema? Ainda é válido $A = LU$?

\begin{sol} 
	% escreva sua solução aqui.  
\end{sol} 

\item Qual é o número $c$ que leva o terceiro pivô a ser 0? É possível resolver esse problema?

\begin{sol} 
	% escreva sua solução aqui.  
\end{sol} 

\end{enumerate}

\item Se $A$ e $B$ são simétricas, quais dessas matrizes são também simétricas:

\begin{enumerate}

\item $A^2 - B^2$;

\begin{sol} 
	% escreva sua solução aqui.  
\end{sol} 

\item $(A + B)(A - B)$;

\begin{sol} 
	% escreva sua solução aqui.  
\end{sol} 

\item $ABA$;

\begin{sol} 
	% escreva sua solução aqui.  
\end{sol} 

\item $ABAB$.

\begin{sol} 
	% escreva sua solução aqui.  
\end{sol} 

\end{enumerate}

\item Prove que é sempre possível escrever $A = B + C$, onde $B$ é simétrica e $C$ anti-simétrica. \textit{Dica: $B$ e $C$ são combinações simples de $A$ e $A^T$.}

\begin{sol} 
	% escreva sua solução aqui.  
\end{sol} 

\item Seja $A$ uma matriz em blocos:
$$A = \begin{bmatrix}
A_{11} & A_{12}\\
A_{21} & A_{22}
\end{bmatrix}$$
onde cada $A_{ii}$ é quadrada $n \times n$ com $A_{11}$ invertível. Ache $L$ e $U$ em blocos tal que $A = LU$:
$$L = \begin{bmatrix}
L_{11} & L_{12}\\
L_{21} & L_{22}
\end{bmatrix} \mbox{ e } U = \begin{bmatrix}
U_{11} & U_{12}\\
U_{21} & U_{22}
\end{bmatrix},$$
onde $L_{11}$, $L_{22}$ são triangulares inferiores com 1's na diagonal e $U_{11}$, $U_{22}$ são triangulares superiores.
\end{enumerate}

\begin{sol} 
	% escreva sua solução aqui.  
\end{sol} 

\end{document} 
