\documentclass[leqno]{article}

\usepackage[brazil]{babel} %\usepackage[latin1]{inputenc}
\usepackage{a4wide}
\setlength{\oddsidemargin}{-0.2in}
% % \setlength{\oddsidemargin}{0.2in}
\setlength{\evensidemargin}{-0.2in}
% % \setlength{\evensidemargin}{0.5in}
% % \setlength{\textwidth}{5.5in}
\setlength{\textwidth}{6.5in}
\setlength{\topmargin}{-1.2in}
\setlength{\textheight}{10in}
\usepackage[]{amsfonts} \usepackage[]{amsmath}
\usepackage[]{amssymb} \usepackage[]{latexsym}
\usepackage{graphicx,color} \usepackage{amsthm}
\usepackage{mathrsfs} \usepackage{url}
\usepackage{cancel} \usepackage{enumerate}
\usepackage{xifthen} \usepackage{tikz}
\usepackage{xcolor}
\usetikzlibrary{automata,arrows,positioning,calc}

\numberwithin{equation}{section}

\setlength{\parindent}{12 pt}

\begin{document}

\newtheorem{teo}{Teorema}[section] \newtheorem*{teo*}{Teorema}
\newtheorem{prop}[teo]{Proposição} \newtheorem*{prop*}{Proposição}
\newtheorem{lema}[teo]{Lemma} \newtheorem*{lema*}{Lema}
\newtheorem{cor}[teo]{Corolário} \newtheorem*{cor*}{Corolário}

\theoremstyle{definition}
\newtheorem{defi}[teo]{Definição} \newtheorem*{defi*}{Definição}
\newtheorem{exem}[teo]{Exemplo} \newtheorem*{exem*}{Exemplo}
\newtheorem{obs}[teo]{Observação} \newtheorem*{obs*}{Observação}
\newtheorem*{hipo}{Hipóteses}
\newtheorem*{nota}{Notação}

\newcommand{\ds}{\displaystyle} \newcommand{\nl}{\newline}
\newcommand{\eps}{\varepsilon} \newcommand{\ssty}{\scriptstyle}
\newcommand{\bE}{\mathbb{E}}
\newcommand{\cB}{\mathcal{B}}
\newcommand{\cF}{\mathcal{F}}
\newcommand{\cA}{\mathcal{A}}
\newcommand{\cM}{\mathcal{M}}
\newcommand{\cD}{\mathcal{D}}
\newcommand{\cN}{\mathcal{N}}
\newcommand{\cL}{\mathcal{L}}
\newcommand{\cLN}{\mathcal{LN}}
\newcommand{\bP}{\mathbb{P}}
\newcommand{\bQ}{\mathbb{Q}}
\newcommand{\bN}{\mathbb{N}}
\newcommand{\bR}{\mathbb{R}}
\newcommand{\bZ}{\mathbb{Z}}

\newcommand{\bfw}{\mathbf{w}}
\newcommand{\bfv}{\mathbf{v}}
\newcommand{\bfu}{\mathbf{u}}

\definecolor{mygreen}{rgb}{0.1, 0.7, 0.1}
\definecolor{myblue}{rgb}{0.1, 0.1, 0.7}
\definecolor{myred}{rgb}{0.7, 0.1, 0.1}

% %% Packages

\usepackage{mathtools, amsthm}
% \usepackage{tikz}
% \usetikzlibrary{positioning}
% \usepackage{csquotes}
% \usepackage{hyperref}

%% Environments

\theoremstyle{plain} % default
\newtheorem{teo}{Teorema}[section]
\newtheorem*{teo*}{Teorema}
\newtheorem{lem}{Lema}[section]
\newtheorem{prop}{Proposição}[section]
\newtheorem{cor}[teo]{Corolário}
\newtheorem*{axiom}{Axioma}

\newtheorem*{TAU}{Teorema da Aproximação Universal}
\newtheorem*{Riesz}{Teorema da Representação de Riesz}

\theoremstyle{definition}
\newtheorem{defn}{Definição}[section]
\newtheorem{conj}{Conjectura}[section]
\newtheorem{exmp}{Exemplo}[section]
\newtheorem{rem}{Observação}[section]
\newtheorem*{rem*}{Observação}

\theoremstyle{remark}
\newtheorem*{note}{Nota}
\newtheorem{case}{Caso}


% Macros

\renewcommand{\vec}[1]{\mathbf{#1}}
\renewcommand{\Re}{\text{Re}}

\newcommand{\K}{\mathbb{K}}
\newcommand{\I}{\mathbb{I}}

\DeclarePairedDelimiter{\dotprod}{\langle}{\rangle}

\DeclareMathOperator{\rk}{rk}
\DeclareMathOperator{\intt}{int}
\DeclareMathOperator{\diam}{diam}
\DeclareMathOperator{\rref}{rref}
\DeclareMathOperator{\vspan}{span}
\DeclareMathOperator{\proj}{proj}
\DeclareMathOperator{\lin}{Lin}
\DeclareMathOperator{\supp}{supp}

\newcommand{\func}[3]{#1 : #2 \rightarrow #3}
\newcommand{\R}{\mathbb{R}}
\newcommand{\Z}{\mathbb{Z}}
\newcommand{\N}{\mathbb{N}}
\newcommand{\Q}{\mathbb{Q}}
\newcommand{\rr}{R_{r}}
\newcommand{\tq}{ : }
\newcommand{\mdc}{\text{mdc}}
\newcommand{\mmc}{\text{mmc}}
\newcommand{\defeq}{\vcentcolon=}
\newcommand{\comp}{\mathscr{C}}


%% Upper and Lower Integrals
\newcommand{\loint}[4]{
    \lefteqn{\int_{ #1 }^{ #2 } #3}\lefteqn{\hspace{0.0ex}\rule[-2.25ex]{1.1ex}{.05ex}} \phantom{\int_{ #1 }^{ #2 } #3}\mathrm{d}#4
}
\newcommand{\upint}[4]{
    \lefteqn{\int_{ #1 }^{ #2 } #3 \ }\lefteqn{\hspace{1.2ex}\rule[ 3.35ex]{1.1ex}{.05ex}} \phantom{\int_{ #1 }^{ #2 } #3 \ }\mathrm{d}#4
}

\DeclarePairedDelimiter\ceil{\lceil}{\rceil}
\DeclarePairedDelimiter\floor{\lfloor}{\rfloor}
\DeclarePairedDelimiter\abs{\lvert}{\rvert}%
\DeclarePairedDelimiter\norm{\lVert}{\rVert}%
\DeclareMathOperator{\sen}{sen}

% Swap the definition of \abs* and \norm*, so that \abs
% and \norm resizes the size of the brackets, and the 
% starred version does not.

\makeatletter
\let\oldabs\abs
\def\abs{\@ifstar{\oldabs}{\oldabs*}}
%
\let\oldnorm\norm
\def\norm{\@ifstar{\oldnorm}{\oldnorm*}}
\makeatother

\newcommand{\transpose}{\mathsf{T}}

\newenvironment{sol}
{
    \vspace{4mm}
    \noindent\textbf{Resolução:}
    \strut\newline
    \smallskip
    \hspace{-3.5mm}
}
{}

\newcommand{\bvecc}[2]{%
    \begin{bmatrix} #1 \\ #2  \end{bmatrix}
}
\newcommand{\bveccc}[3]{%
    \begin{bmatrix} #1 \\ #2 \\ #3  \end{bmatrix}
}


\title{Álgebra Linear - Soluções da Lista de Exercícios 3}

\author{Caio Lins e Tiago Silva}

\date{\today}

\maketitle

\begin{enumerate}

    \item Ache a decomposição LU da matriz:
        $$A =
        \begin{bmatrix}
            1 & 3 & 0  \\
            2 & 4 & 0  \\
            2 & 0 & 1 
        \end{bmatrix}.
        $$

        \begin{sol} 
            Vamos realizar a eliminação Gaussiana até obter uma matriz triangular superior:
            \begin{align*}
                \begin{bmatrix}
                    1 & 3 & 0  \\
                    2 & 4 & 0  \\
                    2 & 0 & 1 
                \end{bmatrix}
                &\xrightarrow{
                    \begin{array}{l}
                        L_{ 2 } - {\color{myred} 2} L_{ 1 } \\
                        L_{ 3 } - {\color{myblue} 2} L_{ 1 }
                    \end{array}
                }
                \begin{bmatrix}
                    1 & 3 & 0  \\
                    0 & -2 & 0  \\
                    0 & -6 & 1 
                \end{bmatrix} \\
                &\xrightarrow{
                    \begin{array}{l}
                        L_{ 3 } - {\color{mygreen} 3} L_{ 2 }
                    \end{array}
                }
                \begin{bmatrix}
                    1 & 3 & 0  \\
                    0 & -2 & 0  \\
                    0 & 0 & 1 
                \end{bmatrix}
                = U
            .\end{align*}
            Para obter a matriz \( L \), basta observar os coeficientes utilizados para eliminar cada entrada da matriz.
            Com isso,
            \begin{equation*}
                L =
                \begin{bmatrix}
                    1 & 0 & 0 \\
                    {\color{myred} 2} & 1 & 0 \\
                    {\color{myblue} 2} & {\color{mygreen} 3} & 1
                \end{bmatrix}
            .\end{equation*}
            Portanto,
            \begin{equation*}
                A = 
                \begin{bmatrix}
                    1 & 0 & 0 \\
                    2 & 1 & 0 \\
                    2 & 3 & 1
                \end{bmatrix}
                \begin{bmatrix}
                    1 & 3 & 0  \\
                    0 & -2 & 0  \\
                    0 & 0 & 1 
                \end{bmatrix}
            .\end{equation*}
        \end{sol} 

    \item Ache a decomposição LU da matriz simétrica:
        $$A = 
        \begin{bmatrix}
            a & a & a & a  \\
            a & b & b & b  \\
            a & b & c & c  \\
            a & b & c & d 
        \end{bmatrix}.
        $$
        Qual condição para $a,b,c,d$ para que $A$ ter quatro pivots?

        \begin{sol} 
            Novamente, prosseguimos por eliminação Gaussiana.
            \begin{align*}
                \begin{bmatrix}
                    a & a & a & a  \\
                    a & b & b & b  \\
                    a & b & c & c  \\
                    a & b & c & d 
                \end{bmatrix}
                    &\xrightarrow{
                        \begin{array}{l}
                            L_{ 2 } - L_{ 1 } \\
                            L_{ 3 } - L_{ 1 } \\
                            L_{ 4 } - L_{ 1 } \\
                        \end{array}
                    }
                    \begin{bmatrix}
                        a & a & a & a  \\
                        0 & b - a & b - a & b - a  \\
                        0 & b - a & c - a & c - a  \\
                        0 & b - a & c - a & d - a 
                    \end{bmatrix} \\
                    &\xrightarrow{
                        \begin{array}{l}
                            L_{ 3 } - L_{ 2 } \\
                            L_{ 4 } - L_{ 2 }
                        \end{array}
                    }
                    \begin{bmatrix}
                        a & a & a & a  \\
                        0 & b - a & b - a & b - a  \\
                        0 & 0     & c - b & c - b  \\
                        0 & 0     & c - b & d - b 
                    \end{bmatrix} \\
                    &\xrightarrow{
                        \begin{array}{l}
                            L_{ 4 } - L_{ 3 }
                        \end{array}
                    }
                    \begin{bmatrix}
                        a & a & a & a  \\
                        0 & b - a & b - a & b - a  \\
                        0 & 0     & c - b & c - b  \\
                        0 & 0     & 0     & d - c 
                    \end{bmatrix} = U
                .\end{align*}
                Pelos coeficientes utilizados na eliminação, temos
                \begin{equation*}
                    L =
                    \begin{bmatrix}
                        1 & 0 & 0 & 0 \\
                        1 & 1 & 0 & 0 \\
                        1 & 1 & 1 & 0 \\
                        1 & 1 & 1 & 1
                    \end{bmatrix}
                .\end{equation*}
                Pela diagonal de \( U \) vemos que, para que \( A \) tenha quatro pivôs, é necessário e suficiente que tenhamos:
                \begin{equation*}
                    \begin{cases}
                        a \neq 0 \\
                        b \neq a \\
                        c \neq b \\
                        d \neq c
                    \end{cases}
                .\end{equation*}
            \end{sol} 

        \item Ache a uma matriz de permutação $P$ tal que:

            \begin{enumerate}

                \item $P$ é 3x3, $P \neq I$ e $P^3 = I$.

                    \begin{sol} 
                        Intuitivamente, nossa matriz desloca uma casa para cima todas as linhas da matriz sobre a qual ela opera.
                        Ou seja, nossa matriz é
                        \begin{equation*}
                            P =
                            \begin{bmatrix}
                                0 & 1 & 0 \\
                                0 & 0 & 1 \\
                                1 & 0 & 0
                            \end{bmatrix}
                        .\end{equation*}
                        Com isso, \( P \neq I \) e
                        \begin{align*}
                            P^2 &=
                            \begin{bmatrix}
                                0 & 0 & 1 \\
                                1 & 0 & 0 \\
                                0 & 1 & 0 \\
                            \end{bmatrix} \\
                            P^3 &=
                            \begin{bmatrix}
                                1 & 0 & 0 \\
                                0 & 1 & 0 \\
                                0 & 0 & 1 \\
                            \end{bmatrix} = I
                        .\end{align*}
                    \end{sol} 

                \item $S$ é 4x4 e $S^4 \neq I$

                    \begin{sol} 
                        Observe que se tomarmos a matriz do item anterior, de fato teremos \( P^{ 4 } = P \neq I \).
                        Entretanto, precisamos de uma matriz \( 4 \times 4 \).
                        Felizmente, basta ``imbutir'' \( P \) em uma matriz identidade \( 4 \times 4 \), obtendo, assim,
                        \begin{equation*}
                            S =
                            \begin{bmatrix}
                                P & 0 \\
                                0 & 1
                            \end{bmatrix}
                            =
                            \begin{bmatrix}
                                0 & 1 & 0 & 0 \\
                                0 & 0 & 1 & 0 \\
                                1 & 0 & 0 & 0 \\
                                0 & 0 & 0 & 1
                            \end{bmatrix}
                        .\end{equation*}
                        Com isso,
                        \begin{equation*}
                            S^{ 4 } =
                            \begin{bmatrix}
                                P^{ 4 } & 0 \\
                                0 & 1
                            \end{bmatrix}
                            =
                            \begin{bmatrix}
                                P & 0 \\
                                0 & 1
                            \end{bmatrix}
                            = S \neq I
                        .\end{equation*}
                    \end{sol} 

            \end{enumerate}

        \item Seja $A$ uma matriz $4x4$. Quantas entradas de $A$ podem ser escolhidas independentemente caso $A$ seja

            \begin{enumerate}

                \item simétrica ($A^T = A$)?

                    \begin{sol} 
                        Precisamos apenas selecionar os elementos da diagonal e todos aqueles abaixo dela, por exemplo.
                        Logo, o número de entradas que podem ser escolhidas independentemente é \( 10 \).
                    \end{sol} 

                \item anti-simétrica ($A^T = -A$)?

                    \begin{sol} 
                        A resposta é parecida com a do item anterior, porém a diagonal já está determinada, pois deve ser igual a \( 0 \).
                        Portanto, apenas \( 6 \) entradas podem ser escolhidas independentemente.
                    \end{sol} 

            \end{enumerate}

        \item Suponha que $A$ já é triangular inferior com 1's na diagonal. Mostre que $U = I$.

            \begin{sol} 
                Observe que \( A \) pode ser decomposta como \( A = A I \), onde \( A \) é uma matriz triangular inferior com 1's na diagonal e \( I \) é uma matriz triangular superior com diagonal não nula.
                Portanto, pela unicidade da decomposição \( LU \), temos \( U = I \).
            \end{sol} 

        \item Seja 
            $$
            A 
            \begin{bmatrix}
                1 & c & 0  \\
                2 & 4 & 1  \\
                3 & 5 & 1 
            \end{bmatrix}.
            $$

            \begin{enumerate}

                \item Qual é o número $c$ que leva o segundo pivô a ser 0? O que podemos fazer para resolver tal problema? Ainda é válido $A = LU$?

                    \begin{sol} 
                        Realizando a primeira eliminação em \( A \) obtemos
                        \begin{equation*}
                            A
                            \xrightarrow{
                                \begin{array}{l}
                                    L_{ 2 } - 2 L_{ 1 }
                                \end{array}
                            }
                            \begin{bmatrix}
                                1 & c & 0  \\
                                0 & 4 - 2 c & 1  \\
                                3 & 5 & 1 
                            \end{bmatrix}
                        .\end{equation*}
                        Logo, se \( c = 2 \) temos \( 4 - 2c = 0 \), fazendo aparecer um pivô nulo.
                        Para concertar isso, podemos, antes de começar o processo de eliminação, permutar a segunda e a terceira linhas utilizando a matriz de permutação
                        \begin{equation*}
                            P =
                            \begin{bmatrix}
                                1 & 0 & 0 \\
                                0 & 0 & 1 \\
                                0 & 1 & 0
                            \end{bmatrix}
                        .\end{equation*}
                        Com isso, supondo \( c = 2 \), obtemos a matriz
                        \begin{equation*}
                            PA =
                            \begin{bmatrix}
                                1 & 2 & 0 \\
                                3 & 5 & 1 \\
                                2 & 4 & 1
                            \end{bmatrix}
                        .\end{equation*}
                        Então, realizando o primeiro passo da eliminação ficamos com
                        \begin{equation*}
                            \begin{bmatrix}
                                1 & 2 & 0 \\
                                3 & 5 & 1 \\
                                2 & 4 & 1
                            \end{bmatrix}
                            \xrightarrow{
                                \begin{array}{l}
                                    L_{ 2 } - 3L_{ 1 }
                                \end{array}
                            }
                            \begin{bmatrix}
                                1 & 2 & 0 \\
                                0 & -1 & 1 \\
                                2 & 4 & 1
                            \end{bmatrix}
                        ,\end{equation*}
                        e o segundo pivô não é mais nulo.
                        Continuando o processo de eliminação, obtemos
                        \begin{align*}
                            \begin{bmatrix}
                                1 & 2 & 0 \\
                                0 & -1 & 1 \\
                                2 & 4 & 1
                            \end{bmatrix}
                                &\xrightarrow{
                                    \begin{array}{l}
                                        L_{ 3 } - 2 L_{ 1 }
                                    \end{array}
                                }
                                \begin{bmatrix}
                                    1 & 2 & 0 \\
                                    0 & -1 & 1 \\
                                    0 & 0 & 1
                                \end{bmatrix} = U
                            .\end{align*}
                            Dessa forma, não temos mais uma decomcoposição \( L = LU \), mas sim \( PA = LU \), dada por
                            \begin{equation*}
                                \underbrace{
                                    \begin{bmatrix}
                                        1 & 0 & 0 \\
                                        0 & 0 & 1 \\
                                        0 & 1 & 0
                                    \end{bmatrix}
                                }_{ P }
                                \underbrace{
                                    \begin{bmatrix}
                                        1 & 2 & 0  \\
                                        2 & 4 & 1  \\
                                        3 & 5 & 1 
                                    \end{bmatrix}
                                }_{ A }
                                =
                                \underbrace{
                                    \begin{bmatrix}
                                        1 & 0 & 0 \\
                                        3 & 1 & 0 \\
                                        2 & 0 & 1
                                    \end{bmatrix}
                                }_{ L }
                                \underbrace{
                                    \begin{bmatrix}
                                        1 & 2 & 0 \\
                                        0 & -1 & 1 \\
                                        0 & 0 & 1
                                    \end{bmatrix}
                                }_{ U }
                            .\end{equation*}
                        \end{sol} 

                    \item Qual é o número $c$ que leva o terceiro pivô a ser 0? É possível resolver esse problema?

                        \begin{sol} 
                            Prosseguindo com a eliminação em \( A \), ficamos com
                            \begin{align*}
                                A
                                &\xrightarrow{
                                    \begin{array}{l}
                                        L_{ 2 } - 2 L_{ 1 }
                                        L_{ 3 } - 3 L_{ 1 }
                                    \end{array}
                                }
                                \begin{bmatrix}
                                    1 & c & 0  \\
                                    0 & 4 - 2 c & 1  \\
                                    0 & 5 - 3 c & 1 
                                \end{bmatrix} \\
                                &\xrightarrow{
                                    \begin{array}{l}
                                        L_{ 3 } - \frac{ 5 - 3c }{ 4 - 2 c } L_{ 2 }
                                    \end{array}
                                }
                                \begin{bmatrix}
                                    1 & c & 0  \\
                                    0 & 4 - 2 c & 1  \\
                                    0 & 0       & 1 - \frac{ 5 - 3c }{ 4 - 2c } 
                                \end{bmatrix}
                            .\end{align*}
                            Logo, o terceiro pivô será zero se, e somente se,
                            \begin{equation*}
                                5 - 3c = 4 - 2c
                            ,\end{equation*}
                            ou seja, se \( c = 1 \).
                            Nesse caso, oberve que o problema não tem concerto, pois mesmo que trocássemos a terceira linha com a segunda, continuaríamos com um pivô nulo, porém, agora, seria o segundo.
                            Analogamente, se trocarmos a terceira linha com a primeira, o primeiro pivô passará a ser nulo.
                        \end{sol} 

                \end{enumerate}

            \item Se $A$ e $B$ são simétricas, quais dessas matrizes são também simétricas:

                \begin{enumerate}

                    \item $A^2 - B^2$;

                        \begin{sol} 
                            Observe que
                            \begin{align*}
                                \left(
                                    A^2 - B^2
                                \right)^{ \mathsf{T} }
                                &=
                                \left(
                                    A^2
                                \right)^{ \transpose }
                                -
                                \left(
                                    B^2
                                \right)^{ \transpose } \\
                                &= ( AA )^{ \transpose } - ( BB )^{ \transpose } \\
                                &= ( A^{ \transpose } A^{ \transpose } )
                                -
                                ( B^{ \transpose } B^{ \transpose } ) \\
                                &= ( AA ) - ( BB ) \\
                                &= A^2 - B^2
                            .\end{align*}
                            Logo, \( A^2 - B^2 \) também é simétrica.
                        \end{sol} 

                    \item $(A + B)(A - B)$;

                        \begin{sol} 
                            Observe que \( ( A + B )( A - B ) = A^2 + BA - AB + B^2 \).
                            Já sabemos que \( A^2 - B^2 \) é simétrica.
                            Como a soma de matrizes simétricas é simétrica, vamos verificar se \( BA - AB \) é simétrica:
                            \begin{align*}
                                ( BA - AB )^{ \transpose }
                                &= ( BA )^{ \transpose } - ( AB )^{ \transpose } \\
                                &= A^{ \transpose } B^{ \transpose }
                                - B^{ \transpose } A^{ \transpose } \\
                                &= AB - BA \\
                                &= - ( BA - AB )
                            .\end{align*}
                            Sendo assim, \( BA - AB \) é antissimétrica e, com isso, a menos que \( A \) e \( B \) comutem, de modo que \( BA - AB = 0 \), a matriz \( ( A + B )( A - B ) \) não será simétrica.
                        \end{sol} 

                    \item $ABA$;

                        \begin{sol} 
                            Temos
                            \begin{align*}
                                ( ABA )^{ \transpose }
                                &= A^{ \transpose } B^{ \transpose } A^{ \transpose } \\
                                &= ABA
                            .\end{align*}
                            Logo, \( ABA \) é simétrica.
                        \end{sol} 

                    \item $ABAB$.

                        \begin{sol} 
                            Calculando a transposta, obtemos
                            \begin{align*}
                                ( ABAB )^{ \transpose }
                                &= B^{ \transpose } A^{ \transpose } B^{ \transpose } A^{ \transpose } \\
                                &= BABA
                            .\end{align*}
                            Portanto, a menos que \( A \) e \( B \) comutem, \( ABAB \) não será simétrica.
                        \end{sol} 

                \end{enumerate}

            \item Prove que é sempre possível escrever $A = B + C$, onde $B$ é simétrica e $C$ anti-simétrica. \textit{Dica: $B$ e $C$ são combinações simples de $A$ e $A^T$.}

                \begin{sol} 
                    Tome
                    \begin{equation*}
                        \begin{cases}
                            B = \frac{ 1 }{ 2 } \left(
                                A + A^{ \transpose }
                            \right) \\
                            C = \frac{ 1 }{ 2 } \left(
                                A - A^{ \transpose }
                            \right)
                        \end{cases}
                    .\end{equation*}
                    Com isso, claramente temos \( A = C + B \) e, ainda,
                    \begin{align*}
                        B^{ \transpose } =
                        \frac{ 1 }{ 2 }
                        \left(
                            A + A^{ \transpose }
                        \right)^{ \transpose }
                        = 
                        \frac{ 1 }{ 2 }
                        \left(
                            A^{ \transpose } + A
                        \right)
                        = B, \\
                        C^{ \transpose } =
                        \frac{ 1 }{ 2 }
                        \left(
                            A - A^{ \transpose }
                        \right)^{ \transpose }
                        = 
                        \frac{ 1 }{ 2 }
                        \left(
                            A^{ \transpose } - A
                        \right)
                        = -C \\
                    ,\end{align*}
                    ou seja, \( B \) é simétrica e \( C \) é antissimétrica.
                \end{sol} 

            \item Seja $A$ uma matriz em blocos:
                $$A = \begin{bmatrix}
                    A_{11} & A_{12}\\
                    A_{21} & A_{22}
                \end{bmatrix}$$
                onde cada $A_{ii}$ é quadrada $n \times n$ e as matrizes $A_{11}$ e \( A_{ 22 } - A_{ 21 } A_{ 11 }^{ -1 } A_{ 12 } \) possuem decomposição LU.
                Ache $L$ e $U$ em blocos tal que $A = LU$:
                $$L = \begin{bmatrix}
                    L_{11} & L_{12}\\
                    L_{21} & L_{22}
                    \end{bmatrix} \mbox{ e } U = \begin{bmatrix}
                    U_{11} & U_{12}\\
                    U_{21} & U_{22}
                \end{bmatrix},$$
                onde $L_{11}$, $L_{22}$ são triangulares inferiores com 1's na diagonal e $U_{11}$, $U_{22}$ são triangulares superiores.
        \end{enumerate}

        \begin{sol} 
            Vamos fazer uma espécie de ``eliminação em blocos'' na matriz \( A \).
            Perceba que
            \begin{equation*}
                \underbrace{
                    \begin{bmatrix}
                        I & 0 \\
                        - A_{ 21 } A_{ 11 }^{ -1 } & I
                    \end{bmatrix}
                }_{ \text{Matriz de eliminação} }
                \begin{bmatrix}
                    A_{ 11 } & A_{ 12 } \\
                    A_{ 21 } & A_{ 22 }
                \end{bmatrix}
                =
                \begin{bmatrix}
                    A_{ 11 } & A_{ 12 } \\
                    0 & A_{ 22 } - A_{ 21 } A_{ 11 }^{ -1 } A_{ 12 }
                \end{bmatrix}
            .\end{equation*}
            Com isso, multiplicando ambos lados à esquerda pela inversa da matriz de eliminação temos
            \begin{equation*}
                A =
                \begin{bmatrix}
                    I & 0 \\
                    A_{ 21 } A_{ 11 }^{ -1 } & I
                \end{bmatrix}
                \begin{bmatrix}
                    A_{ 11 } & A_{ 12 } \\
                    0 & A_{ 22 } - A_{ 21 } A_{ 11 }^{ -1 } A_{ 12 }
                \end{bmatrix}
            .\end{equation*}
            Sejam, agora, \( L_{ 1 }, L_{ 2 } \) matrizes triangulares inferiores com 1's na diagonal e \( U_{ 1 }, U_{ 2 } \) matrizes triangulares superiores tais que
            \begin{equation*}
                \begin{cases}
                    A_{ 11 } = L_{ 1 } U_{ 1 } \\
                    A_{ 22 } - A_{ 21 } A_{ 11 }^{ -1 } A_{ 12 } = L_{ 2 } U_{ 2 }
                \end{cases}
            ,\end{equation*}
            as quais existem por hipótese.
            Com isso, temos
            \begin{align*}
                A &=
                \begin{bmatrix}
                    I & 0 \\
                    A_{ 21 } A_{ 11 }^{ -1 } & I
                \end{bmatrix}
                \begin{bmatrix}
                    A_{ 11 } & A_{ 12 } \\
                    0 & A_{ 22 } - A_{ 21 } A_{ 11 }^{ -1 } A_{ 12 }
                \end{bmatrix} \\
                  &= 
                  \begin{bmatrix}
                      I & 0 \\
                      A_{ 21 } A_{ 11 }^{ -1 } & I
                  \end{bmatrix}
                  \begin{bmatrix}
                      L_{ 1 } U_{ 1 } & A_{ 12 } \\
                      0 & L_{ 2 } U_{ 2 }
                  \end{bmatrix} \\
                  &= 
                  \begin{bmatrix}
                      I & 0 \\
                      A_{ 21 } A_{ 11 }^{ -1 } & I
                  \end{bmatrix}
                  \begin{bmatrix}
                      L_{ 1 } & 0 \\
                      0 & L_{ 2 }
                  \end{bmatrix}
                  \begin{bmatrix}
                      U_{ 1 } & L_{ 1 }^{ -1 } A_{ 12 } \\
                      0 & U_{ 2 }
                  \end{bmatrix} \\
                  &=
                  \begin{bmatrix}
                      L_{ 1 } & 0 \\
                      A_{ 21 } U_{ 1 }^{ -1 } & L_{ 2 }
                  \end{bmatrix}
                  \begin{bmatrix}
                      U_{ 1 } & L_{ 1 }^{ -1 } A_{ 12 } \\
                      0 & U_{ 2 }
                  \end{bmatrix} \\
                  &= LU
              .\end{align*}
          \end{sol} 

          \end{document} 
