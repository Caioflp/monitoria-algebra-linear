\documentclass[leqno]{article}

\usepackage[brazil]{babel} % \usepackage[latin1]{inputenc}
\usepackage{a4wide}
\setlength{\oddsidemargin}{-0.2in}
% % \setlength{\oddsidemargin}{0.2in}
\setlength{\evensidemargin}{-0.2in}
% % \setlength{\evensidemargin}{0.5in}
% % \setlength{\textwidth}{5.5in}
\setlength{\textwidth}{6.5in}
\setlength{\topmargin}{-1.2in}
\setlength{\textheight}{10in}
\usepackage[]{amsfonts} \usepackage[]{amsmath}
\usepackage[]{amssymb} \usepackage[]{latexsym}
\usepackage{graphicx,color} \usepackage{amsthm}
\usepackage{mathrsfs} \usepackage{url}
\usepackage{cancel} \usepackage[inline]{enumitem}
\usepackage{xifthen} \usepackage{tikz}
\usetikzlibrary{automata,arrows,positioning,calc}

\numberwithin{equation}{section}

\setlength{\parindent}{12 pt}

\begin{document}

\newtheorem{teo}{Teorema}[section] \newtheorem*{teo*}{Teorema}
\newtheorem{prop}[teo]{Proposição} \newtheorem*{prop*}{Proposição}
\newtheorem{lema}[teo]{Lemma} \newtheorem*{lema*}{Lema}
\newtheorem{cor}[teo]{Corolário} \newtheorem*{cor*}{Corolário}

\theoremstyle{definition}
\newtheorem{defi}[teo]{Definição} \newtheorem*{defi*}{Definição}
\newtheorem{exem}[teo]{Exemplo} \newtheorem*{exem*}{Exemplo}
\newtheorem{obs}[teo]{Observação} \newtheorem*{obs*}{Observação}
\newtheorem*{hipo}{Hipóteses}
\newtheorem*{nota}{Notação}

\newcommand{\ds}{\displaystyle} \newcommand{\nl}{\newline}
\newcommand{\eps}{\varepsilon} \newcommand{\ssty}{\scriptstyle}
\newcommand{\bE}{\mathbb{E}}
\newcommand{\cB}{\mathcal{B}}
\newcommand{\cF}{\mathcal{F}}
\newcommand{\cA}{\mathcal{A}}
\newcommand{\cM}{\mathcal{M}}
\newcommand{\cD}{\mathcal{D}}
\newcommand{\cN}{\mathcal{N}}
\newcommand{\cL}{\mathcal{L}}
\newcommand{\cLN}{\mathcal{LN}}
\newcommand{\bP}{\mathbb{P}}
\newcommand{\bQ}{\mathbb{Q}}
\newcommand{\bN}{\mathbb{N}}
\newcommand{\bR}{\mathbb{R}}
\newcommand{\bZ}{\mathbb{Z}}

\newcommand{\bfw}{\mathbf{w}}
\newcommand{\bfv}{\mathbf{v}}
\newcommand{\bfu}{\mathbf{u}}
\newcommand{\bfx}{\mathbf{x}}
\newcommand{\bfb}{\mathbf{b}}

\newcommand{\bvecc}[2]{%
  \begin{bmatrix} #1 \\ #2  \end{bmatrix}
}
\newcommand{\bveccc}[3]{%
  \begin{bmatrix} #1 \\ #2 \\ #3  \end{bmatrix}
}

\newenvironment{sol}
{
    \vspace{4mm}
    \noindent\textbf{Resolução:}
    \strut\newline
    \smallskip
    \hspace{-3.5mm}
}
% Objetos que aparecem *após* o ambiente; 
% nestas configurações, estamos desenhando uma 
% linha horizontal. 
% (você pode, por exemplo, modificar 
% ou remover este elemento gráfico) 
{\noindent\rule{4cm}{.1mm}}


\title{Álgebra Linear - Lista de Exercícios 10}

\author{escreva seu nome aqui}

\date{\today}

\maketitle

\begin{enumerate}

%%%%%%%%%%%%%%%%%%%%%%%%%%%%%%%%%%%%%%%%%%%%%%%%%%%%%%%%%
%%%%%%%%%%%%%%%%%%%%%% Exercício 1 %%%%%%%%%%%%%%%%%%%%%%
%%%%%%%%%%%%%%%%%%%%%%%%%%%%%%%%%%%%%%%%%%%%%%%%%%%%%%%%%

\item Seja $A = \begin{bmatrix}
1 & b \\
b & 1
\end{bmatrix}$.

\begin{enumerate}

\item Ache $b$ tal que $A$ tenha um autovalor negativo.

	\begin{sol} 
		% escreva sua solução aqui.    
	\end{sol} 

\item Como podemos concluir que $A$ precisa ter um pivô negativo?

	\begin{sol} 
		% escreva sua solução aqui.    
	\end{sol} 

\item Como podemos concluir que $A$ não pode ter dois autovalores negativos?

	\begin{sol} 
		% escreva sua solução aqui.    
	\end{sol} 

\end{enumerate}

%%%%%%%%%%%%%%%%%%%%%%%%%%%%%%%%%%%%%%%%%%%%%%%%%%%%%%%%%
%%%%%%%%%%%%%%%%%%%%%% Exercício 2 %%%%%%%%%%%%%%%%%%%%%%
%%%%%%%%%%%%%%%%%%%%%%%%%%%%%%%%%%%%%%%%%%%%%%%%%%%%%%%%%

\item Em quais das seguintes classes as matrizes $A$ e $B$ abaixo pertencem: invertível, ortogonal, projeção, permutação, diagonalizável, Markov?
$$A = \begin{bmatrix}
0 & 0 & 1 \\
0 & 1 & 0 \\
1 & 0 & 0
\end{bmatrix} \mbox{ e } B = \frac{1}{3}\begin{bmatrix}
1 & 1 & 1 \\
1 & 1 & 1 \\
1 & 1 & 1
\end{bmatrix}.$$
Quais das seguintes fatorações são possíveis para $A$ e $B$? $LU$, $QR$, $S\Lambda S^{-1}$ ou $Q\Lambda Q^T$?

\begin{sol} 
	% escreva sua solução aqui.    
\end{sol} 

%%%%%%%%%%%%%%%%%%%%%%%%%%%%%%%%%%%%%%%%%%%%%%%%%%%%%%%%%
%%%%%%%%%%%%%%%%%%%%%% Exercício 3 %%%%%%%%%%%%%%%%%%%%%%
%%%%%%%%%%%%%%%%%%%%%%%%%%%%%%%%%%%%%%%%%%%%%%%%%%%%%%%%%

\item Complete a matriz $A$ abaixo para que seja de Markov e ache o autovetor estacionário. Sua conclusão é válida para qualquer matriz simétrica de Markov $A$? Por quê?
$$A = \begin{bmatrix}
0.7 & 0.1 & 0.2 \\
0.1 & 0.6 & 0.3 \\
* & * & *
\end{bmatrix}$$

\begin{sol} 
	% escreva sua solução aqui.    
\end{sol} 

%%%%%%%%%%%%%%%%%%%%%%%%%%%%%%%%%%%%%%%%%%%%%%%%%%%%%%%%%
%%%%%%%%%%%%%%%%%%%%%% Exercício 4 %%%%%%%%%%%%%%%%%%%%%%
%%%%%%%%%%%%%%%%%%%%%%%%%%%%%%%%%%%%%%%%%%%%%%%%%%%%%%%%%

\item Dizemos que $\cM$ é um grupo de matrizes invertíveis se $A, B \in \cM$ implica $AB \in \cM$ e $A^{-1} \in \cM$. Quais dos conjuntos abaixo é um grupo?  

\begin{enumerate}

\item O conjunto das matrizes positivas definidas;

\item o conjunto das matrizes ortogonais;

\item o conjunto $\{e^{tC} \ ; \ t \in \bR\}$, para uma matriz $C$ fixa;

\item o conjunto das matrizes com determinante igual a 1.

\end{enumerate}

\begin{sol} 
	\vspace{-\baselineskip}  
	\begin{enumerate} 
		\item % item (a) 
		\item % item (b) 
		\item % item (c) 
		\item % item (d) 
	\end{enumerate} 
\end{sol} 

%%%%%%%%%%%%%%%%%%%%%%%%%%%%%%%%%%%%%%%%%%%%%%%%%%%%%%%%%
%%%%%%%%%%%%%%%%%%%%%% Exercício 5 %%%%%%%%%%%%%%%%%%%%%%
%%%%%%%%%%%%%%%%%%%%%%%%%%%%%%%%%%%%%%%%%%%%%%%%%%%%%%%%%

\item Sejam $A$ e $B$ matrizes simétricas e positivas definidas. Prove que os autovalores de $AB$ são positivos. Podemos dizer que $AB$ é simétrica e positiva definida?

\begin{sol} 
	% escreva sua solução aqui.    
\end{sol} 

%%%%%%%%%%%%%%%%%%%%%%%%%%%%%%%%%%%%%%%%%%%%%%%%%%%%%%%%%
%%%%%%%%%%%%%%%%%%%%%% Exercício 6 %%%%%%%%%%%%%%%%%%%%%%
%%%%%%%%%%%%%%%%%%%%%%%%%%%%%%%%%%%%%%%%%%%%%%%%%%%%%%%%%

\item Ache a forma quadrática associada à matriz $A = \begin{bmatrix}
1 & 5 \\
7 & 9
\end{bmatrix}$. Qual o sinal dessa forma quadrática? Positivo, negativo ou ambos?

\begin{sol} 
	% escreva sua solução aqui.    
\end{sol} 

%%%%%%%%%%%%%%%%%%%%%%%%%%%%%%%%%%%%%%%%%%%%%%%%%%%%%%%%%
%%%%%%%%%%%%%%%%%%%%%% Exercício 7 %%%%%%%%%%%%%%%%%%%%%%
%%%%%%%%%%%%%%%%%%%%%%%%%%%%%%%%%%%%%%%%%%%%%%%%%%%%%%%%%

\item Prove os seguintes fatos:

\begin{enumerate}

\item Se $A$ e $B$ são similares, então $A^2$ e $B^2$ também o são.

\begin{sol} 
	% escreva sua solução aqui.    
\end{sol} 

\item $A^2$ e $B^2$ podem ser similares sem $A$ e $B$ serem similares.

\begin{sol} 
	% escreva sua solução aqui.    
\end{sol} 

\item $\begin{bmatrix}
3 & 0 \\
0 & 4
\end{bmatrix}$ é similar à $\begin{bmatrix}
3 & 1 \\
0 & 4
\end{bmatrix}$.

\begin{sol} 
	% escreva sua solução aqui.    
\end{sol} 

\item $\begin{bmatrix}
3 & 0 \\
0 & 3
\end{bmatrix}$ não é similar à $\begin{bmatrix}
3 & 1 \\
0 & 3
\end{bmatrix}$.

\begin{sol} 
	% escreva sua solução aqui.    
\end{sol} 

\end{enumerate}

%%%%%%%%%%%%%%%%%%%%%%%%%%%%%%%%%%%%%%%%%%%%%%%%%%%%%%%%%
%%%%%%%%%%%%%%%%%%%%%% Exercício 8 %%%%%%%%%%%%%%%%%%%%%%
%%%%%%%%%%%%%%%%%%%%%%%%%%%%%%%%%%%%%%%%%%%%%%%%%%%%%%%%%

\item Ache os valores singulares (como na decomposição SVD) da matriz $A = \begin{bmatrix}
1 & 1 \\
1 & 0
\end{bmatrix}$.

\begin{sol} 
	% escreva sua solução aqui.    
\end{sol} 

%%%%%%%%%%%%%%%%%%%%%%%%%%%%%%%%%%%%%%%%%%%%%%%%%%%%%%%%%
%%%%%%%%%%%%%%%%%%%%%% Exercício 9 %%%%%%%%%%%%%%%%%%%%%%
%%%%%%%%%%%%%%%%%%%%%%%%%%%%%%%%%%%%%%%%%%%%%%%%%%%%%%%%%

\item Suponha que as colunas de $A$ sejam $\bfw_1, \ldots, \bfw_n$ que são vetores ortogonais com comprimentos $\sigma_1, \ldots, \sigma_n$. Calcule $A^TA$. Ache a decomposição SVD de $A$.

\begin{sol} 
	% escreva sua solução aqui.    
\end{sol} 

\end{enumerate}
\end{document} 
