\documentclass[leqno]{article}

\usepackage[brazil]{babel} % \usepackage[latin1]{inputenc}
\usepackage{a4wide}
\setlength{\oddsidemargin}{-0.2in}
% % \setlength{\oddsidemargin}{0.2in}
\setlength{\evensidemargin}{-0.2in}
% % \setlength{\evensidemargin}{0.5in}
% % \setlength{\textwidth}{5.5in}
\setlength{\textwidth}{6.5in}
\setlength{\topmargin}{-1.2in}
\setlength{\textheight}{10in}
\usepackage[]{amsfonts} \usepackage[]{amsmath}
\usepackage[]{amssymb} \usepackage[]{latexsym}
\usepackage{graphicx,color} \usepackage{amsthm}
\usepackage{mathrsfs} \usepackage{url}
\usepackage{cancel} \usepackage[inline]{enumitem}
\usepackage{xifthen} \usepackage{tikz}
\usepackage{mathtools}
\usetikzlibrary{automata,arrows,positioning,calc}

\DeclareMathOperator{\vspan}{Span}
\DeclareMathOperator{\sen}{sen}

\numberwithin{equation}{section}

\setlength{\parindent}{12 pt}

\newenvironment{sol} 
{
    \vspace{4mm}
    \noindent\textbf{Resolução:}
    \strut\newline
    \smallskip
    \hspace{-3.5mm} 
} 
% Objetos que aparecem *após* o ambiente. 
% (você pode, por exemplo, modificar, 
% ou remover, a barra horizontal} 
{\noindent\rule{4cm}{.1mm}}

\begin{document}

\newtheorem{teo}{Teorema}[section] \newtheorem*{teo*}{Teorema}
\newtheorem{prop}[teo]{Proposição} \newtheorem*{prop*}{Proposição}
\newtheorem{lema}[teo]{Lemma} \newtheorem*{lema*}{Lema}
\newtheorem{cor}[teo]{Corolário} \newtheorem*{cor*}{Corolário}

\theoremstyle{definition}
\newtheorem{defi}[teo]{Definição} \newtheorem*{defi*}{Definição}
\newtheorem{exem}[teo]{Exemplo} \newtheorem*{exem*}{Exemplo}
\newtheorem{obs}[teo]{Observação} \newtheorem*{obs*}{Observação}
\newtheorem*{hipo}{Hipóteses}
\newtheorem*{nota}{Notação}

\newcommand{\ds}{\displaystyle} \newcommand{\nl}{\newline}
\newcommand{\eps}{\varepsilon} \newcommand{\ssty}{\scriptstyle}
\newcommand{\bE}{\mathbb{E}}
\newcommand{\cB}{\mathcal{B}}
\newcommand{\cF}{\mathcal{F}}
\newcommand{\cA}{\mathcal{A}}
\newcommand{\cM}{\mathcal{M}}
\newcommand{\cD}{\mathcal{D}}
\newcommand{\cN}{\mathcal{N}}
\newcommand{\cL}{\mathcal{L}}
\newcommand{\cLN}{\mathcal{LN}}
\newcommand{\bP}{\mathbb{P}}
\newcommand{\bQ}{\mathbb{Q}}
\newcommand{\bN}{\mathbb{N}}
\newcommand{\bR}{\mathbb{R}}
\newcommand{\bZ}{\mathbb{Z}}
\newcommand{\R}{\mathbb{R}}
\newcommand{\defeq}{\vcentcolon =}

\newcommand{\bfw}{\mathbf{w}}
\newcommand{\bfv}{\mathbf{v}}
\newcommand{\bfu}{\mathbf{u}}
\newcommand{\bfx}{\mathbf{x}}
\newcommand{\bfb}{\mathbf{b}}

\newcommand{\bvecc}[2]{%
  \begin{bmatrix} #1 \\ #2  \end{bmatrix}
}
\newcommand{\bveccc}[3]{%
  \begin{bmatrix} #1 \\ #2 \\ #3  \end{bmatrix}
}


\title{Álgebra Linear - Lista de Exercícios 9}

\author{Meu nome Sensacional}

\date{\today}

\maketitle

\begin{enumerate}

%%%%%%%%%%%%%%%%%%%%%%%%%%%%%%%%%%%%%%%%%%%%%%%%%%%%%%%%%
%%%%%%%%%%%%%%%%%%%%%% Exercício 1 %%%%%%%%%%%%%%%%%%%%%%
%%%%%%%%%%%%%%%%%%%%%%%%%%%%%%%%%%%%%%%%%%%%%%%%%%%%%%%%%

\item Seja $B$ uma matriz $3 \times 3$ com autovalores 0, 1 e 2. Com  essa informação, ache:

\begin{enumerate}

\item o posto de $B$;

\item o determinante de $B^TB$;

\item os autovalores de $B^TB$;

\item os autovalores de $(B^2 + I)^{-1}$.

\end{enumerate}

\begin{sol}

    % Minha solução genial
    
\end{sol}

%%%%%%%%%%%%%%%%%%%%%%%%%%%%%%%%%%%%%%%%%%%%%%%%%%%%%%%%%
%%%%%%%%%%%%%%%%%%%%%% Exercício 2 %%%%%%%%%%%%%%%%%%%%%%
%%%%%%%%%%%%%%%%%%%%%%%%%%%%%%%%%%%%%%%%%%%%%%%%%%%%%%%%%

\item Ache os autovalores das seguintes matrizes

\begin{enumerate*}

\item $A = \begin{bmatrix}
1 & 2 & 3\\
0 & 4 & 5 \\
0 & 0 & 6 
\end{bmatrix}$;

\item $B = \begin{bmatrix}
0 & 0 & 1\\
0 & 2 & 0 \\
3 & 0 & 0 
\end{bmatrix}$;

\item $C = \begin{bmatrix}
2 & 2 & 2\\
2 & 2 & 2 \\
2 & 2 & 2 
\end{bmatrix}$.

\end{enumerate*}

\begin{sol}

    % Minha solução genial
    
\end{sol}

%%%%%%%%%%%%%%%%%%%%%%%%%%%%%%%%%%%%%%%%%%%%%%%%%%%%%%%%%
%%%%%%%%%%%%%%%%%%%%%% Exercício 3 %%%%%%%%%%%%%%%%%%%%%%
%%%%%%%%%%%%%%%%%%%%%%%%%%%%%%%%%%%%%%%%%%%%%%%%%%%%%%%%%

\item Descreva todas as matrizes $S$ que diagonalizam as matrizes $A$ e $A^{-1}$:
$$A = \begin{bmatrix}
0 & 4 \\
1 & 2 
\end{bmatrix}.$$

\begin{sol}

    % Minha solução genial
    
\end{sol}

%%%%%%%%%%%%%%%%%%%%%%%%%%%%%%%%%%%%%%%%%%%%%%%%%%%%%%%%%
%%%%%%%%%%%%%%%%%%%%%% Exercício 4 %%%%%%%%%%%%%%%%%%%%%%
%%%%%%%%%%%%%%%%%%%%%%%%%%%%%%%%%%%%%%%%%%%%%%%%%%%%%%%%%

\item Ache $\Lambda$ e $S$ que diagonalizem $A$
$$A = \begin{bmatrix}
0.6 & 0.9 \\
0.4 & 0.1
\end{bmatrix}.$$
Qual limite de $\Lambda^k$ quando $k \to +\infty$? E o limite de $A^k$?

\begin{sol}

    % Minha solução genial
    
\end{sol}

%%%%%%%%%%%%%%%%%%%%%%%%%%%%%%%%%%%%%%%%%%%%%%%%%%%%%%%%%
%%%%%%%%%%%%%%%%%%%%%% Exercício 5 %%%%%%%%%%%%%%%%%%%%%%
%%%%%%%%%%%%%%%%%%%%%%%%%%%%%%%%%%%%%%%%%%%%%%%%%%%%%%%%%

\item Seja $Q(\theta)$ a matriz de rotação do ângulo $\theta$ em $\bR^2$:
$$Q(\theta) = \begin{bmatrix}
\cos \theta & -\mbox{sen} \theta \\
\mbox{sen} \theta & \cos \theta
\end{bmatrix}.$$
Ache os autovalores e autovetores de $Q(\theta)$ (eles podem ser complexos).

\begin{sol}

    % Minha solução genial
    
\end{sol}

%%%%%%%%%%%%%%%%%%%%%%%%%%%%%%%%%%%%%%%%%%%%%%%%%%%%%%%%%
%%%%%%%%%%%%%%%%%%%%%% Exercício 6 %%%%%%%%%%%%%%%%%%%%%%
%%%%%%%%%%%%%%%%%%%%%%%%%%%%%%%%%%%%%%%%%%%%%%%%%%%%%%%%%

\item Suponha que $A$ e $B$ são duas matrizes $n \times n$ com os mesmo autovalores $\lambda_1, \ldots, \lambda_n$ e os mesmos autovetores $x_1, \ldots, x_n$. Suponha ainda que $x_1, \ldots, x_n$ são LI. Prove que $A = B$.

\begin{sol}

    % Minha solução genial
    
\end{sol}

%\item Sejam $B$, $C$ e $D$ matrizes $2 \times 2$ com auto-valores $\{1,2\}$, $\{3,4\}$ e $\{5, 7\}$, respectivamente. Quais são os autovalores da matriz $4 \times 4$:
%$$A = \begin{bmatrix}
%B & C \\
%0 & D
%\end{bmatrix}.$$

%%%%%%%%%%%%%%%%%%%%%%%%%%%%%%%%%%%%%%%%%%%%%%%%%%%%%%%%%
%%%%%%%%%%%%%%%%%%%%%% Exercício 7 %%%%%%%%%%%%%%%%%%%%%%
%%%%%%%%%%%%%%%%%%%%%%%%%%%%%%%%%%%%%%%%%%%%%%%%%%%%%%%%%

\item Seja $Q(\theta)$ como na Questão 5. Diagonalize $Q(\theta)$ e mostre que
$$Q(\theta)^n = Q(n\theta).$$

\begin{sol}

    % Minha solução genial
    
\end{sol}

%%%%%%%%%%%%%%%%%%%%%%%%%%%%%%%%%%%%%%%%%%%%%%%%%%%%%%%%%
%%%%%%%%%%%%%%%%%%%%%% Exercício 8 %%%%%%%%%%%%%%%%%%%%%%
%%%%%%%%%%%%%%%%%%%%%%%%%%%%%%%%%%%%%%%%%%%%%%%%%%%%%%%%%

\item Suponha que $G_{k+2}$ é a média dos dois números anteriores $G_{k+1}$ e $G_k$. Ache a matriz $A$ que faz com que
$$\begin{bmatrix}
G_{k+2}\\
G_{k+1}\end{bmatrix} = A \begin{bmatrix}
G_{k+1}\\
G_k\end{bmatrix}.$$

\begin{enumerate}

\item Ache os autovalores e autovetores de $A$;

\item Ache o limite de $A^n$ quando $n \to +\infty$;

\item Mostre que $G_n$ converge para $2/3$ quando $G_0 = 0$ e $G_1 = 1$.

\end{enumerate}

\begin{sol}

    % Minha solução genial
    
\end{sol}

%%%%%%%%%%%%%%%%%%%%%%%%%%%%%%%%%%%%%%%%%%%%%%%%%%%%%%%%%
%%%%%%%%%%%%%%%%%%%%%% Exercício 9 %%%%%%%%%%%%%%%%%%%%%%
%%%%%%%%%%%%%%%%%%%%%%%%%%%%%%%%%%%%%%%%%%%%%%%%%%%%%%%%%

\item Ache a solução do sistema de EDOs usando o método de diagonalização:
$$\begin{cases}
u_1'(t) = 8u_1(t) + 3u_2(t),\\
u_2'(t) = 2u_1(t) + 7u_2(t),
\end{cases}$$
onde $u(0) = (5, 10)$.

\begin{sol}

    % Minha solução genial
    
\end{sol}

%%%%%%%%%%%%%%%%%%%%%%%%%%%%%%%%%%%%%%%%%%%%%%%%%%%%%%%%%
%%%%%%%%%%%%%%%%%%%%%% Exercício 10 %%%%%%%%%%%%%%%%%%%%%
%%%%%%%%%%%%%%%%%%%%%%%%%%%%%%%%%%%%%%%%%%%%%%%%%%%%%%%%%

\item Seja \( \mathcal{F} ( \R; \R ) \) o espaço vetorial das funções reais de uma variável real.
    Considere em \( \mathcal{F} ( \R; \R ) \) o subespaço
    \begin{equation*}
        S \defeq \vspan \left\{ e^{ 2x } \sen x, e^{ 2x } \cos x, e^{ 2x } \right\}
    .\end{equation*}
    e o operador linear \( D : S \to S \) definido por \( D ( f ) = f' \).
    Considere, ainda, as funções \( f_{ 1 } ( x ) = e^{ 2x } \sen x, f_{ 2 } ( x ) = e^{ 2x } \cos x \) e \( f_{ 3 } ( x ) = e^{ 2x } \) em \( \mathcal{F} ( \R; \R ) \).
    Determine:
    \begin{enumerate}[label=(\alph*)]
        \item a matriz de \( D \) em relação à base \( \mathcal{B} = \left\{ f_{ 1 }, f_{ 2 }, f_{ 3 } \right\} \).
            Lembre-se de que, dada a base \( \mathcal{B} \), podemos enxergar os elementos de \(  \) como vetores em \( \R^{ 3 } \).
            Por exemplo:
            \begin{equation*}
                ( 1, 2, 3 )_{ \mathcal{B} } = f_{ 1 } + 2f_{ 2 } + 3f_{ 3 }
            .\end{equation*}
        \item os autovalores de \( D \) e as funções de \( S \) que são autovetores de \( D \).
    \end{enumerate}

\begin{sol}

    % Minha solução genial
    
\end{sol}
\end{enumerate}
\end{document} 
