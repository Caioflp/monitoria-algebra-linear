\documentclass[leqno]{article}

\usepackage[brazil]{babel}% \usepackage[latin1]{inputenc}
\usepackage{a4wide}
\setlength{\oddsidemargin}{-0.2in}
% % \setlength{\oddsidemargin}{0.2in}
\setlength{\evensidemargin}{-0.2in}
% % \setlength{\evensidemargin}{0.5in}
% % \setlength{\textwidth}{5.5in}
\setlength{\textwidth}{6.5in}
\setlength{\topmargin}{-1.2in}
\setlength{\textheight}{10in}
\usepackage[]{amsfonts} \usepackage[]{amsmath}
\usepackage[]{amssymb} \usepackage[]{latexsym}
\usepackage{graphicx,color} \usepackage{amsthm}
\usepackage{mathrsfs} \usepackage{url}
\usepackage{cancel} \usepackage{enumerate}
\usepackage{xifthen} \usepackage{tikz}
\usepackage{mathtools} 
\usepackage{hyperref}
\usetikzlibrary{automata,arrows,positioning,calc}

%% Packages

\usepackage{mathtools, amsthm}
% \usepackage{tikz}
% \usetikzlibrary{positioning}
% \usepackage{csquotes}
% \usepackage{hyperref}

%% Environments

\theoremstyle{plain} % default
\newtheorem{teo}{Teorema}[section]
\newtheorem*{teo*}{Teorema}
\newtheorem{lem}{Lema}[section]
\newtheorem{prop}{Proposição}[section]
\newtheorem{cor}[teo]{Corolário}
\newtheorem*{axiom}{Axioma}

\newtheorem*{TAU}{Teorema da Aproximação Universal}
\newtheorem*{Riesz}{Teorema da Representação de Riesz}

\theoremstyle{definition}
\newtheorem{defn}{Definição}[section]
\newtheorem{conj}{Conjectura}[section]
\newtheorem{exmp}{Exemplo}[section]
\newtheorem{rem}{Observação}[section]
\newtheorem*{rem*}{Observação}

\theoremstyle{remark}
\newtheorem*{note}{Nota}
\newtheorem{case}{Caso}


% Macros

\renewcommand{\vec}[1]{\mathbf{#1}}
\renewcommand{\Re}{\text{Re}}

\newcommand{\K}{\mathbb{K}}
\newcommand{\I}{\mathbb{I}}

\DeclarePairedDelimiter{\dotprod}{\langle}{\rangle}

\DeclareMathOperator{\rk}{rk}
\DeclareMathOperator{\intt}{int}
\DeclareMathOperator{\diam}{diam}
\DeclareMathOperator{\rref}{rref}
\DeclareMathOperator{\vspan}{span}
\DeclareMathOperator{\proj}{proj}
\DeclareMathOperator{\lin}{Lin}
\DeclareMathOperator{\supp}{supp}

\newcommand{\func}[3]{#1 : #2 \rightarrow #3}
\newcommand{\R}{\mathbb{R}}
\newcommand{\Z}{\mathbb{Z}}
\newcommand{\N}{\mathbb{N}}
\newcommand{\Q}{\mathbb{Q}}
\newcommand{\rr}{R_{r}}
\newcommand{\tq}{ : }
\newcommand{\mdc}{\text{mdc}}
\newcommand{\mmc}{\text{mmc}}
\newcommand{\defeq}{\vcentcolon=}
\newcommand{\comp}{\mathscr{C}}


%% Upper and Lower Integrals
\newcommand{\loint}[4]{
    \lefteqn{\int_{ #1 }^{ #2 } #3}\lefteqn{\hspace{0.0ex}\rule[-2.25ex]{1.1ex}{.05ex}} \phantom{\int_{ #1 }^{ #2 } #3}\mathrm{d}#4
}
\newcommand{\upint}[4]{
    \lefteqn{\int_{ #1 }^{ #2 } #3 \ }\lefteqn{\hspace{1.2ex}\rule[ 3.35ex]{1.1ex}{.05ex}} \phantom{\int_{ #1 }^{ #2 } #3 \ }\mathrm{d}#4
}

\DeclarePairedDelimiter\ceil{\lceil}{\rceil}
\DeclarePairedDelimiter\floor{\lfloor}{\rfloor}
\DeclarePairedDelimiter\abs{\lvert}{\rvert}%
\DeclarePairedDelimiter\norm{\lVert}{\rVert}%
\DeclareMathOperator{\sen}{sen}

% Swap the definition of \abs* and \norm*, so that \abs
% and \norm resizes the size of the brackets, and the 
% starred version does not.

\makeatletter
\let\oldabs\abs
\def\abs{\@ifstar{\oldabs}{\oldabs*}}
%
\let\oldnorm\norm
\def\norm{\@ifstar{\oldnorm}{\oldnorm*}}
\makeatother


\numberwithin{equation}{section}

\setlength{\parindent}{12 pt}


%% \newtheorem{teo}{Teorema}[section] \newtheorem*{teo*}{Teorema}
%% \newtheorem{prop}[teo]{Proposição} \newtheorem*{prop*}{Proposição}
%% \newtheorem{lema}[teo]{Lemma} \newtheorem*{lema*}{Lema}
%% \newtheorem{cor}[teo]{Corolário} \newtheorem*{cor*}{Corolário}

%% \theoremstyle{definition}
%% \newtheorem{defi}[teo]{Definição} \newtheorem*{defi*}{Definição}
%% \newtheorem{exem}[teo]{Exemplo} \newtheorem*{exem*}{Exemplo}
%% \newtheorem{obs}[teo]{Observação} \newtheorem*{obs*}{Observação}
%% \newtheorem*{hipo}{Hipóteses}
%% \newtheorem*{nota}{Notação}

\newcommand{\ds}{\displaystyle} \newcommand{\nl}{\newline}
\newcommand{\eps}{\varepsilon} \newcommand{\ssty}{\scriptstyle}
\newcommand{\bE}{\mathbb{E}}
\newcommand{\cB}{\mathcal{B}}
\newcommand{\cF}{\mathcal{F}}
\newcommand{\cA}{\mathcal{A}}
\newcommand{\cM}{\mathcal{M}}
\newcommand{\cD}{\mathcal{D}}
\newcommand{\cN}{\mathcal{N}}
\newcommand{\cL}{\mathcal{L}}
\newcommand{\cLN}{\mathcal{LN}}
\newcommand{\bP}{\mathbb{P}}
\newcommand{\bQ}{\mathbb{Q}}
\newcommand{\bN}{\mathbb{N}}
\newcommand{\bR}{\mathbb{R}}
\newcommand{\bZ}{\mathbb{Z}} 

\newcommand{\bfw}{\mathbf{w}}
\newcommand{\bfv}{\mathbf{v}}
\newcommand{\bfu}{\mathbf{u}}
\newcommand{\bfz}{\mathbf{z}}
\newcommand{\bfe}{\mathbf{e}}


\newcommand{\bvecc}[2]{%
    \begin{bmatrix} #1 \\ #2  \end{bmatrix}
}
\newcommand{\bveccc}[3]{%
    \begin{bmatrix} #1 \\ #2 \\ #3  \end{bmatrix}
}

\newcommand{\bvecfour}[4]{%
    \begin{bmatrix} #1 \\ #2 \\ #3 \\ #4 \end{bmatrix}
}

\newenvironment{sol}
{
    \vspace{4mm}
    \noindent\textbf{Resolução:}
    \strut\newline
    \smallskip
    \hspace{-3.5mm}
}
{} 

\title{Álgebra Linear - Soluções da Lista de Exercícios 5}

\author{Caio Lins e Tiago Silva}
\date{\today} 

\begin{document}

\maketitle

\begin{enumerate}

    \item Explique porque essas afirmações são falsas

        \begin{enumerate}

            \item A solução completa é qualquer combinação linear de $x_p$ e $x_n$.

                \begin{sol} 
                    % escreva sua solução aqui.    
                \end{sol} 

            \item O sistema $Ax = b$ tem no máximo uma solução particular.

                \begin{sol} 
                    % escreva sua solução aqui.    
                \end{sol} 

            \item Se $A$ é inversível, não existe nenhuma solução $x_n$ no núcleo.

                \begin{sol} 
                    % escreva sua solução aqui.    
                \end{sol} 

        \end{enumerate}

    \item Sejam
        $$U = \begin{bmatrix} 
            1 & 2 & 3 \\
            0 & 0 & 4
            \end{bmatrix} \mbox{ e } c = \begin{bmatrix} 
            5 \\
            8
        \end{bmatrix}.$$

        Use a eliminação de Gauss-Jordan para reduzir as matrizes $[U \ 0]$ e $[U \ c]$ para $[R \ 0]$ e $[R \ d]$. Resolva $Rx = 0$ e $Rx = d$

        \begin{sol} 
            % escreva sua solução aqui.    
        \end{sol} 

    \item Suponha que $Ax = b$ e $Cx = b$ tenham as mesmas soluções (completas) para todo $b$. Podemos concluir que $A = C$?

        \begin{sol} 
            % escreva sua solução aqui.    
        \end{sol} 

    \item Ache o maior número possível de vetores linearmente independentes dentre os vetores:

        $$\bvecfour{1}{-1}{0}{0}, \ \bvecfour{1}{0}{-1}{0}, \ \bvecfour{1}{0}{0}{-1}, \ \bvecfour{0}{1}{-1}{0}, \ \bvecfour{0}{1}{0}{-1} \mbox{ e } \bvecfour{0}{0}{1}{-1}$$

        \begin{sol} 
            % escreva sua solução aqui.    
        \end{sol} 

    \item Ache uma base para o plano $x - 2y + 3z = 0$ em $\bR^3$. Encontre então uma base para a interseção desse plano com o plano $xy$. Ache ainda uma base para todos os vetores perpendiculares a esse plano.

        \begin{sol} 
            % escreva sua solução aqui.    
        \end{sol} 

    \item Ache (na sua forma mais simples) a matriz que é o produto das matrizes de posto 1 $\bfu \bfv^T$ e $\bfw \bfz^T$? Qual seu posto?

        \begin{sol} 
            % escreva sua solução aqui.    
        \end{sol} 

    \item Suponha que a coluna $j$ de $B$ é uma combinação linear das colunas anteriores de $B$. Mostre que a coluna $j$ de $AB$ é uma combinação linear das colunas anteriores de $AB$. Conclua que posto$(AB) \leq $ posto$(B)$.

        \begin{sol} 
            % escreva sua solução aqui.    
        \end{sol} 

    \item O item anterior nos dá posto$(B^T A^T) \leq $ posto$(A^T)$. É possível concluir que posto$(AB) \leq $ posto$(A)$?

        \begin{sol} 
            % escreva sua solução aqui.    
        \end{sol} 

    \item Suponha que $A$ e $B$ são matrizes quadradas e $AB = I$. Prove que posto$(A) = n$. Conclua que $B$ precisa ser a inversa (de ambos lados) de $A$. Então, $BA = I$.

        \begin{sol} 
            % escreva sua solução aqui.    
        \end{sol} 

    \item (\textit{Bônus}) Dado um espaço vetorial real \( V \), definimos o conjunto
        \begin{equation*}
            V^{ * } \defeq \left\{ f : V \to \R \mid f \text{ é linear} \right\}
        .\end{equation*}
        Ou seja, \( V^{ * } \) é o conjunto de todas as funções lineares entre \( V \) e \( \R \).
        Relembramos que uma função \( f : E \to F \), onde \( E \) e \( F \) são espaços vetoriais, é dita \textit{linear} se para todos \( \bfv, \bfw \in E \) e \( \alpha \in \R \) temos \( f ( \bfv + \bfw ) = f ( \bfv ) + f ( \bfw ) \) e \( f ( \alpha \bfv ) = \alpha f ( \bfv ) \).
        Chamamos \( V^{ * } \) de \textit{espaço dual} de \( V \).
        \begin{enumerate}
            \item Mostre que \( V^{ * } \) é um espaço vetorial.

                \begin{sol} 
                    Podemos definir a soma de funcionais (funcionais, pois o contradomínio é \( \R \)) de uma maneira natural.
                    Dadas \( f, g \in V^{ * } \), definimos
                    \begin{align*}
                        f + g : V &\to \R \\
                        \bfv &\mapsto f ( \bfv ) + g ( \bfv )
                    .\end{align*}
                \end{sol} 
                Com isso, a soma de dois funcionais lineares é um funcional linear .
                Além disso, definimos de maneira análoga a multiplicação de um funcional por um escalar.
                Dada \( f \in V^{ * } \) e \( \alpha \in \R \), definimos
                \begin{align*}
                    \alpha f : V &\to \R \\
                    \bfv &\mapsto \alpha f ( \bfv )
                .\end{align*}
                Portanto, a multiplicação de um funcional linear por um escalar é um funcional linear.

            \item Agora, seja \( V = \R^{ n } \).
                Mostre que existe uma bijeção \( \varphi : V^{ * } \to V \) tal que , para toda \( f \in V^{ * } \) e para todo \( \bfv \in V \), tenhamos
                \begin{equation*}
                    f(\bfv) = \dotprod{\varphi(f), \bfv}
                .\end{equation*}
                \textit{Dica}: Utilize a dimensão finita de \( \R^{ n } \) para expandir \( \bfv \) como uma combinação linear dos vetores da base canônica e aplique a linearidade de \( f \).

                \begin{sol} 
                    Denotanto por \( \bfe_{ i } \) o \( i \)-ésimo vetor da base canônica, dado \( \bfv \in V \) podemos escrever
                    \begin{equation*}
                        \bfv = \sum_{ i=1 }^{ n } \alpha_{ i } \bfe_{ i }
                    ,\end{equation*}
                    para alguns \( \alpha_{ i } \in \R \), \( i = 1, \dots, n \).
                    Com isso, se \( f \in V^{ * } \), pela sua linearidade podemos escrever
                    \begin{align*}
                        f ( \bfv )
                        &= f \left(
                            \sum_{ i=1 }^{ n } \alpha_{ i } \bfe_{ i }
                        \right) \\
                        &= \sum_{ i=1 }^{ n } f ( \alpha_{ i } \bfe_{ i } ) \\
                        &= \sum_{ i=1 }^{ n } \alpha_{ i } f ( \bfe_{ i } ) \\
                        &=
                        \begin{bmatrix}
                            f ( \bfe_{ 1 } ) & \cdots & f ( \bfe_{ n } )
                        \end{bmatrix}
                        \bfv
                    .\end{align*}
                    Perceba que, com isso, conhecendo os \( n \) valores \( f ( \bfe_{ i } ), \dots, f ( \bfe_{ n } ) \) podemos calcular \( f ( \bfv ) \) para qualquer vetor \( \bfv \in V \), realizando o produto interno
                    \begin{equation*}
                        \begin{bmatrix}
                            f ( \bfe_{ i } ) \\
                            \vdots \\
                            f ( \bfe_{ n } )
                        \end{bmatrix}^{ \transpose }
                        \bfv
                    .\end{equation*}
                    Sendo assim, podemos definir uma função
                    \begin{align*}
                        \varphi : V^{ * } &\to V \\
                        f &\mapsto \varphi ( f ) =
                        \begin{bmatrix}
                            f ( \bfe_{ 1 } ) \cdots f ( \bfe_{ n } )
                        \end{bmatrix}^{ \transpose }
                    .\end{align*}
                    Pela discussão anterior, já temos a identidade
                    \begin{equation*}
                        f ( \bfv ) = \dotprod{\varphi ( f ), \bfv}
                    \end{equation*}
                    para toda \( f \in V^{ * } \) e \( \bfv \in V \).
                    Resta mostrar que \( \varphi \) é uma bijeção.

                    Vamos mostrar, primeiro, que \( \varphi \) é sobrejetiva.
                    Dado \( \bfu =
                    \begin{bmatrix}
                        \beta_{ 1 } \cdots \beta_{ n }
                    \end{bmatrix}^{ \transpose } \in V\), devemos encontrar uma \( g \in V^{ * } \) tal que \( \varphi ( g ) = \bfu \).
                    Ora, definindo \( g \) por \( g ( \bfv ) = \bfu^{ \transpose } \bfv \) para todo \( \bfv \in V \), claramente \( g \in V^{ * } \) e, ainda, temos
                    \begin{equation*}
                        g ( \bfe_{ i } ) = \bfu^{ \transpose } \bfe_{ i } = \beta_{ i }
                    \end{equation*}
                    para todo \( i = 1, \dots, n \).
                    Logo, \( \varphi ( g ) = \bfu \).

                    Para mostrar a injetividade de \( \varphi \), 
                \end{sol} 
        \end{enumerate}
        Em dimensão infinita, esse resultado é conhecido como \href{https://en.wikipedia.org/wiki/Riesz_representation_theorem}{Teorema da Representação de Riesz}.
\end{enumerate}

\end{document} 
