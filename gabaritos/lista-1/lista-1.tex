\documentclass[leqno]{article}

\usepackage[brazil]{babel} %\usepackage[latin1]{inputenc}
\usepackage{a4wide}
\setlength{\oddsidemargin}{-0.2in}
% % \setlength{\oddsidemargin}{0.2in}
\setlength{\evensidemargin}{-0.2in}
% % \setlength{\evensidemargin}{0.5in}
% % \setlength{\textwidth}{5.5in}
\setlength{\textwidth}{6.5in}
\setlength{\topmargin}{-1.2in}
\setlength{\textheight}{10in}
\usepackage[]{amsfonts} \usepackage[]{amsmath}
\usepackage[]{amssymb} \usepackage[]{latexsym}
\usepackage{graphicx,color} \usepackage{amsthm}
\usepackage{mathrsfs} \usepackage{url}
\usepackage{cancel} \usepackage{enumerate}
\usepackage{xifthen} \usepackage{tikz} 
\usepackage{float, framed}  
\usepackage{enumitem}
\usetikzlibrary{automata,arrows,positioning,calc}

% Preambulo do Caio
%% Packages

\usepackage{mathtools, amsthm}
% \usepackage{tikz}
% \usetikzlibrary{positioning}
% \usepackage{csquotes}
% \usepackage{hyperref}

%% Environments

\theoremstyle{plain} % default
\newtheorem{teo}{Teorema}[section]
\newtheorem*{teo*}{Teorema}
\newtheorem{lem}{Lema}[section]
\newtheorem{prop}{Proposição}[section]
\newtheorem{cor}[teo]{Corolário}
\newtheorem*{axiom}{Axioma}

\newtheorem*{TAU}{Teorema da Aproximação Universal}
\newtheorem*{Riesz}{Teorema da Representação de Riesz}

\theoremstyle{definition}
\newtheorem{defn}{Definição}[section]
\newtheorem{conj}{Conjectura}[section]
\newtheorem{exmp}{Exemplo}[section]
\newtheorem{rem}{Observação}[section]
\newtheorem*{rem*}{Observação}

\theoremstyle{remark}
\newtheorem*{note}{Nota}
\newtheorem{case}{Caso}


% Macros

\renewcommand{\vec}[1]{\mathbf{#1}}
\renewcommand{\Re}{\text{Re}}

\newcommand{\K}{\mathbb{K}}
\newcommand{\I}{\mathbb{I}}

\DeclarePairedDelimiter{\dotprod}{\langle}{\rangle}

\DeclareMathOperator{\rk}{rk}
\DeclareMathOperator{\intt}{int}
\DeclareMathOperator{\diam}{diam}
\DeclareMathOperator{\rref}{rref}
\DeclareMathOperator{\vspan}{span}
\DeclareMathOperator{\proj}{proj}
\DeclareMathOperator{\lin}{Lin}
\DeclareMathOperator{\supp}{supp}

\newcommand{\func}[3]{#1 : #2 \rightarrow #3}
\newcommand{\R}{\mathbb{R}}
\newcommand{\Z}{\mathbb{Z}}
\newcommand{\N}{\mathbb{N}}
\newcommand{\Q}{\mathbb{Q}}
\newcommand{\rr}{R_{r}}
\newcommand{\tq}{ : }
\newcommand{\mdc}{\text{mdc}}
\newcommand{\mmc}{\text{mmc}}
\newcommand{\defeq}{\vcentcolon=}
\newcommand{\comp}{\mathscr{C}}


%% Upper and Lower Integrals
\newcommand{\loint}[4]{
    \lefteqn{\int_{ #1 }^{ #2 } #3}\lefteqn{\hspace{0.0ex}\rule[-2.25ex]{1.1ex}{.05ex}} \phantom{\int_{ #1 }^{ #2 } #3}\mathrm{d}#4
}
\newcommand{\upint}[4]{
    \lefteqn{\int_{ #1 }^{ #2 } #3 \ }\lefteqn{\hspace{1.2ex}\rule[ 3.35ex]{1.1ex}{.05ex}} \phantom{\int_{ #1 }^{ #2 } #3 \ }\mathrm{d}#4
}

\DeclarePairedDelimiter\ceil{\lceil}{\rceil}
\DeclarePairedDelimiter\floor{\lfloor}{\rfloor}
\DeclarePairedDelimiter\abs{\lvert}{\rvert}%
\DeclarePairedDelimiter\norm{\lVert}{\rVert}%
\DeclareMathOperator{\sen}{sen}

% Swap the definition of \abs* and \norm*, so that \abs
% and \norm resizes the size of the brackets, and the 
% starred version does not.

\makeatletter
\let\oldabs\abs
\def\abs{\@ifstar{\oldabs}{\oldabs*}}
%
\let\oldnorm\norm
\def\norm{\@ifstar{\oldnorm}{\oldnorm*}}
\makeatother


% \numberwithin{equation}{section}

\setlength{\parindent}{12 pt} 

\newfloat{Box}{h}{lob}[section]

% Changes tombstone
% \renewcommand{\qedsymbol}{\textcolor{white}{\rule{1.3ex}{1.3ex}}}

\newenvironment{sol}
{
    \vspace{4mm}
    \noindent\textbf{Resolução:}
    \strut\newline
    \smallskip
    \hspace{-3.5mm}
}
{}
\begin{document}

% \newtheorem{teo}{Teorema}[section] \newtheorem*{teo*}{Teorema}
% \newtheorem{prop}[teo]{Proposição} \newtheorem*{prop*}{Proposição}
% \newtheorem{lema}[teo]{Lemma} \newtheorem*{lema*}{Lema}
% \newtheorem{cor}[teo]{Corolário} \newtheorem*{cor*}{Corolário}
% 
% \theoremstyle{definition}
% \newtheorem{defi}[teo]{Definição} \newtheorem*{defi*}{Definição}
% \newtheorem{exem}[teo]{Exemplo} \newtheorem*{exem*}{Exemplo}
% \newtheorem{obs}[teo]{Observação} \newtheorem*{obs*}{Observação}
% \newtheorem*{hipo}{Hipóteses}
% \newtheorem*{nota}{Notação}
% \newtheorem*{sol}{Solução} 

\newcommand{\ds}{\displaystyle} \newcommand{\nl}{\newline}
\newcommand{\eps}{\varepsilon} \newcommand{\ssty}{\scriptstyle}
\newcommand{\bE}{\mathbb{E}}
\newcommand{\cB}{\mathcal{B}}
\newcommand{\cF}{\mathcal{F}}
\newcommand{\cA}{\mathcal{A}}
\newcommand{\cM}{\mathcal{M}}
\newcommand{\cD}{\mathcal{D}}
\newcommand{\cN}{\mathcal{N}}
\newcommand{\cL}{\mathcal{L}}
\newcommand{\cLN}{\mathcal{LN}}
\newcommand{\bP}{\mathbb{P}}
\newcommand{\bQ}{\mathbb{Q}}
\newcommand{\bN}{\mathbb{N}}
\newcommand{\bR}{\mathbb{R}}
\newcommand{\bZ}{\mathbb{Z}}

\newcommand{\bfw}{\mathbf{w}}
\newcommand{\bfv}{\mathbf{v}}
\newcommand{\bfu}{\mathbf{u}}
\newcommand{\bfb}{\mathbf{b}}
\newcommand{\bfx}{\mathbf{x}}
\newcommand{\bfa}{\mathbf{a}}

\newcommand{\bvecc}[2]{%
  \begin{bmatrix} #1 \\ #2  \end{bmatrix}
}
\newcommand{\bveccc}[3]{%
  \begin{bmatrix} #1 \\ #2 \\ #3  \end{bmatrix}
}


\title{Álgebra Linear - Soluções da Lista de Exercícios 1}

\author{Caio Lins e Tiago Silva}

\date{\today}

\maketitle

\begin{enumerate}

%\item Ache uma combinação linear de $\bfw_1$, $\bfw_2$ e $\bfw_3$ que dê o vetor zero:
%$$\bfw_1 = \bveccc{1}{2}{3}, \bfw_2 = \bveccc{4}{5}{6} \mbox{ e } \bfw_3 = \bveccc{7}{8}{9}.$$
%
%\item Multiplique $\begin{bmatrix}
%1 & 2 & 0 \\
%2 & 0 & 3 \\
%4 & 1 & 1
%\end{bmatrix} \bveccc{3}{-2}{1}$.

\item Quais condições para $y_1, y_2$ e $y_3$ fazem com que os pontos $(0, y_1)$, $(1, y_2)$ e $(2, y_3)$ caiam numa reta?

\begin{sol} 
    Vamos ver o que tem que acontecer para que \( (2, y_{ 3 }) \) esteja na mesma reta de \( (0, y_{ 1 }) \) e \( ( 1, y_{ 2 } ) \).
    Essa reta é dada pelo conjunto:
    \begin{equation*}
        \left\{ (0, y_{ 1 }) + \lambda (1, y_{ 2 } - y_{ 1 }) : \lambda \in \R \right\}
    ,\end{equation*}
    ou seja, pelos pontos da forma \( ( \lambda, y_{ 1 } + \lambda ( y_{ 2 } - y_{ 1 } )) \), onde \( \lambda \in \R \).
    Igualando isso a \( ( 2, y_{ 3 } ) \), obtemos \( \lambda = 2 \) e \( y_{ 3 } = y_{ 1 } + \lambda ( y_{ 2 } - y_{ 1 } ) \), o que implica \( y_{ 3 } = 2 y_{ 2 } - y_{ 1 } \).
\end{sol} 

\item Se $(a,b)$ é um múltiplo de $(c,d)$ e são todos não-zeros, mostre que $(a,c)$ é um múltiplo de $(b,d)$. O que isso nos diz sobre a matriz
$$A = \begin{bmatrix}
a & b \\
c & d
\end{bmatrix}?$$ 

\begin{sol} 
    Se \( ( a, b ) \) é múltiplo de \( ( c, d ) \) e são todos não nulos, então \( a = \lambda c \) e \( b = \lambda d \), com \( \lambda \in \R \backslash \left\{ 0 \right\} \).
    Com isso, temos
    \begin{align*}
        ( a, c ) &= ( \lambda c, c ) \\
                 &= c ( \lambda, 1 ) \\
                 &= \frac{ c }{ d } ( \lambda d, d ) \\
                 &= \frac{ c }{ d } ( b, d )
    .\end{align*}
    Logo, \( ( a, c ) \) é um múltiplo de \( ( b, d ) \).
    Pondo \( \alpha = \frac{ c }{ d } \), temos que a matriz \( A \) é singular, visto que
    \begin{equation*}
        \begin{bmatrix}
            a & b \\
            c & d
        \end{bmatrix}
        \begin{bmatrix}
            1 \\
            -\alpha
        \end{bmatrix}
        =
        \begin{bmatrix}
            a \\
            c
        \end{bmatrix}
        - \alpha
        \begin{bmatrix}
            b \\
            d
        \end{bmatrix}
        = 0
    .\end{equation*}
\end{sol} 

\item Se $\bfw$ e $\bfv$ são vetores unitários, calcule os produtos internos de (a) $\bfv$ e $-\bfv$; (b) $\bfv + \bfw$ e $\bfv - \bfw$; (c) $\bfv - 2\bfw$ e $\bfv + 2\bfw$.

\begin{sol} 
    \begin{enumerate}[label=(\alph*)]
        \item \( \dotprod{\vec{v}, \vec{-v}} = - \dotprod{\vec{v}, \vec{v}} = - \norm{ \bfv }^2 = -1 \)
        \item 
            \begin{align*}
                \dotprod{\bfv + \bfw, \bfv - \bfw}
                &= \dotprod{\bfv, \bfv} + \dotprod{\bfv, - \bfw} + \dotprod{\bfw, \bfv} + \dotprod{\bfw, - \bfw} \\
                &= \norm{ \bfv }^2 - \dotprod{\bfv, \bfw} + \dotprod{\bfv, \bfw} - \norm{ \bfw }^2 \\
                &= 1 - 1 \\
                &= 0
            .\end{align*}
        \item Analogamente, \( \dotprod{\bfv - 2\bfw, \bfv + 2\bfw} = \norm{ \bfv }^2 - \norm{ 2\bfw }^2 = \norm{ \bfv }^2 - 4 \norm{ \bfw }^2 = 1 - 4 = -3 \).
    \end{enumerate}
\end{sol} 

\item Se $\| \bfv \| = 5$ e $\| \bfw \| = 3$, quais são o menor e maior valores possíveis para $\|\bfv - \bfw\|$? E para $\bfv \cdot \bfw$?

\begin{sol} 
    \begin{teo*}[Segunda desigualdade triangular]
        Dados \( \bfv, \bfw \in \R^{ n } \), temos
        \begin{equation*}
            \abs{ \norm{ \bfv } - \norm{ \bfw } } \leq \norm{ \bfv - \bfw }
        .\end{equation*}
    \end{teo*}
    \begin{proof}
        Pela primeira desigualdade triangular, temos
        \begin{align*}
            \norm{ \bfv } &= \norm{ \bfv - \bfw + \bfw } \\
                          &\leq \norm{ \bfv - \bfw } + \norm{ \bfw }
        ,\end{align*}
        o que implica \( \norm{ \bfv } - \norm{ \bfw } \leq \norm{ \bfv - \bfw } \).
        Trocando os papeis de \( \bfv \) e \( \bfw \) temos \( \norm{ \bfw } - \norm{ \bfv } \leq \norm{ \bfw - \bfv } = \norm{ \bfv - \bfw } \).
        Sendo assim,
        \begin{equation*}
            - ( \norm{ \bfv - \bfw } ) \leq \norm{ \bfv } - \norm{ \bfw } \leq \norm{ \bfv - \bfw }
        ,\end{equation*}
        o que implica \( \abs{ \norm{ \bfv } - \norm{ \bfw } } \leq \norm{ \bfv - \bfw } \).
    \end{proof}
    \begin{rem*}
        Não era necessário provar essa desigualdade, nós apenas utilizaremos ela na solução do gabarito e achamos importante que vocês vissem a prova.
    \end{rem*}
    Pelas desigualdades triangulares, temos
    \begin{equation*}
        2 = \abs{ \norm{ \bfv } - \norm{ \bfw } } \leq \norm{ \bfv - \bfw } \leq \norm{ \bfv } + \norm{ \bfw } = 8
    .\end{equation*}
    Para provar que de fato esses são os valores mínimo e máximo que a expressão \( \norm{ \bfv - \bfw } \) pode alcançar, precisamos de mostrar que eles podem ser atingidos.
    Observe que se \( \bfv = ( 5, \dots, 0 ) \) e \( \bfw = ( 3, \dots, 0 ) \), entao \( \norm{ \bfv - \bfw } = 2 \).
    Entretanto, se \( \bfv = ( 5, \dots, 0 ) \) mas \( \bfw = ( -3, \dots, 0 ) \), então \( \norm{ \bfv - \bfw } = 8 \).

    Pela desigualdade de Cauchy-Schwarz, temos
    \begin{equation*}
        -15 = - \norm{ \bfv } \norm{ \bfw } \leq \dotprod{\bfv, \bfw} \leq \norm{ \bfv } \norm{ \bfw } = 15
    .\end{equation*}
    Para provar que esses valores de fato podem ser atingidos, basta considerar os mesmos dois exemplos dados anteriormente.
\end{sol} 

\item Considere o desenho dos vetores $\bfw$ e $\bfv$ abaixo. Hachure as regiões definidas pelas combinações lineares $c \bfv + d \bfw$ considerando as seguintes restrições: $c + d = 1$ (não necessariamente positivos), $c,d \in [0,1]$ e $c,d \geq 0$ (note que são três regiões distintas).

\begin{center}
 \begin{tikzpicture}[scale=0.7]
    \draw[->] (-1,0)--(5,0) node[right]{};
    \draw[->] (0,-1)--(0,5) node[above]{};
    
    \draw[black,-stealth,line width = 0.5mm] (0,0) -- (5,2) node[pos=0.5, anchor=north]{$\bfv$};
    
    \draw[black,-stealth,line width = 0.5mm] (0,0) -- (0.7,4)node[pos=0.5, anchor=west]{$\bfw$};

    \end{tikzpicture}
\end{center} 

\begin{sol} 
    \begin{enumerate}[label=(\alph*)]
        \item \( c + d = 1 \).

            Podemos escrever \( c = 1 - d \) e, com isso,
            \begin{align*}
                c \bfv + d \bfw &= ( 1 - d ) \bfv + d \bfw \\
                                &= \bfv - d \bfv + d \bfw \\
                                &= \bfv + d ( \bfw - \bfv )
            .\end{align*}
            Ou seja, o conjunto \( \left\{ c \bfv + d \bfw : c + d = 1 \right\} \) é a reta que passa pelos pontos \( \bfv \) e \( \bfw \).
            \begin{center}
             \begin{tikzpicture}[scale=0.7]
                \draw[->] (-1,0)--(5,0) node[right]{};
                \draw[->] (0,-1)--(0,5) node[above]{};
                
                \draw[black,-stealth,line width = 0.5mm] (0,0) -- (5,2) node[pos=0.5, anchor=north]{$\bfv$};
                \draw[black,-stealth,line width = 0.5mm] (0,0) -- (0.7,4)node[pos=0.5, anchor=west]{$\bfw$};

                \draw[red, line width=0.5mm, <->] (6.4333333,1.33333333)--(-0.7333333,4.66666) node[above]{};
                \end{tikzpicture}
            \end{center} 

        \item \( c, d \in [0, 1] \)

            A região formada será o paralelogramo de vértices \( 0, \bfv, \bfw \) e \( \bfv + \bfw \).
            \begin{center}
             \begin{tikzpicture}[scale=0.7]
                \draw[->] (-1,0)--(5,0) node[right]{};
                \draw[->] (0,-1)--(0,5) node[above]{};
                
                \filldraw [fill=pink, even odd rule] (0,0) -- (5,2) -- (5.7,6) -- (0.7,4) -- cycle;

                \draw[black,-stealth,line width = 0.5mm] (0,0) -- (5,2) node[pos=0.5, anchor=north]{$\bfv$};
                \draw[black,-stealth,line width = 0.5mm] (0,0) -- (0.7,4)node[pos=01, anchor=east]{$\bfw$};
                \draw[black,-stealth,line width = 0.5mm] (5,2) -- (5.7,6);
                \draw[black,-stealth,line width = 0.5mm] (0.7, 4) -- (5.7,6);
                \end{tikzpicture}
            \end{center} 

        \item \( c, d \geq 0 \)

            A região formada será a parcela do plano contida entre as semirretas \( \left\{ \lambda \bfv : \lambda \geq 0 \right\} \) e \( \left\{ \lambda \bfw : \lambda \geq 0 \right\} \).

            \begin{center}
             \begin{tikzpicture}[scale=0.7]
                \draw[->] (-1,0)--(5,0) node[right]{};
                \draw[->] (0,-1)--(0,5) node[above]{};

                \filldraw [fill=pink, even odd rule] (0,0) -- (14,5.6) -- (7,5.6) -- (0.98,5.6) -- cycle;
                
                \draw[black,-stealth,line width = 0.5mm] (0,0) -- (5,2) node[pos=0.5, anchor=north]{$\bfv$};
                \draw[black,-stealth,line width = 0.5mm] (0,0) -- (0.7,4)node[pos=01, anchor=east]{$\bfw$};

                \end{tikzpicture}
            \end{center} 
    \end{enumerate}
\end{sol} 

\item É possível que três vetores em $\bR^2$ tenham $\bfu \cdot \bfv < 0$, $\bfv \cdot \bfw < 0$ e $\bfu \cdot \bfw < 0$? Argumente.

\begin{sol} 
	 É possível.
     Intuitivamente, o produto interno entre dois vetores é negativo se eles ``apontam em direções opostas''.
     Matematicamente, como, em \( \R^{ 2 } \), \( \dotprod{\bfv, \bfw} = \norm{ \bfv } \norm{ \bfw } \cos \theta \), onde \( \theta \) é o menor ângulo entre \( \bfv \) e \( \bfw \), só temos \( \dotprod{\bfv, \bfw} < 0 \) se ambos são não nulos e \( \cos \theta < 0 \), ou seja, se \( \theta > \frac{ \pi }{ 2 } \).
     Logo, tomando \( \bfu, \bfw \) e \( \bfv \) como vetores unitários tais que quaisquer dois deles são separados por um ângulo de \( \frac{ 2 \pi }{ 3 } \) (ou \( 120^{ \circ } \)), como na figura abaixo, teremos necessariamente \( \dotprod{\bfu, \bfv} < 0, \dotprod{\bfu, \bfw} < 0 \) e \( \dotprod{\bfv, \bfw} < 0 \).
    \begin{center}
     \begin{tikzpicture}[scale=0.7]
        \draw[->] (-4,0)--(4,0) node[right]{};
        \draw[->] (0,-4)--(0,5) node[above]{};
        
        \draw[black,-stealth,line width = 0.5mm] (0,0) -- (0,3) node[pos=0.5, anchor=west]{$\bfv$};
        \draw[black,-stealth,line width = 0.5mm] (0,0) -- (2.5980,-1.5)node[pos=1, anchor=north]{$\bfw$};
        \draw[black,-stealth,line width = 0.5mm] (0,0) -- (-2.5980,-1.5)node[pos=1, anchor=north]{$\bfu$};
        \end{tikzpicture}
    \end{center} 
\end{sol} 


\item Sejam $x,y,z$ satisfazendo $x + y + z = 0$. Calcule o ângulo entre os vetores $(x,y,z)$ e $(z,x,y)$.

\begin{sol} 
    Seja \( \bfv = (x, y, z) \) e \( \bfw = (z, x, y) \).
    Vamos supor ambos vetores não nulos, pois caso contrário a pergunta não faz sentido.
    O ângulo entre \( \bfv \) e \( \bfw \) é definido como o número \( \theta \in [ 0, \pi ] \) tal que
    \begin{equation}
        \cos \theta = \frac{ \dotprod{\bfv, \bfw} }{ \norm{ \bfv } \norm{ \bfw } }
        \label{eq: cos}
    .\end{equation}
    Expandindo o lado direito de (\ref{eq: cos}), temos 
    \begin{align}
        \frac{ \dotprod{\bfv, \bfw} }{ \norm{ \bfv } \norm{ \bfw } }
        &= \frac{ xz + yx + zy }{ \sqrt{ x^2 + y^2 + z^2 } \sqrt{ z^2 + x^2 + y^2 } } \nonumber \\
        &= \frac{ xz + yx + zy }{ \sqrt{ (x^2 + y^2 + z^2)^2 } } \nonumber \\
        &= \frac{ xz + yx + zy }{ x^2 + y^2 + z^2 } \label{eq: expand_cos}
    .\end{align}
    Agora perceba que
    \begin{equation*}
        0 = ( x + y + z )^2 = x^2 + y^2 + z^2 + 2 ( xz + yx + zy )
    .\end{equation*}
    Logo,
    \begin{equation*}
        xz + yx + zy = - \frac{ 1 }{ 2 } ( x^2 + y^2 + z^2 )
    .\end{equation*}
    Substituindo em (\ref{eq: expand_cos}), ficamos com
    \begin{equation*}
        \cos \theta = - \frac{ 1 }{ 2 }
    ,\end{equation*}
    o que implica \( \theta = \frac{ 2 \pi }{ 3 } \).
\end{sol} 

\item Resolva o sistema linear abaixo:
$$\begin{bmatrix}
1 & 0 & 0\\
1 & 1 & 0\\
1 & 1 & 1
\end{bmatrix} \begin{bmatrix}
x_1\\
x_2\\
x_3
\end{bmatrix} = \begin{bmatrix}
b_1\\
b_2\\
b_3
\end{bmatrix}.$$
Escreva a solução $\bfx$ como uma matriz $A$ vezes o vetor $\bfb$.

\begin{sol} 
	 Vamos prosseguir por eliminação gaussiana na matriz aumentada do sistema.
     \begin{align*}
         \begin{bmatrix}
             1 & 0 & 0 & \mid & b_{ 1 } \\
             1 & 1 & 0 & \mid & b_{ 2 }\\
             1 & 1 & 1 & \mid & b_{ 3 }
         \end{bmatrix}
         &\xrightarrow{
             \begin{array}{l}
                 L_{ 2 } - L_{ 1 } \\
                 L_{ 3 } - L_{ 1 }
             \end{array}
         }
         \begin{bmatrix}
             1 & 0 & 0 & \mid & b_{ 1 } \\
             0 & 1 & 0 & \mid & b_{ 2 } - b_{ 1 } \\
             0 & 1 & 1 & \mid & b_{ 3 } - b_{ 1 }
         \end{bmatrix} \\
         &\xrightarrow{
             \begin{array}{l}
                 L_{ 3 } - L_{ 2 }
             \end{array}
         }
         \begin{bmatrix}
             1 & 0 & 0 & \mid & b_{ 1 } \\
             0 & 1 & 0 & \mid & b_{ 2 } - b_{ 1 } \\
             0 & 0 & 1 & \mid & b_{ 3 } - b_{ 2 }
         \end{bmatrix}
     .\end{align*}
     Assim,
     \begin{equation*}
         \bfx =
         \begin{bmatrix}
             b_{ 1 } \\
             b_{ 2 } - b_{ 1 } \\
             b_{ 3 } - b_{ 2 }
         \end{bmatrix}
     .\end{equation*}
\end{sol} 
Olhando cada linha de \( \bfx \) como uma combinação linear das entradas de \( \bfb \), podemos reescrevê-lo como
\begin{equation*}
    \bfx =
    \begin{bmatrix}
        1 & 0 & 0 \\
        -1 & 1 & 0 \\
        0 & -1 & 1
    \end{bmatrix}
    \bfb
.\end{equation*}

\item Repita o problema acima para a matriz:
$$\begin{bmatrix}
-1 & 1 & 0\\
0 & -1 & 1\\
0 & 0 & -1
\end{bmatrix}.$$

\begin{sol}
    Novamente, prosseguiremos por eliminação gaussiana na matriz aumentada do sistema.
    \begin{align*}
    \begin{bmatrix}
        -1 & 1  & 0  & \mid & b_{ 1 } \\
        0  & -1 & 1  & \mid & b_{ 2 } \\
        0  & 0  & -1 & \mid & b_{ 3 } \\
    \end{bmatrix}
    & \xrightarrow{ 
        \begin{array}{l}
            L_{ 1 } + L_{ 2 }
        \end{array}
    }
    \begin{bmatrix}
        -1 & 0  & 1  & \mid & b_{ 1 } + b_{ 2 } \\
        0  & -1 & 1  & \mid & b_{ 2 } \\
        0  & 0  & -1 & \mid & b_{ 3 } \\
    \end{bmatrix} \\
    &\xrightarrow{ 
        \begin{array}{l}
            L_{ 1 } + L_{ 3 } \\
            L_{ 2 } + L_{ 3 }
        \end{array}
    }
    \begin{bmatrix}
        -1 & 0  & 0  & \mid & b_{ 1 } + b_{ 2 } + b_{ 3 } \\
        0  & -1 & 0  & \mid & b_{ 2 } + b_{ 3 } \\
        0  & 0  & -1 & \mid & b_{ 3 } \\
    \end{bmatrix} \\
    & \xrightarrow{ 
        \begin{array}{l}
            ( -1 ) \cdot L_{ 1 } \\
            ( -1 ) \cdot L_{ 2 } \\
            ( -1 ) \cdot L_{ 3 }
        \end{array}
    }
    \begin{bmatrix}
        1 & 0  & 0  & \mid & - ( b_{ 1 } + b_{ 2 } + b_{ 3 } ) \\
        0  & 1 & 0  & \mid & - ( b_{ 2 } + b_{ 3 } ) \\
        0  & 0  & 1 & \mid & - ( b_{ 3 } ) \\
    \end{bmatrix} \\
    .\end{align*}
    Analogamente à questão anterior, temos
    \begin{equation*}
        \bfx =
        \begin{bmatrix}
            - ( b_{ 1 } + b_{ 2 } + b_{ 3 } ) \\
            - ( b_{ 2 } + b_{ 3 } ) \\
            - ( b_{ 3 } ) \\
        \end{bmatrix}
        =
        \begin{bmatrix}
            -1 & -1 & -1 \\
            0 & -1 & -1 \\
            0 & 0 & -1
        \end{bmatrix}
        \bfb
    .\end{equation*}
\end{sol}

\item Considere a equação de recorrência $-x_{i+1} + 2x_i - x_{i-1} = i$ para $i=1,2,3,4$ com $x_0 = x_5 = 0$. Escreva essas equações em notação matricial $A\bfx = \bfb$ e ache $\bfx$.

\begin{sol} 
    Vamos escrever cada uma dessas equações:
    \begin{equation*}
        \begin{cases}
            -x_{ 2 } + 2 x_{ 1 } - x_{ 0 } = 1 \\
            -x_{ 3 } + 2 x_{ 2 } - x_{ 1 } = 2 \\
            -x_{ 4 } + 2 x_{ 3 } - x_{ 2 } = 3 \\
            -x_{ 5 } + 2 x_{ 4 } - x_{ 3 } = 4 \\
        \end{cases}
    .\end{equation*}
    Substituindo os valores de \( x_{ 0 } \) e \( x_{ 5 } \) e reordenando os termos, ficamos com o sistema:
    \begin{equation*}
        \begin{cases}
            2 x_{ 1 } - x_{ 2 }  = 1 \\
            - x_{ 1 } + 2 x_{ 2 } - x_{ 3 } = 2 \\
            - x_{ 2 } + 2 x_{ 3 } - x_{ 4 } = 3 \\
            - x_{ 3 } + 2 x_{ 4 } = 4 \\
        \end{cases}
    ,\end{equation*}
    o qual, escrito matricialmente, tem a seguinte forma:
    \begin{equation*}
        \begin{bmatrix}
            2 & -1 & 0 & 0 \\
            -1 & 2 & -1 & 0 \\
            0 & -1 & 2 & -1 \\
            0 & 0 & -1 & 2
        \end{bmatrix}
        \begin{bmatrix}
            x_{ 1 } \\
            x_{ 2 } \\
            x_{ 3 } \\
            x_{ 4 } \\
        \end{bmatrix}
        =
        \begin{bmatrix}
            1 \\
            2 \\
            3 \\
            4
        \end{bmatrix}
    .\end{equation*}
    Resolvendo por eliminação gaussiana na matriz aumentada do sistema, ficamos com:
    \begin{align*}
        \begin{bmatrix}
            2 & -1 & 0 & 0  & \mid & 1 \\
            -1 & 2 & -1 & 0 & \mid & 2 \\
            0 & -1 & 2 & -1 & \mid & 3 \\
            0 & 0 & -1 & 2  & \mid & 4
        \end{bmatrix}
        & \xrightarrow{
            \begin{array}{l}
                L_{ 2 } + \frac{ 1 }{ 2 } L_{ 1 }
            \end{array}
        }
        \begin{bmatrix}
            2 & -1 & 0 & 0  & \mid & 1 \\
            0 & 3/2 & -1 & 0 & \mid & 5/2 \\
            0 & -1 & 2 & -1 & \mid & 3 \\
            0 & 0 & -1 & 2  & \mid & 4
        \end{bmatrix} \\
        & \xrightarrow{
            \begin{array}{l}
                L_{ 1 } + \frac{ 2 }{ 3 }L_{ 2 } \\
                L_{ 3 } + \frac{ 2 }{ 3 }L_{ 2 }
            \end{array}
        }
        \begin{bmatrix}
            2 & 0 & -2/3 & 0  & \mid & 8/3 \\
            0 & 3/2 & -1 & 0 & \mid & 5/2 \\
            0 & 0 & 4/3 & -1 & \mid & 14/3 \\
            0 & 0 & -1 & 2  & \mid & 4
        \end{bmatrix} \\
        & \xrightarrow{
            \begin{array}{l}
                L_{ 1 } + \frac{ 1 }{ 2 } L_{ 3 } \\
                L_{ 2 } + \frac{ 3 }{ 4 } L_{ 3 } \\
                L_{ 4 } + \frac{ 3 }{ 4 } L_{ 3 }
            \end{array}
        }
        \begin{bmatrix}
            2 & 0 & 0 & -1/2  & \mid & 5 \\
            0 & 3/2 & 0 & -3/4 & \mid & 6 \\
            0 & 0 & 4/3 & -1 & \mid & 14/3 \\
            0 & 0 & 0 & 5/4  & \mid & 15/2
        \end{bmatrix} \\
        & \xrightarrow{
            \begin{array}{l}
                1/2 \cdot L_{ 1 } \\
                2/3 \cdot L_{ 2 } \\
                3/4 \cdot L_{ 3 } \\
                4/5 \cdot L_{ 4 }
            \end{array}
        }
        \begin{bmatrix}
            1 & 0 & 0 & -1/4  & \mid & 5/2 \\
            0 & 1 & 0 & -1/2 & \mid & 4 \\
            0 & 0 & 1 & -3/4 & \mid & 7/2 \\
            0 & 0 & 0 & 1  & \mid & 6
        \end{bmatrix} \\
        & \xrightarrow{
            \begin{array}{l}
                L_{ 1 } + \frac{ 1 }{ 4 } L_{ 4 } \\
                L_{ 2 } + \frac{ 1 }{ 2 } L_{ 4 } \\
                L_{ 3 } + \frac{ 3 }{ 4 } L_{ 4 }
            \end{array}
        }
        \begin{bmatrix}
            1 & 0 & 0 & 0  & \mid & 4 \\
            0 & 1 & 0 & 0 & \mid & 7 \\
            0 & 0 & 1 & 0 & \mid & 8 \\
            0 & 0 & 0 & 1  & \mid & 6
        \end{bmatrix}
    .\end{align*}
    Portanto,
    \begin{equation*}
        \begin{bmatrix}
            x_{ 1 } \\
            x_{ 2 } \\
            x_{ 3 } \\
            x_{ 4 } \\
        \end{bmatrix}
        =
        \begin{bmatrix}
            4 \\
            7 \\
            8 \\
            6
        \end{bmatrix}
    .\end{equation*}
\end{sol} 

\newpage

\item (Bônus) Use o seguinte código em \texttt{numpy} para gerar um vetor aleatório $\bfv=$\texttt{numpy.random.normal(size
\\=[2,1])} em $\bR^3$. Fazendo $\bfu = \bfv/\|\bfv\|$ criamos então um vetor unitário aleatório. Crie 30 outros vetores unitários aleatórios $\bfu_j$ (use \texttt{numpy.random.normal(size
=[2,30])}). Calcule a média dos produtos internos $|\bfu \cdot \bfu_j|$ e compare com o valor exato $\frac{1}{\pi}\int_0^\pi |\cos \theta| d\theta = \frac{2}{\pi}$.

\begin{rem*}
    Aqui não há uma ``solução'', mas é interessante perceber intuitivamente por que isso ocorre.
    Como todos os vetores são unitários, temos que \( \abs{ \dotprod{\bfu, \bfu_{ j }} } = \abs{ \cos \theta } \), onde \( \theta \in [0, \pi] \) é o ângulo entre os vetores \( \bfu \) e \( \bfu_{ j } \).
    Como esses vetores são escolhidos aleatoriamente, o que estamos fazendo na verdade é calcular a média dos módulos dos cossenos de \( 30 \) ângulos escolhidos aleatoriamente entre \( [0, \pi] \).
    Quando o número de amostragens cresce, essa média deve se aproximar da ``média'' da função \( \abs{ \cos \theta } \) no intervalo \( [0, \pi] \), dada pela integral fornecida.
    A realidade, entretanto, é mais complicada que isso.
    O que acontece é que a distribuição de probabilidade do ângulo entre cada vetor \( \bfu_{ j } \) e o vetor \( \bfu \) é uniforme no intervalo \( [0, \pi] \), sendo que esse é um fato verificado apenas em dimensão \( 2 \).
    Com isso, pela Lei dos Grandes Números, que vocês estudarão em Teoria da Probabilidade, a média dos valores \( \abs{ \dotprod{\bfu, \bfu_{ j }} } \) deve se aproximar do valor esperado\footnotemark da variável aleatória \( Y = \abs{ \cos(\Theta) } \), onde \( \Theta \) tem distribuição uniforme em \( [0, \pi] \).
    Esse valor esperado é dado pela integral fornecida.

    \footnotetext{Intuitivamente, o valor esperado de uma variável aleatória é uma média ponderada dos valores que ela pode assumir, utilizando como pesos as probabilidades dela assumir cada valor.
    Por exemplo, suponha que você jogou uma moeda justa para cima, e seja \( X \) uma variável aleatória que é igual a \( 0 \) se caiu cara e \( 1 \) se caiu coroa.
Então o valor esperado de \( X \) é dado por
\begin{equation*}
    \mathbb{E}[X] = 1 \cdot \mathbb{P}(X = 1) + 0 \cdot \mathbb{P}(X = 0)
                  = 1 \cdot \frac{ 1 }{ 2 } + 0 \cdot \frac{ 1 }{ 2 }
                  = \frac{ 1 }{ 2 }
.\end{equation*}
    No caso em que a variável aleatória assume valores num contínuo, seu valor esperado é dado por uma integral.
}
\end{rem*}
\end{enumerate}
\end{document} 
