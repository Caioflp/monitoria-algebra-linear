\documentclass[leqno]{article}

\usepackage[brazil]{babel} % \usepackage[latin1]{inputenc}
\usepackage{a4wide}
\setlength{\oddsidemargin}{-0.2in}
% % \setlength{\oddsidemargin}{0.2in}
\setlength{\evensidemargin}{-0.2in}
% % \setlength{\evensidemargin}{0.5in}
% % \setlength{\textwidth}{5.5in}
\setlength{\textwidth}{6.5in}
\setlength{\topmargin}{-1.2in}
\setlength{\textheight}{10in}
\usepackage[]{amsfonts} \usepackage[]{amsmath}
\usepackage[]{amssymb} \usepackage[]{latexsym}
\usepackage{graphicx,color} \usepackage{amsthm}
\usepackage{mathrsfs} \usepackage{url}
\usepackage{cancel} \usepackage{enumerate}
\usepackage{xifthen} \usepackage{tikz}
\usepackage{mathtools} 
\usepackage{hyperref}
\hypersetup{
    colorlinks=true,
    urlcolor=blue
}
\usetikzlibrary{automata,arrows,positioning,calc}
% %% Packages

\usepackage{mathtools, amsthm}
% \usepackage{tikz}
% \usetikzlibrary{positioning}
% \usepackage{csquotes}
% \usepackage{hyperref}

%% Environments

\theoremstyle{plain} % default
\newtheorem{teo}{Teorema}[section]
\newtheorem*{teo*}{Teorema}
\newtheorem{lem}{Lema}[section]
\newtheorem{prop}{Proposição}[section]
\newtheorem{cor}[teo]{Corolário}
\newtheorem*{axiom}{Axioma}

\newtheorem*{TAU}{Teorema da Aproximação Universal}
\newtheorem*{Riesz}{Teorema da Representação de Riesz}

\theoremstyle{definition}
\newtheorem{defn}{Definição}[section]
\newtheorem{conj}{Conjectura}[section]
\newtheorem{exmp}{Exemplo}[section]
\newtheorem{rem}{Observação}[section]
\newtheorem*{rem*}{Observação}

\theoremstyle{remark}
\newtheorem*{note}{Nota}
\newtheorem{case}{Caso}


% Macros

\renewcommand{\vec}[1]{\mathbf{#1}}
\renewcommand{\Re}{\text{Re}}

\newcommand{\K}{\mathbb{K}}
\newcommand{\I}{\mathbb{I}}

\DeclarePairedDelimiter{\dotprod}{\langle}{\rangle}

\DeclareMathOperator{\rk}{rk}
\DeclareMathOperator{\intt}{int}
\DeclareMathOperator{\diam}{diam}
\DeclareMathOperator{\rref}{rref}
\DeclareMathOperator{\vspan}{span}
\DeclareMathOperator{\proj}{proj}
\DeclareMathOperator{\lin}{Lin}
\DeclareMathOperator{\supp}{supp}

\newcommand{\func}[3]{#1 : #2 \rightarrow #3}
\newcommand{\R}{\mathbb{R}}
\newcommand{\Z}{\mathbb{Z}}
\newcommand{\N}{\mathbb{N}}
\newcommand{\Q}{\mathbb{Q}}
\newcommand{\rr}{R_{r}}
\newcommand{\tq}{ : }
\newcommand{\mdc}{\text{mdc}}
\newcommand{\mmc}{\text{mmc}}
\newcommand{\defeq}{\vcentcolon=}
\newcommand{\comp}{\mathscr{C}}


%% Upper and Lower Integrals
\newcommand{\loint}[4]{
    \lefteqn{\int_{ #1 }^{ #2 } #3}\lefteqn{\hspace{0.0ex}\rule[-2.25ex]{1.1ex}{.05ex}} \phantom{\int_{ #1 }^{ #2 } #3}\mathrm{d}#4
}
\newcommand{\upint}[4]{
    \lefteqn{\int_{ #1 }^{ #2 } #3 \ }\lefteqn{\hspace{1.2ex}\rule[ 3.35ex]{1.1ex}{.05ex}} \phantom{\int_{ #1 }^{ #2 } #3 \ }\mathrm{d}#4
}

\DeclarePairedDelimiter\ceil{\lceil}{\rceil}
\DeclarePairedDelimiter\floor{\lfloor}{\rfloor}
\DeclarePairedDelimiter\abs{\lvert}{\rvert}%
\DeclarePairedDelimiter\norm{\lVert}{\rVert}%
\DeclareMathOperator{\sen}{sen}

% Swap the definition of \abs* and \norm*, so that \abs
% and \norm resizes the size of the brackets, and the 
% starred version does not.

\makeatletter
\let\oldabs\abs
\def\abs{\@ifstar{\oldabs}{\oldabs*}}
%
\let\oldnorm\norm
\def\norm{\@ifstar{\oldnorm}{\oldnorm*}}
\makeatother


\newtheorem{teo}{Teorema}[section] 

\numberwithin{equation}{section}

\setlength{\parindent}{12 pt}

\begin{document}

% \newtheorem{teo}{Teorema}[section] \newtheorem*{teo*}{Teorema}
% \newtheorem{prop}[teo]{Proposição}
\newtheorem*{prop*}{Proposição}
\newtheorem{lema}[teo]{Lemma} \newtheorem*{lema*}{Lema}
% \newtheorem{cor}[teo]{Corolário}
\newtheorem*{cor*}{Corolário}

\theoremstyle{definition}
\newtheorem{defi}[teo]{Definição} \newtheorem*{defi*}{Definição}
\newtheorem{exem}[teo]{Exemplo} \newtheorem*{exem*}{Exemplo}
\newtheorem{obs}[teo]{Observação} \newtheorem*{obs*}{Observação}
\newtheorem*{hipo}{Hipóteses}
\newtheorem*{nota}{Notação}

\newcommand{\ds}{\displaystyle} \newcommand{\nl}{\newline}
\newcommand{\eps}{\varepsilon} \newcommand{\ssty}{\scriptstyle}
\newcommand{\bE}{\mathbb{E}}
\newcommand{\cB}{\mathcal{B}}
\newcommand{\cF}{\mathcal{F}}
\newcommand{\cA}{\mathcal{A}}
\newcommand{\cM}{\mathcal{M}}
\newcommand{\cD}{\mathcal{D}}
\newcommand{\cN}{\mathcal{N}}
\newcommand{\cL}{\mathcal{L}}
\newcommand{\cLN}{\mathcal{LN}}
\newcommand{\bP}{\mathbb{P}}
\newcommand{\bQ}{\mathbb{Q}}
\newcommand{\bN}{\mathbb{N}}
\newcommand{\R}{\mathbb{R}}
\newcommand{\bZ}{\mathbb{Z}}

\DeclarePairedDelimiter{\dotprod}{\langle}{\rangle} 
\newcommand{\defeq}{\vcentcolon=}
\newcommand{\bfw}{\mathbf{w}}
\newcommand{\bfv}{\mathbf{v}}
\newcommand{\bfu}{\mathbf{u}}

\newcommand{\bvecc}[2]{%
    \begin{bmatrix} #1 \\ #2  \end{bmatrix}
}
\newcommand{\bveccc}[3]{%
    \begin{bmatrix} #1 \\ #2 \\ #3  \end{bmatrix}
}

\newenvironment{sol}
{
    \vspace{4mm}
    \noindent\textbf{Resolução:}
    \strut\newline
    \smallskip
    \hspace{-3.5mm}
}
{} 

\title{Álgebra Linear - Lista de Exercícios 4}

\author{escreva seu nome aqui}

\date{}

\maketitle

\begin{enumerate}

    \item Sejam $S$ e $T$ dois subespaços de um espaço vetorial $V$.

        \begin{enumerate}

            \item Defina $S + T = \{s + t \ ; \ s \in S \mbox{ e } t \in T\}$. Mostre que $S + T$ é um subespaço vetorial.
	
	\begin{sol} 
		% escreva sua solução aqui.  
	\end{sol} 

            \item Defina $S \cup T = \{x \ ; \ x \in S \mbox{ ou } x \in T\}$. Argumente que $S \cup T$ não é necessariamente um subespaço vetorial.
	
	\begin{sol} 
		% escreva sua solução aqui.  
	\end{sol} 

            \item Se $S$ e $T$ são retas no $\R^3$, o que é $S + T$ e $S \cup T$?
	
	\begin{sol} 
		% escreva sua solução aqui.  
	\end{sol} 
        \end{enumerate}

    \item Como o núcleo $N(C)$ é relacionado aos núcleos $N(A)$ e $N(B)$, onde $C = \begin{bmatrix}A \\ B \end{bmatrix}$?
	
    \begin{sol} 
    	% escreva sua solução aqui.  
    \end{sol} 

    \item Considere a matriz
        $$A = \begin{bmatrix} 
            1 & 5 & 7 & 9\\
            0 & 4 & 1 & 7 \\
            2 & -2 & 11 & -3
        \end{bmatrix}.$$

        \begin{enumerate}

            \item Ache a sua forma escalonada reduzida.
	    
	    \begin{sol} 
	    	% escreva sua solução aqui.  
	    \end{sol} 

            \item Qual é o posto dessa matriz?
	    
	    \begin{sol} 
	    	% escreva sua solução aqui.  
	    \end{sol} 

            \item Ache uma solução especial para a equação $Ax = 0$.
	    
	    \begin{sol} 
	    	% escreva sua solução aqui.  
	    \end{sol} 
        \end{enumerate}

    \item Ache a matrizes $A_1$ e $A_2$ (não triviais) tais que posto$(A_1B) = 1$ e posto$(A_2B) = 0$ para $B = \begin{bmatrix}1 & 1 \\ 1 & 1 \end{bmatrix}$.
	
    \begin{sol} 
    	% escreva sua solução aqui.  
    \end{sol} 

    \item Verdadeiro ou Falso:

        \begin{enumerate}

            \item O espaço das matrizes simétricas é subespaço.
	    
	    \begin{sol} 
	    	% escreva sua solução aqui.  
	    \end{sol} 

            \item O espaço das matrizes anti-simétricas é um subespaço.
	    
            \begin{sol} 
	    	% escreva sua solução aqui.  
	    \end{sol} 

            \item O espaço das matrizes não-simétricas ($A^T \neq A$) é um subespaço.
	    
	    \begin{sol} 
	    	% escreva sua solução aqui.  
	    \end{sol} 

        \end{enumerate}

    \item Se $A$ é $4\times 4$ e inversível, descreva todos os vetores no núcleo da matriz $B = \begin{bmatrix}A & A \end{bmatrix}$ (que é $4\times 8$).
    
    \begin{sol} 
    	% escreva sua solução aqui.  
    \end{sol} 

    \item Mostre por contra-exemplos que as seguintes afirmações são falsas em geral:

        \begin{enumerate}

            \item $A$ e $A^T$ tem os mesmo núcleos.
	    
	    \begin{sol} 
	    	% escreva sua solução aqui.  
	    \end{sol} 

            \item $A$ e $A^T$ tem as mesmas variáveis livres.
	    
	    \begin{sol} 
	   	% escreva sua solução aqui.  
	    \end{sol} 

            \item Se $R$ é a forma escalonada de $A$, então $R^T$ é a forma escalonada de $A$.
	    
	    \begin{sol} 
	    	% escreva sua solução aqui.  
	    \end{sol} 
        \end{enumerate}

    \item Construa uma matriz cujo espaço coluna contenha $(1,1,5)$ e $(0,3,1)$ e cujo núcleo contenha $(1,1,2)$.

    \begin{sol} 
    	% escreva sua solução aqui.  
    \end{sol} 

    \item Construa uma matriz cujo núcleo contenha todos os múltiplos de $(4,3,2,1)$.
    
    \begin{sol} 
    	% escreva sua solução aqui.  
    \end{sol} 

    \item (\textit{Bônus}) Dado um espaço vetorial real \( V \), definimos o conjunto
        \begin{equation*}
            V^{ * } \defeq \left\{ f : V \to \R \mid f \text{ é linear} \right\}
        .\end{equation*}
        Ou seja, \( V^{ * } \) é o conjunto de todas as funções lineares entre \( V \) e \( \R \).
        Relembramos que uma função \( f : E \to F \), onde \( E \) e \( F \) são espaços vetoriais, é dita \textit{linear} se para todos \( \bfv, \bfw \in E \) e \( \alpha \in \R \) temos \( f ( \bfv + \bfw ) = f ( \bfv ) + f ( \bfw ) \) e \( f ( \alpha \bfv ) = \alpha f ( \bfv ) \).
        Chamamos \( V^{ * } \) de \textit{espaço dual} de \( V \).
        \begin{enumerate}
            \item Mostre que \( V^{ * } \) é um espaço vetorial.
	    
	    \begin{sol} 
	    	% escreva sua solução aqui.  
	    \end{sol} 
            \item Agora, seja \( V = \R^{ n } \).
                Mostre que existe uma bijeção \( \varphi : V^{ * } \to V \) tal que , para toda \( f \in V^{ * } \) e para todo \( \bfv \in V \), tenhamos
                \begin{equation*}
                    f(\bfv) = \dotprod{\varphi(f), \bfv}
                .\end{equation*} 
                \textit{Dica}: Utilize a dimensão finita de \( \R^{ n } \) para expandir \( \bfv \) como uma combinação linear dos vetores da base canônica e aplique a linearidade de \( f \).
	    
	    \begin{sol} 
	    	% escreva sua solução aqui.  
	    \end{sol} 	
        \end{enumerate}
        Em dimensão infinita, esse resultado é conhecido como \href{https://en.wikipedia.org/wiki/Riesz_representation_theorem}{Teorema da Representação de Riesz}.
\end{enumerate}

\end{document} 
