\documentclass[leqno]{article}

\usepackage[brazil]{babel} 
\usepackage{a4wide}
\setlength{\oddsidemargin}{-0.2in}
% % \setlength{\oddsidemargin}{0.2in}
\setlength{\evensidemargin}{-0.2in}
% % \setlength{\evensidemargin}{0.5in}
% % \setlength{\textwidth}{5.5in}
\setlength{\textwidth}{6.5in}
\setlength{\topmargin}{-1.2in}
\setlength{\textheight}{10in}
\usepackage[]{amsfonts} \usepackage[]{amsmath}
\usepackage[]{amssymb} \usepackage[]{latexsym}
\usepackage{graphicx,color} \usepackage{amsthm}
\usepackage{mathrsfs} \usepackage{url}
\usepackage{cancel} \usepackage{enumerate}
\usepackage{xifthen} \usepackage{tikz}
\usetikzlibrary{automata,arrows,positioning,calc}

\numberwithin{equation}{section}

\setlength{\parindent}{12 pt}

\begin{document}

\newtheorem{teo}{Teorema}[section] \newtheorem*{teo*}{Teorema}
\newtheorem{prop}[teo]{Proposição} \newtheorem*{prop*}{Proposição}
\newtheorem{lema}[teo]{Lemma} \newtheorem*{lema*}{Lema}
\newtheorem{cor}[teo]{Corolário} \newtheorem*{cor*}{Corolário}

\theoremstyle{definition}
\newtheorem{defi}[teo]{Definição} \newtheorem*{defi*}{Definição}
\newtheorem{exem}[teo]{Exemplo} \newtheorem*{exem*}{Exemplo}
\newtheorem{obs}[teo]{Observação} \newtheorem*{obs*}{Observação}
\newtheorem*{hipo}{Hipóteses}
\newtheorem*{nota}{Notação}

\newcommand{\ds}{\displaystyle} \newcommand{\nl}{\newline}
\newcommand{\eps}{\varepsilon} \newcommand{\ssty}{\scriptstyle}
\newcommand{\bE}{\mathbb{E}}
\newcommand{\cB}{\mathcal{B}}
\newcommand{\cF}{\mathcal{F}}
\newcommand{\cA}{\mathcal{A}}
\newcommand{\cM}{\mathcal{M}}
\newcommand{\cD}{\mathcal{D}}
\newcommand{\cN}{\mathcal{N}}
\newcommand{\cL}{\mathcal{L}}
\newcommand{\cLN}{\mathcal{LN}}
\newcommand{\bP}{\mathbb{P}}
\newcommand{\bQ}{\mathbb{Q}}
\newcommand{\bN}{\mathbb{N}}
\newcommand{\bR}{\mathbb{R}}
\newcommand{\bZ}{\mathbb{Z}}

\newcommand{\bfw}{\mathbf{w}}
\newcommand{\bfv}{\mathbf{v}}
\newcommand{\bfu}{\mathbf{u}}
\newcommand{\bfx}{\mathbf{x}}
\newcommand{\bfb}{\mathbf{b}}

\newcommand{\bvecc}[2]{%
  \begin{bmatrix} #1 \\ #2  \end{bmatrix}
}
\newcommand{\bveccc}[3]{%
  \begin{bmatrix} #1 \\ #2 \\ #3  \end{bmatrix}
}

\newenvironment{sol}
{
    \vspace{4mm}
    \noindent\textbf{Resolução:}
    \strut\newline
    \smallskip
    \hspace{-3.5mm}
}
{}

\title{Álgebra Linear - Lista de Exercícios 6}

\author{escreva seu nome aqui}

\date{}

\maketitle

\begin{enumerate}

\item Seja $A$ uma matriz $m \times n$ com posto $r$. Suponha que existem $\bfb$ tais que $A \bfx = \bfb$ não tenha solução.

\begin{enumerate}

\item Escreva todas as desigualdade ($<$ e $\leq$) que os números $m,n$ e $r$ precisam satisfazer.

	\begin{sol} 
		% escreva sua solução aqui.    
	\end{sol} 

\item Como podemos concluir que $A^T \bfx = 0$ tem solução fora $\bfx = 0$?

	\begin{sol} 
		% escreva sua solução aqui.    
	\end{sol} 
\end{enumerate}

\item Sem calcular $A$ ache uma bases para os quatro espaços fundamentais:
$$A = \begin{bmatrix}
1 & 0 & 0 \\
6 & 1 & 0 \\
9 & 8 & 1
\end{bmatrix}
\begin{bmatrix}
1 & 2 & 3 & 4 \\
0 & 1 & 2 & 3 \\
0 & 0 & 1 & 2
\end{bmatrix}$$

\begin{sol} 
	% escreva sua solução aqui.    
\end{sol} 

\item Explique porque $v = (1, 0, -1)$ não pode ser uma linha de $A$ e estar também no seu núcleo.

\begin{sol} 
	% escreva sua solução aqui.    
\end{sol} 

\item A equação $A^T \bfx = \bfw$ tem solução quando $\bfw$ está em qual dos quatro subespaços? Quando a solução é única (condição sobre algum dos quatro subespaços)?

\begin{sol} 
	% escreva sua solução aqui.    
\end{sol} 

\item Seja $M$ o espaço de todas as matrizes $3 \times 3$. Seja
$$A = \begin{bmatrix}
1 & 0 & -1 \\
-1 & 1 & 0 \\
0 & -1 & 1
\end{bmatrix}$$
e note que $A \bveccc{1}{1}{1} = \bveccc{0}{0}{0}$.

\begin{enumerate}

\item Quais matrizes $X \in M$ satisfazem $AX = 0$?

	\begin{sol} 
		% escreva sua solução aqui.    
	\end{sol} 
\item Quais matrizes $Y \in M$ podem ser escritas como $Y = AX$, para algum $X \in M$?

	\begin{sol} 
		% escreva sua solução aqui.    
	\end{sol} 
\end{enumerate}

\item Sejam $A$ e $B$ matrizes $m \times n$ com os mesmos quatro subespaços fundamentais. Se ambas estão na sua forma escalonada reduzida, prove que $F$ e $G$ são iguais, onde:
$$A = \begin{bmatrix}
I & F \\
0 & 0
\end{bmatrix} \mbox{ e } B \begin{bmatrix}
I & G \\
0 & 0
\end{bmatrix}.$$

\begin{sol} 
	% escreva sua solução aqui.    
\end{sol} 
\end{enumerate}















\end{document} 
